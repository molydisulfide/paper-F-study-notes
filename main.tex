\documentclass{report}
\input{preamble}
\input{macros}
\input{letterfonts}
\usepackage{adforn}
\usepackage{graphicx}
\usepackage[T1]{fontenc}
\usepackage{lmodern}
\usepackage{float}
\usepackage{mathtools}
\setlength\parindent{0pt}
\newcommand{\n}{\newline}
\newcommand{\p}{\adforn{61} \ }
\newcommand{\s}{\adforn{74} \ }



\begin{document}
	
	\thispagestyle{empty}
	\mytitleb{Paper F Notes}{Jakub Jadwiszczak}{jadwiszj@tcd.ie}{2026}
	\newpage% or \cleardoublepage
	\tableofcontents

\chapter{Introduction to the Patent Cooperation Treaty --- PCT}
\section{Historical Notes and Some Quick Facts}

The \textbf{Patent Cooperation Treaty} (PCT) is an international patent law treaty, concluded in 1970. It provides a unified procedure for filing patent applications to protect inventions in each of its contracting states. \newline

\p   Signed on June 19, 1970 in Washington D.C.. Entered into force on January 24, 1978. The first application was filed on June 1, 1978.\newline

\p A patent application filed under the PCT is called an \underline{\textbf{international application}}, or \textbf{PCT application}. \n

\p No case law. No appeal body -- with exception of questions relating to unity of invention. \n

\p Rules have to cover all situations that might occur -- therefore are very detailed. \n

\p Contracting States have agreed to accept international filing date and the form and content of an international application having the effect of a national application, but they \textbf{have not limited the freedom to grant patents} to an \textbf{International Authority (IA)}. This runs in contrast to the EPC --- for example --- where that freedom has been ceded. \n

\p The international phase includes filing + search + publication + (optional) non-binding examination. No decision to grant. \n

\nt{Definition}{\textbf{``National phase'' for the purposes of PCT should be understood to mean ``national'' \underline{or} ``regional phase.''}} 
\vspace{5mm}
\p Any signatory of the Paris Convention may accede to the PCT. \n

\p The national office (NO) must approve form and content of the application as approved in the international phase. \n

\p The cost of entering national phase is similar to a direct national application. Further search + examination may be carried out in the national phase, but the use of the \textbf{international search report (ISR)} and examination may result in a fee reduction at national phase. \n

\p NO will grant patent on initial application with the same effect as on a direct national application. \n

\p PCT is administered by the Word Intellectual Property Organization (WIPO), which is a UN agency. The International Bureau (IB) of WIPO carries out admin for the International Patent Cooperation Union that the Contracting States make up. \n

\p The international phase is carried out before International Authorities. There are > 120 of them and > 20 can act as \textbf{International Search Authorities (ISAs)} or \textbf{International Preliminary Examination Authorities (IPEAs)}.  \n

\p After the international phase, all NOs act as so-called \textbf{designated Offices} (dO) or \textbf{elected Offices} (eO) [the latter in \textit{Chapter II}.]

\section{The Paris Convention for the Protection of Industrial Property}

\begin{center}
 \href{https://en.wikisource.org/wiki/Paris_Convention_for_the_Protection_of_Industrial_Property_(1883)}{\textbf{Full Text}}

\end{center}


\p Signed on March 20, 1883. \n

\begin{figure}[htbp] % h=here, t=top, b=bottom, p=page: Platzierungs-Optionen
  \centering
  \includegraphics[width=0.6\textwidth]{images/Fig_1.jpg} % Pfad und Dateiname anpassen
  \caption{--- \textbf{\textsc{PCT Hierarchy}}}
\end{figure}

\p Agreement between countries for mutual recognition of IP rights. Nationals of Signatory Countries enjoy the same rights in other States as nationals of those other States. \n

\p It secured \textbf{the right of priority} of a first filing in one State for subsequent applications in other States. \n

\p A \textbf{priority right} is a time-limited right triggered by the \textbf{first filing} of an application for a patent, industrial design, or trademark. It allows the applicant to file a \textbf{subsequent application} in another country that is effectively treated as if filed on the date of the first application, known as the \textbf{priority date}. To use this right, the applicant (\textbf{or their successor in title}) must \textbf{claim priority} in the subsequent application. \n

\p The priority period is \textbf{12 months} for patents and utility models (the \textbf{priority year}) and \textbf{6 months} for industrial designs and trademarks. In the original Paris Convention it was 6 months and 3 months, respectively. \n

\p For patents, this right is crucial because \textbf{novelty} and \textbf{inventive step} are assessed against prior art that was made public \textbf{before the priority date}, not the actual (later) filing date of the subsequent application. \n

\nt{Rationale}{\textbf{according to the EPO: \textit{``(...) basic purpose [of the right of priority] is to safeguard, for a limited period, the interests of a patent applicant in his endeavour to obtain international protection for his invention, thereby alleviating the negative consequences of the principle of territoriality in patent law.''}}} 
\vspace{5mm}

\p \textbf{\s Art. 19} of the Paris Convention allows for special agreements between Signatory Countries. The Paris Convention \underline{takes precedence} over laws of the Countries and over such special agreements. \n

\p EPC is a ``regional patent treaty'' in the sense of \textbf{Art. 19} of the Paris Convention, e.g., under \textbf{Art. 45 PCT}, a PCT applicant can obtain an \underline{\underline{EP patent}} by filing an initial international application. \n

\nt{Definition}{\textbf{\[
    \begin{dcases}
        \mathrm{Patent \ \underline{in} \ a \ state} & = \ \ \mathrm{\underline{national}}\\
        \mathrm{Patent \ \underline{for} \ a \ state} & = \ \ \mathrm{\underline{regional}} \\
    \end{dcases}
\]}} 
\vspace{5mm}

\p Under \textbf{Art. 45 PCT}, a PCT Applicant can obtain a European patent by filing a PCTa. \n

\p In case of conflict between PCT and EPC/national provisions, the PCT takes precedence. \n

\p A ``Euro--PCT'' application is a PCT application with EPO as dO or eO. \n

\p The PCT timeline for a PCTa claiming priority from a national application: 

\newpage
\begin{figure}[H] % h=here, t=top, b=bottom, p=page: Platzierungs-Optionen
  \centering
  \includegraphics[width=0.85\textwidth]{images/Fig_2.jpg} % Pfad und Dateiname anpassen
  \caption{--- \textbf{\textsc{PCT Timeline}}}
\end{figure}




\p The international phase includes: international search, publication and (optionally) international preliminary examination. The international phase has a \underline{Chapter I} (\textbf{Art. 3--30}; \textbf{R. 3--52}) and a \underline{Chapter II} (\textbf{Art. 31--42}; \textbf{R. 53--78}). These correspond to search + pub and IPE. \n

\p Each nO has a time limit for entering national phase (usually 30 or 31 months from the priority date). A PCTa can be simultaneously in the international phase and in the national phase for some jurisdictions. \n

\nt{Chapter I}{\textbf{Stage 1}: filing, accordance of date, fees, formal requirements. \newline
\textbf{Stage 2}: ISR with written opinion on Novelty, IS and industrial applicability. \newline
\textbf{Stage 3}: publication of PCTa + ISR @18 months after priority date. } 
\vspace{5mm}


\nt{Chapter II}{\textbf{Stage 1}: filing a \underline{\textbf{demand}}, i.e., a request for an IPE --- typically @22 months after priority date.\newline
\textbf{Stage 2}: can amend claims and discuss Novelty, IS with Examiner. \newline
\textbf{Stage 3}: receipt of the \underline{IPRP II} typically @28 months from priority date. } 
\vspace{5mm}

\p Applicant can pay for a Supplementary ISR carried out by a SISA (can be different to ISA of Chapter I, e.g., in order to find prior art in a specific language.) \n

\p If no IPE demanded, written opinion from ISR is converted into the International Preliminary Report on Patentability I (\underline{IPRP I}) with an English translation thereof + documents are communicated to the dO. Usually happens @30 months after priority date.\n

\p IPRP is not binding on the dO but is made available to the dO. dO then becomes the eO.\n

\p Only about 3\% of applications ever enter Chapter II. \n


\p \textsc{LU} is a notable exception for entering national phase -- 20 months. \n

\p PCT advantages over direct national application include: 

\begin{itemize}
 \item cost reduction,
 \item 30-month delay to check market developments, own product developments, etc.,
 \item patentability opinion to inform later actions,
 \item delay of formalities/payments right away.
\end{itemize}

\p If no priority claim, then \textbf{the priority date is the PCTa filing date}. \n

\p ``Agent'' is ``professional representative'' in EPC terminology. \n

\p Terminology for document copies:

\begin{itemize}
\centering
 \item \textbf{Home copy} --- stays with rO.
 \item \textbf{Record copy} --- sent to IB.
 \item \textbf{Search copy} --- sent to ISA.
\end{itemize}

\p Record copy is the \textbf{true} copy. \n

\p It is possible that scope of PCTa only includes \textit{certain territories} of a CS, e.g., a Euro--PCT for Denmark does not include the Faroe Islands. \n

\p States can join to form a single legal territory for patent law, e.g., Switzerland and Liechtenstein. \n

\newpage

\chapter{The International Application (\textsc{PCTa})}

\section{Basics}

\p Elements that a PCTa must contain:

\begin{itemize}
 \item a request (\textbf{Art. 4)},
 \item a description (\textbf{Art. 5)},
 \item one or more claims (\textbf{Art. 6)},
 \item where required, one or more drawings (\textbf{Art. 7)},
 \item an abstract (\textbf{Art. 8)}.
\end{itemize}

\p To accord a \textbf{filing date}, you need:

\begin{itemize}
 \item an indication that it is inteded as a PCTa, including a designation of a Contracting State and name of Applicant,
 \item a description,
 \item at least one claim.
\end{itemize}

\p The Abstract may be submitted later though. \n

\p On filing, the description, claims and drawings may be replaced by \underline{reference to a priority application,} from which the rO will copy them [\textbf{R. 4.18} \& \textbf{R. 20.6}].

\section{Art. 3(4)(i) \& Rule 12 --- Languages and Translations}

\subsection{Languages accepted by the rO}

\p A PCTa must be filed in a language prescribed by the rO. \textbf{Every} rO accepts at least one language, which is \textbf{both} a language accepted by the competent ISA \underline{and} a language of publication. Languages of publication are listed under \textbf{R. 48.3(a)}. If using the language of publication, no translation need be filed.  \n

\p To get a filing date, claims and description must be in a language accepted by that rO. For cases of mixed languages of claims/description, an invitation to correct this will be issued under [\textbf{R. 26.3\textit{ter}(a)}]. \n

\p Languages of the request and the description/claims may be different though, e.g., an English request and Dutch claims/description. \n

\p The rO may require a translation for the purposes of search or publication, e.g., \textsc{RO/EP} requires \textbf{FR/EN/DE} translation. \n

\p All rOs of EPC CSs have specified the EPO as their competent ISA. Search in Dutch is possible at the EPO if the application was filed in Dutch for historical reasons. \n

\p The language of the rO need not be the official language of that CS. See also national security considerations under \textbf{Art. 27.8}, which may become relevant in such a case. \n

\p If a nO as rO does not accept the application language, it will transmit the PCTa to the IB to act as the rO under [\textbf{R. 19.4}]. IB accepts \textbf{any language}. In that case, new fees will need to be paid to the IB, the fees already paid to the nO will be refunded --- \textbf{except the transmittal fee} --- and the Applicant has to send the priority document to the IB. \n

\subsection{Translations}

\p The Applicant must provide a translation if:

\begin{enumerate}
 \item the language of filing is not accepted by the ISA (or SISA);
 \item the language of the ISA is not a publication language.
\end{enumerate}

\p The translation must be provided within \textbf{1 month of filing at the rO} -- otherwise a late furnishing fee is also due (equal to \textbf{25 \% of the international filing fee}). \n

\p The translation must be in the language of the ISA, language of publication \underline{and} language of the rO. Sequence listing free text needs to be translated as well. \n

\p If no translation is provided, the PCTa will be declared by the rO to have been withdrawn . \n

\p The rO will invite a translation witin 1 month of filing, if not received at filing. The late submission deadline is \textbf{2 months after filing} or \textbf{1 month after the invitation} -- whichever expires \textbf{later}. \n

\p Nonetheless, a translation and payment received before 15 months after priority date will be considered as received in time. Furthermore, a translation and payment received \textbf{before the rO makes the declaration} is also considered as received on time. \n

\p Under [\textbf{R. 12.4}] --- where a PCTa has been filed in a language accepted by the ISA, which is not a language of publication:

\begin{itemize}
 \item the Applicant must file a translation to a language of publication within \textbf{14 months} of the priority date;
 \item the rO will send an invitation for the late translation within \textbf{16 months} of the priority date;
 \item if the translation and late payment fee are received within \textbf{17 months} of the priority date, it is considered on time; and
 \item if the translation and late payment fee are received before the rO \textbf{declares withdrawal}, it is still considered on time.
\end{itemize}

\p PCT has a provision for retroactive scope limitation due to an incorrect translation under [\textbf{Art. 46}].

\section{Fees}

\p On filing of a PCTa, three fees (in Swiss francs) become due at the rO:

\begin{enumerate}
 \item transmittal fee;
 \item international filing fee (including sheet fee if $>$ 30 pages;
 \item international search fee.
\end{enumerate}

\p There are no claims fees. They may be due for some offices in the national phase. \n

\p PCT has no provisions for methods of payment or according a date of payment. Each rO will have its own rules on that. \n

\p The \textbf{transmittal fee} [\textbf{R. 14}] --- includes checking the PCTa, transmitting a record copy to the IB and the search copy to the ISA. The time limit for receipt is \textbf{1 month from the date of receipt of the application by the rO} (not from the filing date). The filing date might be shifted due to missing elements, etc., so the fees deadlines are not dependent on the filing date. This deadline is also shifted if the rO sends the PCTa to the IB when the rO is not the Competent Authority. \n

\p The \textbf{international filing fee} [\textbf{R. 15}] --- includes publication, translation/s and communication to ISA/SISA/IPEA. Must be paid before formalities check of \textbf{Art. 14(1)(a)} and any resulting corrections. The 30 pages limit before any extra fees is calculated on the basis of documents as filed. Within \textbf{1 month of date of receipt of the PCTa} again (not the filing date).




\newpage

\chapter{Fillun Homework Questions}

\section{September 22, 2025}

\begin{enumerate}[label=\textbf{Question \arabic*}]

    \item % Question 1
    Two applicants wish to appoint an agent to file their international application.
    \begin{enumerate}[label=(\alph*)]
        \item Who can be appointed to act as an agent?
        \item How must the agent be appointed?
        \item Do all receiving Offices require the filing of a separate power of attorney?
        \item Does the EPO require the filing of a separate power of attorney?
    \end{enumerate}
    
    \vspace{1em} % Adds a little vertical space
    The applicants always work via the same patent attorney office for which a general power of attorney has been prepared.
    
    \begin{enumerate}[label=(\alph*), resume]
        \item Is it necessary to attach a copy of the general power of attorney to the Request [PCT/RO/101]?
        \item Does the EPO require filing a copy of the general power of attorney?
    \end{enumerate}

    \item % Question 2
    An international application is filed at the Japanese National Office. The applicant first mentioned in the Request [PCT/RO/101] is a Taiwanese national resident in Taiwan. The second applicant is a Korean national resident in Korea and the third applicant is a Japanese national resident in Japan. The three applicants have not appointed a common agent or a common representative.
    \begin{enumerate}[label=(\alph*)]
        \item Who will be considered to be the common representative of the applicants?
        \item Would your answer to question (a) have been different if the international application had been filed at the International Bureau?
        \item Which acts may not be performed by the representative in question (a)?
        \item Which acts may not be performed by an agent appointed by the representative in question (a)?
    \end{enumerate}

    \item % Question 3
    Today, 16 March 2020, the applicant discovers an obvious mistake in the description of his international application filed on Friday 9 March 2018 as a first filing.
    \begin{enumerate}[label=(\alph*)]
        \item Can this obvious mistake be corrected? If so, what are the conditions?
        \item Who is competent to rectify the obvious mistake? What if the mistake is only detected after a demand for international preliminary examination has been made?
        \item What parts of the international application are taken into account upon correcting the mistake?
        \item When at the latest can a request for rectification be filed?
        \item Would your answer have been different if the applicant had discovered an obvious mistake in the abstract of his international application?
    \end{enumerate}

    \item % Question 4
    Is it possible to file third party observations in relation to an international application during the international phase? If so, where and how can these be filed?

    \item % Question 5
    A Danish applicant filed an international application PCT-DK as a first filing on 25 May 2025 with the EPO as receiving Office. Due to cash-flow problems, no fees were paid upon filing. On 30 June 2025, the EPO issues an invitation to pay the missing fees together with a late-payment surcharge.

    \vspace{1em}
    Indicate True/False:
    \begin{enumerate}[label=(\alph*)]
        \item If the applicant has, in fact, paid the fees due at filing on 27 June 2025, the payment will be considered in time.
        \item The time limit to pay the missing fees together with late-payment surcharge expires on 1 September 2025.
        \item If the applicant pays the missing fees + surcharge one day after the time limit to do so has expired, this is too late and the EPO is obliged to declare under Art. 14(3) that PCT-DK is considered withdrawn.
    \end{enumerate}

    \item % Question 6
    What happens if the applicant is a resident or national of one of the PCT Contracting States but files the international application with a "non-competent" receiving Office? What are the consequences for according the international filing date for such an application? Must an additional fee be paid?

\end{enumerate}

\begin{center}

\p \p \p
 
\end{center}

\begin{enumerate}[label=\textbf{Answer \arabic*}]

    \item % Question 1
    Two applicants wish to appoint an agent to file their international application.
    \begin{enumerate}[label=(\alph*)]
        \item Under \textbf{R. 90}: a person having the right to practice before the national Office with which the PCTa is filed \underline{or} having the right to practice before the IB as rO. The latter is governed by \textbf{Rule 83}, i.e., the agent must have a right to practice before the nO of the Contracting State of which the Applicant is a resident or national.
        \item Applicant enters and signs the name \textbf{and} address of the agent in the request \underline{or} the demand \underline{or} in a separate power of attorney (applicable to a \textbf{specific} PCTa) \underline{or} by a general power of attorney (applicable to \textbf{any} PCTa).
        \item No.
        \item No.

        \item Yes. It says so on the \textsc{[PCT/RO/101]} form.
        \item No. Waiver under \textbf{Rule 90.5(c)}. Two exceptions relating to suspicions as to the nature of the person performing acts apply.
    \end{enumerate}

    \item % Question 2
    An international application is filed at the Japanese National Office. The applicant first mentioned in the Request [PCT/RO/101] is a Taiwanese national resident in Taiwan. The second applicant is a Korean national resident in Korea and the third applicant is a Japanese national resident in Japan. The three applicants have not appointed a common agent or a common representative.
    \begin{enumerate}[label=(\alph*)]
        \item According to \textbf{Rule 90}, the first Applicant named in the Request entitled to file a PCTa with an rO will become the common representative. In this case, this should be the Korean national.
        \item No?
        \item The common representative may not sign any notice of withdrawal [under \textbf{R. 90\textit{bis}}], i.e.: withdrawal of application, designation, priority claim, supplementary search request, demand or election. Also, \textbf{not sure}, but it seems like he cannot perform an act in relation to only one Applicant or a subset of Applicants when there are multiple Applicants. 
        \item The common agent may not file a PCTa without signature of the Applicant/s [\textbf{R. 4.15}] and cannot make declarations as to entitlement on behalf of the Applicant/s [\textbf{R. 4.17}].
    \end{enumerate}

    \item % Question 3
    Today, 16 March 2020, the applicant discovers an obvious mistake in the description of his international application filed on Friday 9 March 2018 as a first filing.
    \begin{enumerate}[label=(\alph*)]
        \item Yes --- under \textbf{Rule 91.2}, there is a 26 month-deadline for correction of obvious mistakes. In this case, we have 24 months + 7 days.
        \item The Applicant (or his agent) are competent. After demand, the IPEA is the Competent Authority to be addressed. Notably, the dO/eO need not take rectification into account if processing/examination started prior to the notification [\textbf{R. 91.3(e)}] and the dO/eO may disregard an authorized notification if it finds it would not have authorized it itself had it been the Competent Authority [\textbf{R. 91.1(f)}].
        \item For mistakes in the claims, description or drawings (or corrections thereof), only the claims, description and drawings (and corrections thereof) will be taken into account [\textbf{R. 91.1(d)}]. 
        For mistakes in the Request (or corrections thereof), the contents of the whole PCTa including priority documents, corrections, etc., will be taken into account [\textbf{R. 91.1(e)}].
        \item 26 months after the priority date.
        \item Yes. Abstract mistakes may not be corrected under \textbf{Rule 91.1(g)}. They may be corrected under [\textbf{R. 38.3}] within \textbf{1 month} after the date of mailing of ISR by submission of corrections to the ISA.
    \end{enumerate}

    \item % Question 4
    Is it possible to file third party observations in relation to an international application during the international phase? If so, where and how can these be filed?


      Third party observations may be submitted at any time \textbf{after the date of publication} of the international application \underline{and} \textbf{before the expiration of 28 months from the priority date}, provided that the application is not withdrawn or considered withdrawn.
      
                They can be submitted through ePCT at no cost (you need a WIPO account). Each observation must                                                                                          include at least one citation that refers to a document published before the international                                         filing date, or a patent document having a priority date before the international filing date, together with a brief explanation of how each document is considered to be relevant to                                                                                                                                                                                                                                                                                                                                                                                                                                          the questions of novelty and/or inventive step of the claimed invention. Observations                                                                                                                                                                                                                                                                                                                                                                                                                                           should preferably be accompanied by a copy of each cited document.      They should be submitted in a language of publication (copies of prior art may be in any language). A single party may only submit a single observation for any PCTa, with a cap of ten observations (generally existing) per PCTa. 
                
                
                
                
                
    \item % Question 5                         
    A Danish applicant filed an international application PCT-DK as a first filing on 25 May 2025 with the EPO as receiving Office. Due to cash-flow problems, no fees were paid upon filing. On 30 June 2025, the EPO issues an invitation to pay the missing fees together with a late-payment surcharge.
                                                                                          
                                                                                          
    Indicate True/False:
    \begin{enumerate}[label=(\alph*)]
        \item \textbf{True} --- any payment received by rO before invitation to pay fees is considered to have been received before the time limit.
        \item \textbf{False} --- \textit{1 month} from the date of the invitation is the time limit. So it should be 30 July 2025.
        \item \textbf{False} --- any payment received before this declaration by the rO shall be considered to have been received before the expiration of the time limit. 
    \end{enumerate}

    \item % Question 6
    What happens if the applicant is a resident or national of one of the PCT Contracting States but files the international application with a "non-competent" receiving Office? What are the consequences for according the international filing date for such an application? Must an additional fee be paid?
    
    The PCTa will be considered to have been received by the Office with which it was filed on behalf of the IB as rO. The PCTa will be date-stamped by the nO (or regional Office) concerned and promptly submitted to the IB (unless for national security reasons).
    
    The filing date will be the date of receipt by the nO, but for calculating time limits for fee payments, the date on which the IB received the application will be used.
    
    That transmittal from nO to IB may be subjected to the payment of a fee equal to the transmittal fee, with other fees being refunded and then pending for payment again at the IB. 

\end{enumerate}

\section{September 29, 2025}

\begin{enumerate}[label=\textbf{Question \arabic*}]

    \item % Question 1
   An Austrian applicant A files an international application in German with the EPO. Before publication of the application, A sells this international application to US-company B, based in San Francisco, USA.
You are a European patent attorney representing company A and, after the purchase, company B in respect of this application.


For each of the statements below, indicate whether the statement is true or false:

    \begin{enumerate}[label=(\alph*)]
        \item  The change of applicant can be recorded by the IB during the international phase on request of the applicant (provided that the request is filed within the applicable time limit). 
\item A request to record the change of applicant can be filed with the EPO.
\item The change of applicant can be recorded by the IB during the national phase after the expiry of 30 months of priority date, with effect to all designated offices (provided that the request is filed within the applicable time limit).
\item The change of applicant can no longer be recorded during the international phase once international publication has taken place. 
\item The change of applicant can be recorded by the IB after the expiry of 30 months from the priority date until expiry of 31 months from the priority date, but such recordal only has effect with respect to the designated offices where the 31
month-period applies for national/regional entry.

   \end{enumerate}
    
    \item % Question 2
An applicant resident in Spain filed an international application in Spanish at the Spanish national Office indicating the EPO as International Searching Authority. Within one month the applicant furnishes pursuant to PCT Rule 12.3(a) a translation of the application into English.
    \begin{enumerate}[label=(\alph*)]
        \item   In which language will the international application be published?
\item  Which parts of the international application will (also) be published in English?
    \end{enumerate}

    \end{enumerate}
    
\begin{center}

\p \p \p
 
\end{center}

\begin{enumerate}[label=\textbf{Answer \arabic*}]

    \item % Question 1
   An Austrian applicant A files an international application in German with the EPO. Before publication of the application, A sells this international application to US-company B, based in San Francisco, USA.
You are a European patent attorney representing company A and, after the purchase, company B in respect of this application.


For each of the statements below, indicate whether the statement is true or false:

    \begin{enumerate}[label=(\alph*)]
        \item  \textbf{\textsc{True}} --- [\textbf{R. 92\textit{bis}.1}] deals with recording of changes in certain indications
in the Request or the Demand. Data changes on Applicant/Common Representative/Common Agent will be recorded by the IB up until 30 months after priority date. 
\item  \textbf{\textsc{True}} --- it may be filed with the rO (in this case, the EPO), but it is strongly recommended to go directly to the IB especially if closer to the 30-month deadline. 
\item  \textbf{\textsc{False}} --- [\textbf{R. 92\textit{bis}.1(b)} \& \textsc{AG--IP 11.021}] make it clear that after the deadline, the change must be requested directly with the dO/eO. 
\item \textbf{\textsc{False}} --- if change is to be taken into account before international publication, it should reach the IB before deadline for technical preparations, i.e., 1 day before the 15$^{\mathrm{th}}$ day before scheduled publication (see also: [\textsc{AG--IP 9.014}]).
\item \textbf{\textsc{False}} --- [\textbf{R. 92\textit{bis}.1(b)} \& \textsc{AG--IP 11.021}] make it clear that after the deadline, the change must be requested directly with the dO/eO, regardless of exact limit for entry into national/regional phase. 
    \end{enumerate}



    \item % Question 2
An applicant resident in Spain filed an international application in Spanish at the Spanish national Office indicating the EPO as International Searching Authority. Within one month the applicant furnishes pursuant to PCT Rule 12.3(a) a translation of the application into English.
    \begin{enumerate}[label=(\alph*)]
        \item  Spanish --- because it is a language of publication and the PCTa was filed in the language of publication [\textbf{R. 48.3(a)}].
\item  The declaration under \textbf{Art. 17.2(a)} [\textit{i.e., non-patentability}], the title of invention, the abstract and any text matter pertaining to the figure accompanying the abstract shall be published in English by the IB [\textbf{R. 48.3(a)}]..
    \end{enumerate}
   \end{enumerate}

\end{document}
