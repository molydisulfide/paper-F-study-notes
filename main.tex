\documentclass{report}
\input{preamble}
\input{macros}
\input{letterfonts}
\usepackage{adforn}
\usepackage{graphicx}
\usepackage[T1]{fontenc}
\usepackage{lmodern}
\usepackage{float}
\usepackage{mathtools}
\usepackage[dvipsnames]{xcolor}
\setlength\parindent{0pt}
\newcommand{\n}{\newline}
\newcommand{\p}{\adforn{61} \ }
\newcommand{\s}{\adforn{74} \ }
\newcommand{\T}{--- \textbf{True} }
\newcommand{\F}{--- \textbf{False} }
\usepackage{enumitem}
\newcommand{\art}[1]{\textbf{Art.~#1~EPC}}
\newcommand{\ru}[1]{\textbf{Rule~#1~EPC}}


\begin{document}
	
	\thispagestyle{empty}
	\mytitleb{Paper F Notes}{Jakub Jadwiszczak}{jadwiszj@tcd.ie}{2026}
	\newpage% or \cleardoublepage
	\tableofcontents

\chapter{Introduction to the Patent Cooperation Treaty --- PCT}
\section{Historical Notes and Some Quick Facts}

The \textbf{Patent Cooperation Treaty} (PCT) is an international patent law treaty, concluded in 1970. It provides a unified procedure for filing patent applications to protect inventions in each of its contracting states. \newline

\p   Signed on June 19, 1970 in Washington D.C.. Entered into force on January 24, 1978. The first application was filed on June 1, 1978.\newline

\p A patent application filed under the PCT is called an \underline{\textbf{international application}}, or \textbf{PCT application}. \n

\p No case law. No appeal body -- with exception of questions relating to unity of invention. \n

\p Rules have to cover all situations that might occur -- therefore are very detailed. \n

\p Contracting States have agreed to accept international filing date and the form and content of an international application having the effect of a national application, but they \textbf{have not limited the freedom to grant patents} to an \textbf{International Authority (IA)}. This runs in contrast to the EPC --- for example --- where that freedom has been ceded. \n

\p The international phase includes filing + search + publication + (optional) non-binding examination. No decision to grant. \n

\nt{Definition}{\textbf{``National phase'' for the purposes of PCT should be understood to mean ``national'' \underline{or} ``regional phase.''}} 
\vspace{5mm}
\p Any signatory of the Paris Convention may accede to the PCT. \n

\p The national office (NO) must approve form and content of the application as approved in the international phase. \n

\p The cost of entering national phase is similar to a direct national application. Further search + examination may be carried out in the national phase, but the use of the \textbf{international search report (ISR)} and examination may result in a fee reduction at national phase. \n

\p NO will grant patent on initial application with the same effect as on a direct national application. \n

\p PCT is administered by the Word Intellectual Property Organization (WIPO), which is a UN agency. The International Bureau (IB) of WIPO carries out admin for the International Patent Cooperation Union that the Contracting States make up. \n

\p The international phase is carried out before International Authorities. There are > 120 of them and > 20 can act as \textbf{International Search Authorities (ISAs)} or \textbf{International Preliminary Examination Authorities (IPEAs)}.  \n

\p After the international phase, all NOs act as so-called \textbf{designated Offices} (dO) or \textbf{elected Offices} (eO) [the latter in \textit{Chapter II}.]

\section{The Paris Convention for the Protection of Industrial Property}

\begin{center}
 \href{https://en.wikisource.org/wiki/Paris_Convention_for_the_Protection_of_Industrial_Property_(1883)}{\textbf{Full Text}}

\end{center}


\p Signed on March 20, 1883. \n

\begin{figure}[htbp] % h=here, t=top, b=bottom, p=page: Platzierungs-Optionen
  \centering
  \includegraphics[width=0.6\textwidth]{images/Fig_1.jpg} % Pfad und Dateiname anpassen
  \caption{--- \textbf{\textsc{PCT Hierarchy}}}
\end{figure}

\p Agreement between countries for mutual recognition of IP rights. Nationals of Signatory Countries enjoy the same rights in other States as nationals of those other States. \n

\p It secured \textbf{the right of priority} of a first filing in one State for subsequent applications in other States. \n

\p A \textbf{priority right} is a time-limited right triggered by the \textbf{first filing} of an application for a patent, industrial design, or trademark. It allows the applicant to file a \textbf{subsequent application} in another country that is effectively treated as if filed on the date of the first application, known as the \textbf{priority date}. To use this right, the applicant (\textbf{or their successor in title}) must \textbf{claim priority} in the subsequent application. \n

\p The priority period is \textbf{12 months} for patents and utility models (the \textbf{priority year}) and \textbf{6 months} for industrial designs and trademarks. In the original Paris Convention it was 6 months and 3 months, respectively. \n

\p For patents, this right is crucial because \textbf{novelty} and \textbf{inventive step} are assessed against prior art that was made public \textbf{before the priority date}, not the actual (later) filing date of the subsequent application. \n

\nt{Rationale}{\textbf{according to the EPO: \textit{``(...) basic purpose [of the right of priority] is to safeguard, for a limited period, the interests of a patent applicant in his endeavour to obtain international protection for his invention, thereby alleviating the negative consequences of the principle of territoriality in patent law.''}}} 
\vspace{5mm}

\p \textbf{\s Art. 19} of the Paris Convention allows for special agreements between Signatory Countries. The Paris Convention \underline{takes precedence} over laws of the Countries and over such special agreements. \n

\p EPC is a ``regional patent treaty'' in the sense of \textbf{Art. 19} of the Paris Convention, e.g., under \textbf{Art. 45 PCT}, a PCT applicant can obtain an \underline{\underline{EP patent}} by filing an initial international application. \n

\nt{Definition}{\textbf{\[
    \begin{dcases}
        \mathrm{Patent \ \underline{in} \ a \ state} & = \ \ \mathrm{\underline{national}}\\
        \mathrm{Patent \ \underline{for} \ a \ state} & = \ \ \mathrm{\underline{regional}} \\
    \end{dcases}
\]}} 
\vspace{5mm}

\p Under \textbf{Art. 45 PCT}, a PCT Applicant can obtain a European patent by filing a PCTa. \n

\p In case of conflict between PCT and EPC/national provisions, the PCT takes precedence. \n

\p A ``Euro--PCT'' application is a PCT application with EPO as dO or eO. \n

\p The PCT timeline for a PCTa claiming priority from a national application: 

\newpage
\begin{figure}[H] % h=here, t=top, b=bottom, p=page: Platzierungs-Optionen
  \centering
  \includegraphics[width=0.85\textwidth]{images/Fig_2.jpg} % Pfad und Dateiname anpassen
  \caption{--- \textbf{\textsc{PCT Timeline}}}
\end{figure}




\p The international phase includes: international search, publication and (optionally) international preliminary examination. The international phase has a \underline{Chapter I} (\textbf{Art. 3--30}; \textbf{R. 3--52}) and a \underline{Chapter II} (\textbf{Art. 31--42}; \textbf{R. 53--78}). These correspond to search + pub and IPE. \n

\p Each nO has a time limit for entering national phase (usually 30 or 31 months from the priority date). A PCTa can be simultaneously in the international phase and in the national phase for some jurisdictions. \n

\nt{Chapter I}{\textbf{Stage 1}: filing, accordance of date, fees, formal requirements. \newline
\textbf{Stage 2}: ISR with written opinion on Novelty, IS and industrial applicability. \newline
\textbf{Stage 3}: publication of PCTa + ISR @18 months after priority date. } 
\vspace{5mm}


\nt{Chapter II}{\textbf{Stage 1}: filing a \underline{\textbf{demand}}, i.e., a request for an IPE --- typically @22 months after priority date.\newline
\textbf{Stage 2}: can amend claims and discuss Novelty, IS with Examiner. \newline
\textbf{Stage 3}: receipt of the \underline{IPRP II} typically @28 months from priority date. } 
\vspace{5mm}

\p Applicant can pay for a Supplementary ISR carried out by a SISA (can be different to ISA of Chapter I, e.g., in order to find prior art in a specific language.) \n

\p If no IPE demanded, written opinion from ISR is converted into the International Preliminary Report on Patentability I (\underline{IPRP I}) with an English translation thereof + documents are communicated to the dO. Usually happens @30 months after priority date.\n

\p IPRP is not binding on the dO but is made available to the dO. dO then becomes the eO.\n

\p Only about 3\% of applications ever enter Chapter II. \n


\p \textsc{LU} is a notable exception for entering national phase -- 20 months. \n

\p PCT advantages over direct national application include: 

\begin{itemize}
 \item cost reduction,
 \item 30-month delay to check market developments, own product developments, etc.,
 \item patentability opinion to inform later actions,
 \item delay of formalities/payments right away.
\end{itemize}

\p If no priority claim, then \textbf{the priority date is the PCTa filing date}. \n

\p ``Agent'' is ``professional representative'' in EPC terminology. \n

\p Terminology for document copies:

\begin{itemize}
\centering
 \item \textbf{Home copy} --- stays with rO.
 \item \textbf{Record copy} --- sent to IB.
 \item \textbf{Search copy} --- sent to ISA.
\end{itemize}

\p Record copy is the \textbf{true} copy. \n

\p It is possible that scope of PCTa only includes \textit{certain territories} of a CS, e.g., a Euro--PCT for Denmark does not include the Faroe Islands. \n

\p States can join to form a single legal territory for patent law, e.g., Switzerland and Liechtenstein. \n

\newpage

\chapter{The International Application (\textsc{PCTa})}

\section{Basics}

\p Elements that a PCTa must contain:

\begin{itemize}
 \item a request (\textbf{Art. 4)},
 \item a description (\textbf{Art. 5)},
 \item one or more claims (\textbf{Art. 6)},
 \item where required, one or more drawings (\textbf{Art. 7)},
 \item an abstract (\textbf{Art. 8)}.
\end{itemize}

\p To accord a \textbf{filing date}, you need:

\begin{itemize}
 \item an indication that it is inteded as a PCTa, including a designation of a Contracting State and name of Applicant,
 \item a description,
 \item at least one claim.
\end{itemize}

\p The Abstract may be submitted later though. \n

\p On filing, the description, claims and drawings may be replaced by \underline{reference to a priority application,} from which the rO will copy them [\textbf{R. 4.18} \& \textbf{R. 20.6}].

\section{Art. 3(4)(i) \& Rule 12 --- Languages and Translations}

\subsection{Languages accepted by the rO}

\p A PCTa must be filed in a language prescribed by the rO. \textbf{Every} rO accepts at least one language, which is \textbf{both} a language accepted by the competent ISA \underline{and} a language of publication. Languages of publication are listed under \textbf{R. 48.3(a)}. If using the language of publication, no translation need be filed.  \n

\p To get a filing date, claims and description must be in a language accepted by that rO. For cases of mixed languages of claims/description, an invitation to correct this will be issued under [\textbf{R. 26.3\textit{ter}(a)}]. \n

\p Languages of the request and the description/claims may be different though, e.g., an English request and Dutch claims/description. \n

\p The rO may require a translation for the purposes of search or publication, e.g., \textsc{RO/EP} requires \textbf{FR/EN/DE} translation. \n

\p All rOs of EPC CSs have specified the EPO as their competent ISA. Search in Dutch is possible at the EPO if the application was filed in Dutch for historical reasons. \n

\p The language of the rO need not be the official language of that CS. See also national security considerations under \textbf{Art. 27.8}, which may become relevant in such a case. \n

\p If a nO as rO does not accept the application language, it will transmit the PCTa to the IB to act as the rO under [\textbf{R. 19.4}]. IB accepts \textbf{any language}. In that case, new fees will need to be paid to the IB, the fees already paid to the nO will be refunded --- \textbf{except the transmittal fee} --- and the Applicant has to send the priority document to the IB. \n

\subsection{Translations}

\p The Applicant must provide a translation if:

\begin{enumerate}
 \item the language of filing is not accepted by the ISA (or SISA);
 \item the language of the ISA is not a publication language.
\end{enumerate}

\p The translation must be provided within \textbf{1 month of filing at the rO} -- otherwise a late furnishing fee is also due (equal to \textbf{25 \% of the international filing fee}). \n

\p The translation must be in the language of the ISA, language of publication \underline{and} language of the rO. Sequence listing free text needs to be translated as well. \n

\p If no translation is provided, the PCTa will be declared by the rO to have been withdrawn . \n

\p The rO will invite a translation witin 1 month of filing, if not received at filing. The late submission deadline is \textbf{2 months after filing} or \textbf{1 month after the invitation} -- whichever expires \textbf{later}. \n

\p Nonetheless, a translation and payment received before 15 months after priority date will be considered as received in time. Furthermore, a translation and payment received \textbf{before the rO makes the declaration} is also considered as received on time. \n

\p Under [\textbf{R. 12.4}] --- where a PCTa has been filed in a language accepted by the ISA, which is not a language of publication:

\begin{itemize}
 \item the Applicant must file a translation to a language of publication within \textbf{14 months} of the priority date;
 \item the rO will send an invitation for the late translation within \textbf{16 months} of the priority date;
 \item if the translation and late payment fee are received within \textbf{17 months} of the priority date, it is considered on time; and
 \item if the translation and late payment fee are received before the rO \textbf{declares withdrawal}, it is still considered on time.
\end{itemize}

\p PCT has a provision for retroactive scope limitation due to an incorrect translation under [\textbf{Art. 46}].

\section{Fees}

\p On filing of a PCTa, three fees (in Swiss francs) become due at the rO:

\begin{enumerate}
 \item transmittal fee;
 \item international filing fee (including sheet fee if $>$ 30 pages;
 \item international search fee.
\end{enumerate}

\p There are no claims fees. They may be due for some offices in the national phase. \n

\p PCT has no provisions for methods of payment or according a date of payment. Each rO will have its own rules on that. \n

\p The \textbf{transmittal fee} [\textbf{R. 14}] --- includes checking the PCTa, transmitting a record copy to the IB and the search copy to the ISA. The time limit for receipt is \textbf{1 month from the date of receipt of the application by the rO} (not from the filing date). The filing date might be shifted due to missing elements, etc., so the fees deadlines are not dependent on the filing date. This deadline is also shifted if the rO sends the PCTa to the IB when the rO is not the Competent Authority. \n

\p The \textbf{international filing fee} [\textbf{R. 15}] --- includes publication, translation/s and communication to ISA/SISA/IPEA. Must be paid before formalities check of \textbf{Art. 14(1)(a)} and any resulting corrections. The 30 pages limit before any extra fees is calculated on the basis of documents as filed. Within \textbf{1 month of date of receipt of the PCTa} again (not the filing date).




\newpage

\chapter{Fillun Homework Questions}

\section{September 22, 2025}

\begin{enumerate}[label=\textbf{Question \arabic*}]

    \item % Question 1
    Two applicants wish to appoint an agent to file their international application.
    \begin{enumerate}[label=(\alph*)]
        \item Who can be appointed to act as an agent?
        \item How must the agent be appointed?
        \item Do all receiving Offices require the filing of a separate power of attorney?
        \item Does the EPO require the filing of a separate power of attorney?
    \end{enumerate}
    
    \vspace{1em} % Adds a little vertical space
    The applicants always work via the same patent attorney office for which a general power of attorney has been prepared.
    
    \begin{enumerate}[label=(\alph*), resume]
        \item Is it necessary to attach a copy of the general power of attorney to the Request [PCT/RO/101]?
        \item Does the EPO require filing a copy of the general power of attorney?
    \end{enumerate}

    \item % Question 2
    An international application is filed at the Japanese National Office. The applicant first mentioned in the Request [PCT/RO/101] is a Taiwanese national resident in Taiwan. The second applicant is a Korean national resident in Korea and the third applicant is a Japanese national resident in Japan. The three applicants have not appointed a common agent or a common representative.
    \begin{enumerate}[label=(\alph*)]
        \item Who will be considered to be the common representative of the applicants?
        \item Would your answer to question (a) have been different if the international application had been filed at the International Bureau?
        \item Which acts may not be performed by the representative in question (a)?
        \item Which acts may not be performed by an agent appointed by the representative in question (a)?
    \end{enumerate}

    \item % Question 3
    Today, 16 March 2020, the applicant discovers an obvious mistake in the description of his international application filed on Friday 9 March 2018 as a first filing.
    \begin{enumerate}[label=(\alph*)]
        \item Can this obvious mistake be corrected? If so, what are the conditions?
        \item Who is competent to rectify the obvious mistake? What if the mistake is only detected after a demand for international preliminary examination has been made?
        \item What parts of the international application are taken into account upon correcting the mistake?
        \item When at the latest can a request for rectification be filed?
        \item Would your answer have been different if the applicant had discovered an obvious mistake in the abstract of his international application?
    \end{enumerate}

    \item % Question 4
    Is it possible to file third party observations in relation to an international application during the international phase? If so, where and how can these be filed?

    \item % Question 5
    A Danish applicant filed an international application PCT-DK as a first filing on 25 May 2025 with the EPO as receiving Office. Due to cash-flow problems, no fees were paid upon filing. On 30 June 2025, the EPO issues an invitation to pay the missing fees together with a late-payment surcharge.

    \vspace{1em}
    Indicate True/False:
    \begin{enumerate}[label=(\alph*)]
        \item If the applicant has, in fact, paid the fees due at filing on 27 June 2025, the payment will be considered in time.
        \item The time limit to pay the missing fees together with late-payment surcharge expires on 1 September 2025.
        \item If the applicant pays the missing fees + surcharge one day after the time limit to do so has expired, this is too late and the EPO is obliged to declare under Art. 14(3) that PCT-DK is considered withdrawn.
    \end{enumerate}

    \item % Question 6
    What happens if the applicant is a resident or national of one of the PCT Contracting States but files the international application with a "non-competent" receiving Office? What are the consequences for according the international filing date for such an application? Must an additional fee be paid?

\end{enumerate}

\begin{center}

\p \p \p
 
\end{center}

\begin{enumerate}[label=\textbf{Answer \arabic*}]

    \item % Question 1
    Two applicants wish to appoint an agent to file their international application.
    \begin{enumerate}[label=(\alph*)]
        \item Under \textbf{R. 90}: a person having the right to practice before the national Office with which the PCTa is filed \underline{or} having the right to practice before the IB as rO. The latter is governed by \textbf{Rule 83}, i.e., the agent must have a right to practice before the nO of the Contracting State of which the Applicant is a resident or national.
        \item Applicant enters and signs the name \textbf{and} address of the agent in the request \underline{or} the demand \underline{or} in a separate power of attorney (applicable to a \textbf{specific} PCTa) \underline{or} by a general power of attorney (applicable to \textbf{any} PCTa).
        \item No.
        \item No.

        \item Yes. It says so on the \textsc{[PCT/RO/101]} form.
        \item No. Waiver under \textbf{Rule 90.5(c)}. Two exceptions relating to suspicions as to the nature of the person performing acts apply.
    \end{enumerate}

    \item % Question 2
    An international application is filed at the Japanese National Office. The applicant first mentioned in the Request [PCT/RO/101] is a Taiwanese national resident in Taiwan. The second applicant is a Korean national resident in Korea and the third applicant is a Japanese national resident in Japan. The three applicants have not appointed a common agent or a common representative.
    \begin{enumerate}[label=(\alph*)]
        \item According to \textbf{Rule 90}, the first Applicant named in the Request entitled to file a PCTa with an rO will become the common representative. In this case, this should be the Korean national.
        \item No?
        \item The common representative may not sign any notice of withdrawal [under \textbf{R. 90\textit{bis}}], i.e.: withdrawal of application, designation, priority claim, supplementary search request, demand or election. Also, \textbf{not sure}, but it seems like he cannot perform an act in relation to only one Applicant or a subset of Applicants when there are multiple Applicants. 
        \item The common agent may not file a PCTa without signature of the Applicant/s [\textbf{R. 4.15}] and cannot make declarations as to entitlement on behalf of the Applicant/s [\textbf{R. 4.17}].
    \end{enumerate}

    \item % Question 3
    Today, 16 March 2020, the applicant discovers an obvious mistake in the description of his international application filed on Friday 9 March 2018 as a first filing.
    \begin{enumerate}[label=(\alph*)]
        \item Yes --- under \textbf{Rule 91.2}, there is a 26 month-deadline for correction of obvious mistakes. In this case, we have 24 months + 7 days.
        \item The Applicant (or his agent) are competent. After demand, the IPEA is the Competent Authority to be addressed. Notably, the dO/eO need not take rectification into account if processing/examination started prior to the notification [\textbf{R. 91.3(e)}] and the dO/eO may disregard an authorized notification if it finds it would not have authorized it itself had it been the Competent Authority [\textbf{R. 91.1(f)}].
        \item For mistakes in the claims, description or drawings (or corrections thereof), only the claims, description and drawings (and corrections thereof) will be taken into account [\textbf{R. 91.1(d)}]. 
        For mistakes in the Request (or corrections thereof), the contents of the whole PCTa including priority documents, corrections, etc., will be taken into account [\textbf{R. 91.1(e)}].
        \item 26 months after the priority date.
        \item Yes. Abstract mistakes may not be corrected under \textbf{Rule 91.1(g)}. They may be corrected under [\textbf{R. 38.3}] within \textbf{1 month} after the date of mailing of ISR by submission of corrections to the ISA.
    \end{enumerate}

    \item % Question 4
    Is it possible to file third party observations in relation to an international application during the international phase? If so, where and how can these be filed?


      Third party observations may be submitted at any time \textbf{after the date of publication} of the international application \underline{and} \textbf{before the expiration of 28 months from the priority date}, provided that the application is not withdrawn or considered withdrawn.
      
                They can be submitted through ePCT at no cost (you need a WIPO account). Each observation must                                                                                          include at least one citation that refers to a document published before the international                                         filing date, or a patent document having a priority date before the international filing date, together with a brief explanation of how each document is considered to be relevant to                                                                                                                                                                                                                                                                                                                                                                                                                                          the questions of novelty and/or inventive step of the claimed invention. Observations                                                                                                                                                                                                                                                                                                                                                                                                                                           should preferably be accompanied by a copy of each cited document.      They should be submitted in a language of publication (copies of prior art may be in any language). A single party may only submit a single observation for any PCTa, with a cap of ten observations (generally existing) per PCTa. 
                
                
                
                
                
    \item % Question 5                         
    A Danish applicant filed an international application PCT-DK as a first filing on 25 May 2025 with the EPO as receiving Office. Due to cash-flow problems, no fees were paid upon filing. On 30 June 2025, the EPO issues an invitation to pay the missing fees together with a late-payment surcharge.
                                                                                          
                                                                                          
    Indicate True/False:
    \begin{enumerate}[label=(\alph*)]
        \item \textbf{True} --- any payment received by rO before invitation to pay fees is considered to have been received before the time limit.
        \item \textbf{False} --- \textit{1 month} from the date of the invitation is the time limit. So it should be 30 July 2025.
        \item \textbf{False} --- any payment received before this declaration by the rO shall be considered to have been received before the expiration of the time limit. 
    \end{enumerate}

    \item % Question 6
    What happens if the applicant is a resident or national of one of the PCT Contracting States but files the international application with a "non-competent" receiving Office? What are the consequences for according the international filing date for such an application? Must an additional fee be paid?
    
    The PCTa will be considered to have been received by the Office with which it was filed on behalf of the IB as rO. The PCTa will be date-stamped by the nO (or regional Office) concerned and promptly submitted to the IB (unless for national security reasons).
    
    The filing date will be the date of receipt by the nO, but for calculating time limits for fee payments, the date on which the IB received the application will be used.
    
    That transmittal from nO to IB may be subjected to the payment of a fee equal to the transmittal fee, with other fees being refunded and then pending for payment again at the IB. 

\end{enumerate}

\section{September 29, 2025}

\begin{enumerate}[label=\textbf{Question \arabic*}]

    \item % Question 1
   An Austrian applicant A files an international application in German with the EPO. Before publication of the application, A sells this international application to US-company B, based in San Francisco, USA.
You are a European patent attorney representing company A and, after the purchase, company B in respect of this application.


For each of the statements below, indicate whether the statement is true or false:

    \begin{enumerate}[label=(\alph*)]
        \item  The change of applicant can be recorded by the IB during the international phase on request of the applicant (provided that the request is filed within the applicable time limit). 
\item A request to record the change of applicant can be filed with the EPO.
\item The change of applicant can be recorded by the IB during the national phase after the expiry of 30 months of priority date, with effect to all designated offices (provided that the request is filed within the applicable time limit).
\item The change of applicant can no longer be recorded during the international phase once international publication has taken place. 
\item The change of applicant can be recorded by the IB after the expiry of 30 months from the priority date until expiry of 31 months from the priority date, but such recordal only has effect with respect to the designated offices where the 31
month-period applies for national/regional entry.

   \end{enumerate}
    
    \item % Question 2
An applicant resident in Spain filed an international application in Spanish at the Spanish national Office indicating the EPO as International Searching Authority. Within one month the applicant furnishes pursuant to PCT Rule 12.3(a) a translation of the application into English.
    \begin{enumerate}[label=(\alph*)]
        \item   In which language will the international application be published?
\item  Which parts of the international application will (also) be published in English?
    \end{enumerate}

    \item % Question 3
A Dutch applicant filed a European patent application with the EPO on Friday 11 May 2018 comprising just a description and drawings, but no claims. On the same day, he filed the same documents with the IB together with a PCT Request Form. On Wednesday 23 May 2018, he filed a set of claims with the EPO and also with the IB, referring to his earlier submissions of 11 May 2018.


For each of the statements below, indicate whether the statement is true or false:

The filing date accorded to the European patent application is...


    \begin{enumerate}[label=(\alph*)]
        \item  … 11 May 2018.
\item … 23 May 2018.
    
\vspace{0.5cm}

The filing date accorded to the international patent application is ...
 
        \item  … 11 May 2018.
\item … 23 May 2018.
    \end{enumerate}

    \item % Question 4
An international application is jointly filed at the German national Office on 3 April 2017 by a British applicant and a German applicant claiming priority from an earlier German national application filed on 4 April 2016.


For each of the statements below, indicate whether the statement is true or false:

    \begin{enumerate}[label=(\alph*)]
        \item  All applicants must be indicated in the Request (PCT/RO/101).
\item The rO considers it sufficient if the British applicant signs the Request.
\item The receiving Office will invite the applicants to furnish any missing address, nationality and residence.
    
\vspace{0.5cm}

Later on the applicants wish to withdraw the priority claim.
 
        \item The priority claim may be withdrawn until 3 October 2019.
\item It is sufficient if the first-named applicant, being considered to be the common representative under R.90.2(b), signs the notice of withdrawal.
\item  If the applicants would have appointed a common representative, the common representative may sign the notice of withdrawal.


\vspace{0.5cm}

Later on the international application enters the regional phase before the EPO.


\item The EPO will ask for any missing indication of the applicants in the Request.


    \end{enumerate}


    \item % Question 5

An international application is filed on 28 July 2018 indicating a priority date of 27 January 2018. The applicant wishes to correct the priority date to 26 September 2017 and asks you until when may he can request the competent authority of a correction of the priority date. Today is 12 November 2018.

For each of the statements below, indicate whether the statement is true or false.

The last day to request the correction is...


    \begin{enumerate}[label=(\alph*)]
        \item  Wednesday 28 November 2018. 
\item Monday 28 January 2019.
\item Monday 27 May 2019.   
\vspace{0.5cm}

The request may be filed with…
 
        \item the receiving Office.
\item the International Bureau.



    \end{enumerate}

\item % Q6

    \begin{enumerate}[label=(\alph*)]
        \item  Can the applicant withdraw a priority claim made in the international application? If so, until when?
\item To whom must the request be addressed? Who must sign the withdrawal?
\item Is a fee due for the withdrawal of a priority claim? 

\item What are the consequences of a withdrawal of a priority claim?

     \end{enumerate}


\item % Q7

Today, Monday 27 November 2017, an applicant wants to file an international application at the EPO as receiving Office claiming the priority of an earlier national application filed on 24 September 2016.

    \begin{enumerate}[label=(\alph*)]
        \item  Is it still possible to file the international application and claim the priority?
\item  If so, how must the applicant proceed?
\item Do all rOs accept such requests?

\item Are dOs required to accept the claimed priority?
\item What about the EPO?

     \end{enumerate}
       \end{enumerate}

       \section{October 09, 2025}

\begin{enumerate}[label=\textbf{Question \arabic*}]

    \item % Question 1
    L3-06 - T/F question (Basic selection) \\
    An international application is filed by a Danish national at the Danish Patent and Trademark Office as receiving Office. \\
    For each of the statements below, indicate whether the statement is true or false.
    \begin{enumerate}[label=(\alph*)]
        \item The IA may be filed in…
        \begin{enumerate}[label={(\alph{enumi}.\arabic*)}]
            \item ... Danish
            \item ... English
            \item ... Swedish
        \end{enumerate}

        \item The applicant may select as ISA:
        \begin{enumerate}[label={(\alph{enumi}.\arabic*)}]
            \item ... the Danish national office.
            \item ... the EPO
            \item ... the Swedish national office.
            \item ... the Nordic Patent Institute.
        \end{enumerate}
        
        \item \textit{The application is filed in Danish.} \\
        For each of the statements below, indicate whether the statement is true or false
        
        \vspace{0.5em}
        The EPO as ISA…
        \begin{enumerate}[label={(\alph{enumi}.\arabic*)}]
            \item ... does not require a translation for the purposes of international search.
            \item ... requires a translation into any one of the 10 publication languages for the purposes of international search.
            \item ... requires a translation into English, French or German for the purposes of international search.
            \item[] \textit{The Nordic Patent Institute as ISA...}
            \item ... does not require a translation for the purposes of international search.
            \item ... requires a translation into any one of the 10 publication languages for the purposes of international search.
            \item ... requires a translation into English, French or German for the purposes of international search.
        \end{enumerate}

        \item \textit{The applicant has filed the international application in Danish and has indicated the EPO as International Searching Authority. The applicant is in the progress of making a translation into English. The application was filed by fax with the Danish Patent and Trademark Office on a day on which the Danish Patent and Trademark Office and the International Bureau were closed.} \\
        For each of the statements below, indicate whether the statement is true or false: \\
        The translation must be filed ...
        \begin{enumerate}[label={(\alph{enumi}.\arabic*)}]
            \item ... within one month from the date of receipt by the Danish Patent and Trademark Office to avoid any late furnishing fee(s).
            \item ... within one month from the date of receipt by the EPO to avoid any late furnishing fee(s).
            \item ... when the 1m period was missed, the translation may still be provided as long as a late furnishing fee is paid.
            \item ... when the rO issues an invitation to supply a missing translation, the ultimate time limit to respond is always one month from the date of the invitation.
        \end{enumerate}
        
        \item \textit{The applicant has timely filed the translation into English to the Danish Patent and Trademark Office.} \\
        The international publication of the international application will take place in…
        \begin{enumerate}[label={(\alph{enumi}.\arabic*)}]
            \item ... Danish.
            \item ... English.
            \item ... in Danish with the title and the abstract also in English.
        \end{enumerate}
    \end{enumerate}

    \item % Question 2
    L3-08 - T/F question (Basic selection) \\
    A PCT application was filed with the EPO. As the International Searching Authority, the EPO considered that the application was not unitary. The invention first mentioned in the claims was searched and an invitation to pay two additional international search fees was sent to the applicant last week, 8 March 2023. The third invention, which has not yet been searched, is the only invention that the applicant would like to pursue in the European phase before the EPO. \\
    Today is 17 March 2023.
    \begin{enumerate}[label=(\alph*)]
        \item For each of the statements below, indicate whether the statement is true or false:
        \begin{enumerate}[label={(\alph{enumi}.\arabic*)}]
            \item In the international PCT phase, the applicant can file a protest with the EPO and request that a full search is made. The protest is free of charge, but it has to be supported by arguments.
            \item The applicant can timely pay one additional fee for the third invention to be searched in the international PCT phase. In the European phase, the applicant can limit the application to the third invention.
            \item The applicant can ignore the invitation. In the European phase, the applicant can file a divisional application directed to the third invention.
            \item The applicant can ignore the invitation. In the European phase, the applicant will again receive an invitation to pay additional search fees.
        \end{enumerate}
        
        \item (from L3-14)
        \begin{enumerate}[label={(\alph{enumi}.\arabic*)}]
            \item What is the purpose of the requirement of "unity of invention"?
            \item What is the time limit to pay the additional search fee in the international phase?
            \item To whom must the additional search fee be paid?
            \item When must the applicant pay a "protest fee"? What time limit applies?
        \end{enumerate}
    \end{enumerate}

    \item % Question 3
    L3-16 
    \begin{enumerate}[label=(\alph*)]
        \item When must the international search report be established?
        \item What are the contents of the international search report?
        \item What can the applicant do after receiving the international search report, apart from filing a demand for international preliminary examination?
        \item What time limits apply? Where must the applicant file the amendments?
        \item In what form must the applicant file the amendments and any accompanying letter or statement? In which language?
        \item When are amendments to the claims under PCT Article 19 not allowed?
    \end{enumerate}

    \item % Question 4
    L3-20 (Basic selection) \\
    An applicant has filed an international application on 9 May 2023 as a first filing at the EPO as receiving Office, in English. The EPO acted as International Searching Authority and the international search report was transmitted to the applicant in December 2023; the written opinion (WO-ISA) suggests that the invention is patentable. \\
    The applicant being afraid that there is prior art which was not discovered by the EPO as ISA, considers filing a request for supplementary international search to be carried out by the Intellectual Property Office of Singapore.
    \begin{enumerate}[label=(\alph*)]
        \item Until when can the applicant file a request for supplementary international search [PCT/IB/375]? Where should the request be filed? In which language should the request be filed?
        \item What fees must be paid in respect of the request for supplementary international search? By when must the fees be paid? What is the amount of the fees?
        \item When will the supplementary international search start?
        \item By when must the supplementary international search report be established? Will a written opinion be issued together with the supplementary international search report?
        \item Will the supplementary international search report cite any document which was already cited in the international search report established by the EPO?
        \item Will the supplementary international search report be published? If so, in what form?
    \end{enumerate}

    \item % Question 5
    L3-27 \\
    A US applicant wants to request a demand for international preliminary examination at the USPTO for an international application filed on 29 December 2016 without any claiming priority. The ISR and WO-ISA, established by the USPTO as ISA, were transmitted to the applicant on 26 June 2018. \\
    However, due to a severe hurricane, the USPTO was closed on 29 and 30 October 2018, and the applicant was not able to submit the demand on 29 October 2018. \\
    Can the demand still be validly filed if today is 31 October 2018?

    \item % Question 6
    In which cases can the EPO act as IPEA? \\
    (multiple choice)
    \begin{enumerate}[label=(\alph*)]
        \item Spanish office was the ISA
        \item Nordic Patent Institute was the ISA
        \item USPTO was the ISA
        \item EPO was the ISA
    \end{enumerate}

    \item % Question 7
    What documents must an applicant file when filing Article 19 amendments? \\
    (single choice)
    \begin{enumerate}[label=(\alph*)]
        \item Only the amended claims \& a letter indicating the differences plus the basis for the amendments
        \item A complete set of claims in replacement of the claims originally filed \& a letter indicating the differences plus the basis for the amendments 
        \item Only the amended claims \& a letter indicating the differences plus the basis for the amendments \& statement by the applicant explaining the amendment and indicating any impact it might have on the description and the drawings
        \item A complete set of claims in replacement of the claims originally filed \& a letter indicating the differences plus the basis for the amendments \& statement by the applicant explaining the amendment and indicating any impact it might have on the description and the drawings
    \end{enumerate}

    \section{October 20, 2025}

\begin{enumerate}[label=\textbf{Question \arabic*}]

    \item % Question 1
    L4-06 (Basic selection) \\
    A US applicant, resident in the US, files his international application at the USPTO. Subsequently he wishes to pursue his application before the EPO as designated office.
    \begin{enumerate}[label=(\alph*)]
        \item Can the US applicant initiate the national entry procedure himself?
        \item What happens if a representative is not appointed once the processing has started by the EPO?
        \item What are the minimum acts to enter the regional phase before the EPO if surcharges are to be avoided?
    \end{enumerate}

    \item % Question 2
    L4-12 - T/F question (Basic selection) \\
    An international application was published in Korean together with the international search report established by the Korean patent office as ISA. The international application does not claim priority. The applicant wants to enter the European regional phase today, the last day of the 31-month period without any penalty fees. \\
    Which fees need to be paid within what time limit to avoid any penalty fees?

    \vspace{1em}
    For each of the statements below, indicate whether the statement is true or false.
    
    \vspace{0.5em}
    The examination fee….
    \begin{enumerate}[label=(\alph*)]
        \item ... needs to be paid upon entry today.
        \item ... may be paid, without any additional fee, until 6 months from the publication of the translation of the international application under Art.153(4) EPC, first sentence, as the publication of the translation comprises the translation of the international search report.
        \item ... may be paid, without any additional fee, until 6 months from the mention of the supplementary European search report in the European patent Bulletin.
        \item ... will be refunded if the applicant already paid the examination fee on entry and withdraws the application shortly after receipt of the supplementary European search report.
    \end{enumerate}

    \item % Question 3
    L4-22 \\
    How many claims fees must the applicant pay or does he get refunded, and when, in the following cases:
    
    \vspace{1em}
    \textbf{Case 1:} A Euro-PCT application X contains 27 claims on expiry of the 31-month period. The applicant pays five claims fees within the 31 month period. No amendments are filed after expiry of the 31-month period and before expiry of the six-month period under Rule 161.
    
    \vspace{1em}
    \textbf{Case 2:} A Euro-PCT application Y contains 27 claims on expiry of the 31-month period. The applicant pays five claims fees within the 31-month period. After expiry of the 31-month period and before expiry of the six-month period under Rule 161, the applicant files an amended set of 32 claims.

    \item % Question 4
    L4-23 (Basic selection) \\
    An international application is filed on 7 September 2015 without claiming priority. The applicant wants to enter the regional phase before the EPO as designated Office. Today is 26 February 2018. \\
    What is the last day on which the renewal fee in respect of the third year can be paid to the EPO…
    \begin{enumerate}[label=(\alph*)]
        \item ... without additional fee?
        \item ... with additional fee?
    \end{enumerate}

    \vspace{1em}
    L4-24 \\
    An international application is filed on 9 June 2016 claiming priority of a national application filed on 7 September 2015. The applicant wants to enter the regional phase before the EPO as designated Office. Today is 26 February 2018. \\
    What is the last day on which the renewal fee in respect of the third year can be paid to the EPO…
    \begin{enumerate}[label=(\alph*), resume]
        \item … without additional fee?
        \item … with additional fee?
    \end{enumerate}

    \item % Question 5
    L4-30 (Basic selection) \\
    A Swedish applicant has filed an international application indicating the Swedish Patent and Registration Office as International Searching Authority. The application is filed in English. During the international search, the ISA is of the opinion that the international application contains 3 inventions, A, B and C, and that with respect to the second and third invention an additional fee has to be paid for each of them if the International Search Report is to cover these inventions. An additional fee is only paid for the second invention B. No demand was filed. The applicant wants to enter the regional phase before the EPO in May 2023, and let the EPO conduct searches on all three inventions A, B and C in the regional phase before deciding with which invention to continue into examination.
    
    \vspace{1em}
    Which amount of search fee(s) and/or further search fees must the applicant pay at or shortly after entry into the regional phase before the EPO (assuming the EPO shares the non-unity opinion)?

    \item % Question 6
    L4-41 (Basic selection) \\
    Today, 25 February 2019, your Japanese client asks you to enter the regional phase before the EPO for a PCT-application filed in Japanese with the Japanese Patent Office on 26 July 2017, validly claiming priority from a Japanese patent application P-JP dated 1 August 2016. The International Preliminary Examination was carried out by the Japanese Patent Office. Due to an overload of work in the translation department of your client, you are informed that you will receive the English translation of the PCT-application on 23 April 2019.
    
    \vspace{1em}
    How would you proceed in order to enter the regional phase, keeping costs for your client to a minimum?

    \item % Question 7
    L4-42 (Advanced selection) \\
    An applicant wanted to enter the EP phase with his international application and duly filed a translation into English, and paid the filing fee and search fee. Shortly after the 31m time limit, the applicant received a loss-of-rights communication, indicating that the application was deemed to be withdrawn due to non-payment of the designation and examination fee and not filing the request for examination. One week after having received the communication, the applicant got into the hospital and despite all due care, he missed the time limit to request further processing with respect to the missed periods. 
    
    \vspace{1em}
    How many re-establishment fees does the applicant need to pay? 

\end{enumerate}
 \section{October 27, 2025}


\begin{enumerate}[label=\textbf{Question \arabic*}]

    \item % Question 1
    B4-02 \quad T/F question (Basic selection) \\
    Today is late February 2025. A Dutchman living in Sweden wishes to file a European patent application at the EPO in Munich via a Polish friend, who is a European patent attorney.
    
    \begin{enumerate}[label=(\alph*)]
        \item For each of the statements below, indicate whether the statement is true or false.
        \begin{enumerate}[label={(\alph{enumi}.\arabic*)}]
            \item The Dutchman is entitled to file the European patent application in Dutch.
            \item The Dutchman is entitled to file the European patent application in Swedish.
            \item The Dutchman is entitled to file the European patent application in Polish.
            \item The Dutchman is entitled to file the European patent application in German.
        \end{enumerate}
        
        \item For each of the statements below, indicate whether the statement is true or false
        \begin{enumerate}[label={(\alph{enumi}.\arabic*)}]
            \item The Dutchman is entitled to a reduction of the filing fee under R.7a(1) when the European patent application is filed in Dutch, and a translation into English is filed at the same time.
            \item The Dutchman is entitled to a reduction of the filing fee under R.7a(1) when the European patent application is filed in Swedish, and a translation into English is filed at the same time.
            \item The Dutchman is entitled to a reduction of the filing fee under R.7a(1) when the European patent application is filed in Polish, and a translation into English is filed at the same time.
            \item The Dutchman is entitled to a reduction of the filing fee under R.7a(1) when the European patent application is filed in German, and a translation into English is filed at the same time.
            \item The reduction of the filing fee that the Dutchman is entitled to under R.7a(1) if the relevant requirements as to languages, translations and time limits are met is 20\% of the filing fee, including a 20\% reduction of any "page fees".
            \item The reduction of the filing fee that the Dutchman is entitled to under R.7a(1) if the relevant requirements as to languages, translations and time limits are met is 30\% of the filing fee, including a 30\% reduction of any "page fees".
            \item The reduction of the filing fee that the Dutchman is entitled to under R.7a(1) if the relevant requirements as to languages, translations and time limits are met is 50\% of the filing fee, including a 50\% reduction of any “page fees".
            \item The Dutchman is (also) entitled to a reduction of the filing fee under R.7a(3) when the European patent application is filed in any language, and, where applicable, a translation into an official EPO language is filed at the same time.
        \end{enumerate}
        
        \item For each of the statements below, indicate whether the statement is true or false.
        \begin{enumerate}[label={(\alph{enumi}.\arabic*)}]
            \item The Dutchman is entitled to a reduction of the filing fee under R.7a(1) when the European patent application is filed in Dutch, and a translation into French is filed at the same time.
            \item The Dutchman is entitled to a reduction of the filing fee under R.7a(1) when the European patent application is filed in Dutch, and a translation into German is filed at the same time.
            \item The Dutchman is entitled to a reduction of the filing fee under R.7a(1) when the European patent application is filed in Dutch, and a translation into English is filed a week later.
            \item The Dutchman is entitled to a reduction of the filing fee under R.7a(1) when the European patent application is filed in English, and a translation into Dutch is filed a week later.
        \end{enumerate}
        
        \item \textit{Upon filing the EP application in Dutch by the Polish friend, the friend also pays all necessary fees.} \\
        For each of the statements below, indicate whether the statement is true or false.
        \begin{enumerate}[label={(\alph{enumi}.\arabic*)}]
            \item The EP application is validly filed when the EP application is filed in Dutch, and a translation into English is filed at own motion, without an invitation from the EPO, one week later.
            \item The EP application is validly filed when the EP application is filed in Dutch, and a translation into English is filed six weeks of filing the application.
            \item The EP application is validly filed when the EP application is filed in Dutch, and a translation into English is filed three months after of filing the application.
            \item If no translation is filed within 1 month of filing the EP application in Dutch, the EPO will issue an invitation to file a translation into an official EPO language within a time limit of two months.
        \end{enumerate}
    \end{enumerate}

    \item % Question 2
    B4-07
    \begin{enumerate}[label=(\alph*)]
        \item Which documents filed after filing the European patent application may a Dutch person file in Dutch?
        \item Must a translation be filed? In what language? What period applies?
        \item What is the consequence if a required translation is not filed?
    \end{enumerate}

    \item % Question 3
    B4-08 \quad T/F question (Basic selection) \\
    A European patent EP1 has been granted in English.
    
    \begin{enumerate}[label=(\alph*)]
        \item Mr. Jansen, a Dutch national living in the Netherlands, wants to file an opposition against European patent EP1. \\
        For each of the statements below, indicate whether the statement is true or false.
        \begin{enumerate}[label={(\alph{enumi}.\arabic*)}]
            \item Mr. Jansen may file the notice of opposition in Dutch.
            \item Mr. Jansen may file the notice of opposition in English.
            \item Mr. Jansen may file the notice of opposition in German.
            \item Mr. Jansen may file the notice of opposition in Chinese.
            \item Mr. Jansen may file the notice of opposition in Korean.
        \end{enumerate}
        
        \item Mr. Xiao, a Chinese national living in the Netherlands, wants to file an opposition against European patent EP1. \\
        For each of the statements below, indicate whether the statement is true or false.
        \begin{enumerate}[label={(\alph{enumi}.\arabic*)}]
            \item Mr. Xiao may file the notice of opposition in Dutch.
            \item Mr. Xiao may file the notice of opposition in English.
            \item Mr. Xiao may file the notice of opposition in German.
            \item Mr. Xiao may file the notice of opposition in Chinese.
            \item Mr. Xiao may file the notice of opposition in Korean.
        \end{enumerate}
        
        \item Mrs. Lee, a Korean national living in Korea, wants to file an opposition against European patent EP1 via Mr. Koch, her European patent attorney based in Munich. \\
        For each of the statements below, indicate whether the statement is true or false.
        \begin{enumerate}[label={(\alph{enumi}.\arabic*)}]
            \item Mrs. Lee may file the notice of opposition in Dutch through Mr. Koch.
            \item Mrs. Lee may file the notice of opposition in English through Mr. Koch.
            \item Mrs. Lee may file the notice of opposition in German through Mr. Koch.
            \item Mrs. Lee may file the notice of opposition in Chinese through Mr. Koch.
            \item Mrs. Lee may file the notice of opposition in Korean through Mr. Koch.
        \end{enumerate}
    \end{enumerate}

    \item % Question 4
    B4-13 \quad T/F question (Advanced selection) \\
    A European patent application has been filed in Swedish by a Danish applicant living in Sweden. A translation of the patent application into English was duly filed. In reply to a communication from the Examining Division setting a period of 4 months to respond, the applicant files, on the last day of the period:
    \begin{itemize}
        \item a letter with his observations in Danish,
        \item in an Annex amendments to the claims in Swedish, and 
        \item in a further Annex as evidence a Japanese patent application in Japanese.
    \end{itemize}
    
    \begin{enumerate}[label=(\alph*)]
        \item For each of the statements below, indicate whether the statement with regard to the letter is true or false.
        \begin{enumerate}[label={(\alph{enumi}.\arabic*)}]
            \item The letter is filed in a non-admissible language, so the letter is deemed not filed.
            \item A translation of the letter has to be filed within 1 month of filing the letter. The translation has to be into English.
            \item A translation of the letter has to be filed within 1 month of filing the letter. The translation has to be into any one of English, French or German.
            \item If the applicant fails to file a translation within the required time limit, the letter is deemed not filed.
            \item If the applicant fails to file a translation of the letter nor of the claims within the required time limit, the application is deemed withdrawn.
        \end{enumerate}
        
        \item For each of the statements below, indicate whether the statement with regard to the Annex with amendments to the claims is true or false.
        \begin{enumerate}[label={(\alph{enumi}.\arabic*)}]
            \item The claims are filed in a non-admissible language, so the amended claims are deemed not filed.
            \item A translation of the amended claims has to be filed within 1 month of filing the amended claims. The translation has to be into English.
            \item A translation of the amended claims has to be filed within 1 month of filing the amended claims. The translation has to be into any one of English, French or German.
            \item If the applicant fails to file a translation within the required time limit, the amended claims are deemed not filed.
            \item If the applicant fails to file a translation of the claims nor of the letter within the required time limit, the application is deemed withdrawn.
        \end{enumerate}
        
        \item For each of the statements below, indicate whether the statement with regard to the Annex with the Japanese patent application is true or false.
        \begin{enumerate}[label={(\alph{enumi}.\arabic*)}]
            \item The Japanese patent application is filed in a non-admissible language, so the Japanese patent application is deemed not filed.
            \item A translation of the Japanese patent application has to be filed within 1 month of filing the letter. The translation has to be into English.
            \item A translation of the Japanese patent application has to be filed within 1 month of filing the letter. The translation has to be into any one of English, French or German.
            \item If the applicant fails to file a translation within the required time limit, the Japanese patent application is deemed not filed.
            \item If the applicant fails to file a translation within the required time limit, the application is deemed withdrawn.
        \end{enumerate}
    \end{enumerate}

    \item % Question 5
    B4-17 (Basic selection) \\
    An Italian applicant files two European patent applications, EP1, which claims the priority of Italian application IT1, and EP2, which claims the priority of Italian application IT2. 
    EP1 is filed in Italian and is identical to IT1. An English translation of EP1 is supplied to the EPO two weeks later.
    EP2 is filed in English. It is the translation of IT2. 
    During the examination procedure, the English texts of EP1 and EP2 are found to contain translation errors.
    
    \begin{enumerate}[label=(\alph*)]
        \item Would it be possible to correct these errors for EP1?
        \item Would it be possible to correct these errors for EP2?
    \end{enumerate}

    \item % Question 6
    B5-08 \quad T/F question (Basic selection) \\
    A Dutch national lives in the USA and wishes to file a European patent application.
    
    \begin{enumerate}[label=(\alph*)]
        \item For each of the statements below, indicate whether the statement is true or false.
        \begin{enumerate}[label={(\alph{enumi}.\arabic*)}]
            \item He does not need to appoint a professional representative for filing a European patent application in Dutch at the EPO, but he can do so.
            \item He does not need to appoint a professional representative for filing a European patent application in English at the EPO, but he can do so.
        \end{enumerate}
        
        \item For each of the statements below, indicate whether the statement is true or false.
        \begin{enumerate}[label={(\alph{enumi}.\arabic*)}]
            \item He does not need to appoint a professional representative for paying the filing fee and search fee to the EPO, but he can do so.
            \item He does not need to appoint a professional representative for paying the filing fee and search fee to the EPO if he pays the fees simultaneously with filing the application, but he shall appoint a professional representative when the fees are paid later.
            \item He is obliged to let a professional representative pay the filing fee and search fee to the EPO.
        \end{enumerate}
        
        \item \textit{The Dutch national living in the USA files a European patent application at the EPO in Dutch.} \\
        For each of the statements below, indicate whether the statement is true or false.
        \begin{enumerate}[label={(\alph{enumi}.\arabic*)}]
            \item He does not need to appoint a professional representative when he files a translation in English 3 weeks after the filing date, but he can do so.
            \item He does not need to appoint a professional representative when he files a translation in English simultaneously with filing the application in Dutch.
        \end{enumerate}
        
        \item \textit{The Dutch national files only a fully completed request form, but does not file a description nor a reference to a previously filed application, nor any claims.} \\
        For the statement below, indicate whether the statement is true or false.
        \begin{enumerate}[label={(\alph{enumi}.\arabic*)}]
            \item He does not need to appoint a professional representative when he files a description 5 days after having filed the request form.
        \end{enumerate}
        
        \item \textit{The Dutch national files only a fully completed request form as well as a description but no claims.} \\
        For the statement below, indicate whether the statement is true or false.
        \begin{enumerate}[label={(\alph{enumi}.\arabic*)}]
            \item He does not need to appoint a professional representative when he files a set of claims 5 days after having filed the request form.
        \end{enumerate}
    \end{enumerate}

    \item % Question 7
    B5-09 (Basic selection) \\
    Who is deemed to be the common representative for a European patent application with three applicants, where no common representative has been appointed, in the following situations?
    
    \begin{enumerate}[label=(\alph*)]
        \item All applicants are from a Contracting State and none has appointed a professional representative.
        \item All applicants are from a Contracting State and applicant 3 has appointed a professional representative.
        \item Applicant 1 and 3 are from a Contracting State; applicant 2 is a US company; no professional representative has been appointed.
        \item Applicant 1 and 3 are from a Contracting State; applicant 2 is a US company; applicant 1 has appointed a professional representative.
        \item Applicant 1 and 3 are from a Contracting State; applicant 2 is a US company; applicant 3 has appointed a professional representative.
    \end{enumerate}

\end{enumerate}
\end{enumerate}

\begin{center}

\p \p \p
 
\end{center}

\begin{enumerate}[label=\textbf{Answer \arabic*}]

    \item % Question 1
   An Austrian applicant A files an international application in German with the EPO. Before publication of the application, A sells this international application to US-company B, based in San Francisco, USA.
You are a European patent attorney representing company A and, after the purchase, company B in respect of this application.


For each of the statements below, indicate whether the statement is true or false:

    \begin{enumerate}[label=(\alph*)]
        \item  \textbf{\textsc{True}} --- [\textbf{R. 92\textit{bis}.1}] deals with recording of changes in certain indications
in the Request or the Demand. Data changes on Applicant/Common Representative/Common Agent will be recorded by the IB up until 30 months after priority date. 
\item  \textbf{\textsc{True}} --- it may be filed with the rO (in this case, the EPO), but it is strongly recommended to go directly to the IB especially if closer to the 30-month deadline. 
\item  \textbf{\textsc{False}} --- [\textbf{R. 92\textit{bis}.1(b)} \& \textsc{AG--IP 11.021}] make it clear that after the deadline, the change must be requested directly with the dO/eO. 
\item \textbf{\textsc{False}} --- if change is to be taken into account before international publication, it should reach the IB before deadline for technical preparations, i.e., 1 day before the 15$^{\mathrm{th}}$ day before scheduled publication (see also: [\textsc{AG--IP 9.014}]).
\item \textbf{\textsc{False}} --- [\textbf{R. 92\textit{bis}.1(b)} \& \textsc{AG--IP 11.021}] make it clear that after the deadline, the change must be requested directly with the dO/eO, regardless of exact limit for entry into national/regional phase. 
    \end{enumerate}



    \item % Question 2
An applicant resident in Spain filed an international application in Spanish at the Spanish national Office indicating the EPO as International Searching Authority. Within one month the applicant furnishes pursuant to PCT Rule 12.3(a) a translation of the application into English.
    \begin{enumerate}[label=(\alph*)]
        \item  Spanish --- because it is a language of publication and the PCTa was filed in the language of publication [\textbf{R. 48.3(a)}].
\item  The declaration under \textbf{Art. 17.2(a)} [\textit{i.e., non-patentability}], the title of invention, the abstract and any text matter pertaining to the figure accompanying the abstract shall be published in English by the IB [\textbf{R. 48.3(a)}].
    \end{enumerate}

   \item % Question 3
A Dutch applicant filed a European patent application with the EPO on Friday 11 May 2018 comprising just a description and drawings, but no claims. On the same day, he filed the same documents with the IB together with a PCT Request Form. On Wednesday 23 May 2018, he filed a set of claims with the EPO and also with the IB, referring to his earlier submissions of 11 May 2018.


For each of the statements below, indicate whether the statement is true or false:

The filing date accorded to the European patent application is...


    \begin{enumerate}[label=(\alph*)]
        \item  … 11 May 2018. –––   \textbf{\textsc{True}}. For EP you don't need a claim to accord a filing date. Assuming other formalities such as name of Applicant, indication a patent is sought, etc., have also been provided.
\item … 23 May 2018. 
    
\vspace{0.5cm}

The filing date accorded to the international patent application is ...
 
        \item  … 11 May 2018.
\item … 23 May 2018. –––   \textbf{\textsc{True}}. For PCT, you need a claim to accord a filing date. He filed the claim/s within the 2-month deadline of \textbf{Rule 20.3(b)(i)}, so he will get a filing date on the later date within this 2-month window.
    \end{enumerate}
    \item % Question 4
An international application is jointly filed at the German national Office on 3 April 2017 by a British applicant and a German applicant claiming priority from an earlier German national application filed on 4 April 2016.


For each of the statements below, indicate whether the statement is true or false:

    \begin{enumerate}[label=(\alph*)]
        \item  \textbf{\textsc{False}} --- [\textbf{R. 26.2\textit{bis}(b)} \& \textsc{AG--IP 5.032.}] make it clear that it is sufficient for one Applicant to be indicated in the Request.
\item \textbf{\textsc{True}} --- [\textbf{R. 26.2}] makes it clear that it is sufficient for one Applicant to sign the Request. It is irrelevant whether he is entitled to file the PCTa.
\item \textbf{\textsc{True}} --- [\textbf{R. 26.1} \& \textbf{R. 26.2}] provide for an invitation of corrections from \textbf{Art. 14(1)(b)} to be sent within 1 month of date of receipt, with a time limit of \textbf{2 months from the date of invitation} to provide the corrections.
\vspace{0.5cm}

Later on the applicants wish to withdraw the priority claim.
 
 \item  \textbf{\textsc{False}} --- [\textbf{R. 90\textit{bis}.3}] makes it clear that a priority claim may be withdrawn at any point \newline \underline{before 30 months after the priority date}. In this case: \textsc{04.10.2018}.
\item \textbf{\textsc{False}} --- [\textbf{R. 90\textit{bis}.5}] makes it clear that a (deemed) common representative may not sign any notice of withdrawals on behalf of all the applicants. 
\item  \textbf{\textsc{False}} --- an appointed common representative may not withdraw a priority claim. [\textbf{R. 90\textit{bis}.5}] makes it clear that \underline{all} Applicants must sign the withdrawal notice. ??? See also [\textsc{AG--IP 11.056.} ]


\vspace{0.5cm}

Later on the international application enters the regional phase before the EPO.


\item \textbf{\textsc{True}} --- [\textbf{R. 163(4) EPC}]??? --- \textit{Where, at the expiry of the period under Rule 159, paragraph 1, the address, the nationality or the State in which their residence or principal place of business is located is missing in respect of any applicant, the European Patent Office shall invite the applicant to furnish these indications within two months.}


    \end{enumerate}
    \item % Question 5

An international application is filed on 28 July 2018 indicating a priority date of 27 January 2018. The applicant wishes to correct the priority date to 26 September 2017 and asks you until when may he can request the competent authority of a correction of the priority date. Today is 12 November 2018.

For each of the statements below, indicate whether the statement is true or false.

The last day to request the correction is...


    \begin{enumerate}[label=(\alph*)]
        \item  Wednesday 28 November 2018. --- \textbf{True} --- 4 months from international filing date. (+ 1 month maybe under R. 26bis.2(b)??? ask Zsofia
\item Monday 28 January 2019. --- \textbf{False} --- red herring as this is 16 months from the "new" priority date, but it is superceded by the earlier date of \textit{(a)}. Relevant Rule is [\textbf{R. 26\textit{bis}}] -- \textit{Correction or Addition of Priority Claim}.
\item Monday 27 May 2019.   --- \textbf{False}.
\vspace{0.5cm}
    

The request may be filed with…
 
        \item the receiving Office --- \textbf{\textsc{True}} --- [\textbf{R. 26\textit{bis}.1}] makes it clear that either rO or IB can be used.
\item the International Bureau --- \textbf{\textsc{True}} --- [\textbf{R. 26\textit{bis}.1}] makes it clear that either rO or IB can be used.



    \end{enumerate}

\item % Q6

    \begin{enumerate}[label=(\alph*)]
        \item  Yes, a priority claim for a PCTa may be withdrawn at any point until 30 months after the priority date.
\item rO, IB or IPEA under \textbf{Art. 39(1)}. All Applicants must sign the withdrawal.
\item All withdrawals are free of charge [\textsc{AG--IP 11.048}].

\item International publication may be postponed by withdrawing the priority claim, as any time-limit computation (which has not already expired) then takes into account the "new" priority date after withdrawal. [\textbf{R. 90\textit{bis}.3(b)}].

     \end{enumerate}

\item % Q7

Today, Monday 27 November 2017, an applicant wants to file an international application at the EPO as receiving Office claiming the priority of an earlier national application filed on 24 September 2016.

    \begin{enumerate}[label=(\alph*)]
        \item  No, --- [\textbf{R. 26\textit{bis}.3}] provides for restoration of the right of priority by the rO \newline \underline{within 2 months after the priority year ends}. In this case, we have $>$2 months by 1 day.
\item  If he were able to do it, he would have to (i) file the request with the rO within the 2-month deadline; and (ii) state the reasons for failure to file on time; and (iii) preferably provide a declaration and/or evidence. He would also have to pay a fee depending on rO, and add the priority claim to the PCTa (if applicable) [\textbf{R. 26\textit{bis}.3}].
\item  No; not all rOs accept such requests. Notable exceptions include \textsc{DE}, \textsc{KR} and \textsc{IN}. 

\item No; no guarantees are made for the regional phase regarding the restored priority claim. [\textbf{R. 49\textit{ter}.2(e)}] provides for the dO having to give the Applicant an opportunity to make observations on any such intended refusal.
\item The EPO applies the criterion of \underline{"due care"} [can always check this in \textsc{Annex C}]. \textit{Due care is considered to have been taken if non-compliance with the time limit results either from exceptional circumstances or from an isolated mistake within a normally satisfactory monitoring system.}

     \end{enumerate}
       \end{enumerate}

    \section{October 09, 2025}

\begin{enumerate}[label=\textbf{Answer \arabic*}]

    \item % Question 1
    L3-06 - T/F question (Basic selection) \\
    An international application is filed by a Danish national at the Danish Patent and Trademark Office as receiving Office. \\
    For each of the statements below, indicate whether the statement is true or false.
    \begin{enumerate}[label=(\alph*)]
        \item The IA may be filed in…
        \begin{enumerate}[label={(\alph{enumi}.\arabic*)}]
            \item ... Danish --- \textbf{True}
            \item ... English --- \textbf{True}
            \item ... Swedish --- \textbf{True}
        \end{enumerate}

        \item The applicant may select as ISA:
        \begin{enumerate}[label={(\alph{enumi}.\arabic*)}]
            \item ... the Danish national office. --- \textbf{False}
            \item ... the EPO --- \textbf{True}
            \item ... the Swedish national office. --- \textbf{True}
            \item ... the Nordic Patent Institute. --- \textbf{True}
        \end{enumerate}
        
        \item \textit{The application is filed in Danish.} \\
        For each of the statements below, indicate whether the statement is true or false
        
        \vspace{0.5em}
        The EPO as ISA…
        \begin{enumerate}[label={(\alph{enumi}.\arabic*)}]
            \item ... does not require a translation for the purposes of international search. --- \textbf{False}
            \item ... requires a translation into any one of the 10 publication languages for the purposes of international search. --- \textbf{False}
            \item ... requires a translation into English, French or German for the purposes of international search. --- \textbf{True}
            \item[] \textit{The Nordic Patent Institute as ISA...}
            \item ... does not require a translation for the purposes of international search. --- \textbf{True}
            \item ... requires a translation into any one of the 10 publication languages for the purposes of international search. --- \textbf{True}
            \item ... requires a translation into English, French or German for the purposes of international search. --- \textbf{False}
        \end{enumerate}

        \item \textit{The applicant has filed the international application in Danish and has indicated the EPO as International Searching Authority. The applicant is in the progress of making a translation into English. The application was filed by fax with the Danish Patent and Trademark Office on a day on which the Danish Patent and Trademark Office and the International Bureau were closed.} \\
        For each of the statements below, indicate whether the statement is true or false: \\
        The translation must be filed ...
        \begin{enumerate}[label={(\alph{enumi}.\arabic*)}]
            \item ... within one month from the date of receipt by the Danish Patent and Trademark Office to avoid any late furnishing fee(s). --- \textbf{True}
            \item ... within one month from the date of receipt by the EPO to avoid any late furnishing fee(s). --- \textbf{False}
            \item ... when the 1m period was missed, the translation may still be provided as long as a late furnishing fee is paid.
            --- \textbf{True}
            \item ... when the rO issues an invitation to supply a missing translation, the ultimate time limit to respond is always one month from the date of the invitation. --- \textbf{False}
        \end{enumerate}
        
        \item \textit{The applicant has timely filed the translation into English to the Danish Patent and Trademark Office.} \\
        The international publication of the international application will take place in…
        \begin{enumerate}[label={(\alph{enumi}.\arabic*)}]
            \item ... Danish. --- \textbf{False}
            \item ... English. --- \textbf{True}
            \item ... in Danish with the title and the abstract also in English. --- \textbf{False}
        \end{enumerate}
    \end{enumerate}

    \item % Question 2
    L3-08 - T/F question (Basic selection) \\
    A PCT application was filed with the EPO. As the International Searching Authority, the EPO considered that the application was not unitary. The invention first mentioned in the claims was searched and an invitation to pay two additional international search fees was sent to the applicant last week, 8 March 2023. The third invention, which has not yet been searched, is the only invention that the applicant would like to pursue in the European phase before the EPO. \\
    Today is 17 March 2023.
    \begin{enumerate}[label=(\alph*)]
        \item For each of the statements below, indicate whether the statement is true or false:
        \begin{enumerate}[label={(\alph{enumi}.\arabic*)}]
            \item In the international PCT phase, the applicant can file a protest with the EPO and request that a full search is made. The protest is free of charge, but it has to be supported by arguments. --- \textbf{False} --- a protest fee has to be paid under [\textbf{Rule 40}].
            \item The applicant can timely pay one additional fee for the third invention to be searched in the international PCT phase. In the European phase, the applicant can limit the application to the third invention. --- \textbf{True}
            \item The applicant can ignore the invitation. In the European phase, the applicant can file a divisional application directed to the third invention. --- \textbf{True}
            \item The applicant can ignore the invitation. In the European phase, the applicant will again receive an invitation to pay additional search fees. --- \textbf{True}
        \end{enumerate}
        
        \item (from L3-14)
        \begin{enumerate}[label={(\alph{enumi}.\arabic*)}]
            \item The purpose of the requirement of "unity of invention" is to avoid people getting multiple inventions patented for the price of one search. Relevant provisions are: [\textbf{Art. 17.3}], [\textbf{Art. 34.3}], [\textbf{R. 13}], [\textbf{R. 40}],
            \item What is the time limit to pay the additional search fee in the international phase? --- within 1 month from date of invitation [\textbf{R. 40.1(ii)}].
            \item To whom must the additional search fee be paid? --- payable directly to the ISA [\textbf{Art. 40.2(b)}],
            \item A "protest fee" may be paid if you don't agree with the assessment of lack of unity. Accompanied by a reasoned statement as to how the \textbf{application is unitary} or as to how the \textbf{additional fees are excessive}. The time limit for the protest fee is \underline{1 month from the date of the invitation} to pay additional search fees. 
        \end{enumerate}
    \end{enumerate}

    \item % Question 3
    L3-16 
    \begin{enumerate}[label=(\alph*)]
        \item The time limit for establishing the ISR [or the declaration under \textbf{Art. 17(2)(a)} that there won't be an ISR] is \textbf{3 months from the receipt of the search copy} by the ISA \textit{or} \textbf{9 months from priority date} -- whichever expires \textsc{later} [\textbf{R. 42}].
        \item The contents of the international search report (under \textbf{Rule 43}) are: (i) identification of the ISA; (ii) date of ISR completion and relevant priority date; (iii) classification of subject-matter; (iv) citations of relevant documents (relevant by claim and with specific passages in the prior art); (v) classification of fields searched; (vi) remarks considering unity; (vii) name of officer.
        \item After receiving the international search report, the Applicant may amend the claims (only the claims and only once) under \textbf{Art. 19}. There are no direct provisions for commenting/responding to the ISR otherwise, other than submitting comments to the IB on an informal basis. Any informal comments received after 30 months from the priority date will only be kept in the file of the International Bureau and not be transmitted to the designated Offices. [see also \textsc{AG--IP 7.030}].
        \item The time limit for \textbf{Art. 19} amendments is \textbf{2 months from transmittal of ISR to the IB} or \textbf{16 months after priority date} -- whichever comes \textsc{later}. They must be sent directly to the IB.
        \item The amendments and any accompanying letter or statement are to be filed in a form \textit{reasonably free from erasures and shall be free from alterations, overwritings, and interlineations} [\textbf{R. 11.12}]. The language of the amendments shall be the language of publication if the PCTa has been filed in a language other than the language in which it is published [\textbf{R. 46.3}]. The statement shall be in the language in which the PCTa is published and shall not exceed 500 words if in/or translated to English.    [\textbf{R. 46.4}]. Replacement sheets with a complete set of claims outlining amendments and a letter mentioning the amendments and indicating basis therefor shall be included [\textbf{R. 46.5}].
        \item Amendments to the claims under PCT \textbf{Article 19} are not allowed when the ISA has declared under \textbf{Art. 17(2)(a)} that there will be no ISR [see also \textsc{AG--IP 9.004}]. Amendments are allowed to claims that were searched even if other claims were considered unsearchable.
    \end{enumerate}

    \item % Question 4
    L3-20 (Basic selection) \\
    An applicant has filed an international application on 9 May 2023 as a first filing at the EPO as receiving Office, in English. The EPO acted as International Searching Authority and the international search report was transmitted to the applicant in December 2023; the written opinion (WO-ISA) suggests that the invention is patentable. \\
    The applicant being afraid that there is prior art which was not discovered by the EPO as ISA, considers filing a request for supplementary international search to be carried out by the Intellectual Property Office of Singapore.
    \begin{enumerate}[label=(\alph*)]
        \item The applicant may file a request for supplementary international search [\textsc{PCT/IB/375}] until \textbf{22 months after the priority date} -- in this case; \textit{10 March 2025}. The request should be filed \textbf{directly with the IB} in \underline{English} or in \underline{French}. [\textbf{R. 45\textit{bis}}] and [\textsc{AG--IP 8.007}].
        \item Fees to be paid include \textbf{(i)} the supplementary search fee for the benefit of the Authority specified for supplementary search (here the \textsc{SG} office); and \textbf{(ii)} the supplementary search handling fee for the benefit of the International Bureau.
        They must \textsc{both} be paid to the IB within \textbf{1 month of the receipt of the request for SISR} by the IB.
        The amount of the fees is \textbf{(i)} 200 \textsc{CHF} and \textbf{(ii)} 1458 \textsc{CHF} [found in \textsc{Annex SISA} of the \texttt{SG} office.]
        \item The supplementary international search will start ``\textit{promptly after receipt of the documents.}'' The SISA may also wait until receipt of the ISR and WO--ISR, and even up to 22 months after the priority date [\textbf{Rule 45\textit{bis}.5}].
        \item The supplementary international search report is to be established within \textit{28 months of the priority date}. [\textbf{R. 45\textit{bis}.7}]. \underline{\textsc{No written opinion is established}} with the supplementary international search report, but additional comments on the prior art found in the ISR may be included. [see also \textsc{AG--IP 8.049}].
        \item The supplementary international search report does not repeat relevant prior art documents which have already been cited in the international search report, unless this is necessary because of new relevance when read in conjunction with other documents discovered during the supplementary international search. On occasion, the supplementary international search report may contain more detailed explanations concerning citations of documents than those in the main international search report.
        \item The supplementary international search report is not published \textit{per se} nor as part of the international publication. Nevertheless, once the international application has been published, and the supplementary international search report has been received, it is \textit{made available for public inspection} by the International Bureau on \textsc{PATENTSCOPE}. 
    \end{enumerate}

    \item % Question 5
    L3-27 \\
    A US applicant wants to request a demand for international preliminary examination at the USPTO for an international application filed on 29 December 2016 without any claiming priority. The ISR and WO-ISA, established by the USPTO as ISA, were transmitted to the applicant on 26 June 2018. \\
    However, due to a severe hurricane, the USPTO was closed on 29 and 30 October 2018, and the applicant was not able to submit the demand on 29 October 2018. \\
The demand may still be validly filed on the next working day of 31 October 2018. This falls under \textit{natural calamity} mentioned under [\textbf{R. 82\textit{quater}}] pertaining to \textit{force majeure} events. \newline Only \underline{priority} and \underline{entering national phase} cannot be excused due to \textit{force majeure}. So filing a demand for IPEA will be accepted. 

    \item % Question 6
    In which cases can the EPO act as IPEA? \\
    (multiple choice)
    
    \begin{enumerate}[label=(\alph*)]
        \item Spanish office was the ISA --- \textbf{True}
        \item Nordic Patent Institute was the ISA --- \textbf{True}
        \item USPTO was the ISA --- \textbf{False}
        \item EPO was the ISA --- \textbf{True}
        
        \underline{\texttt{Annex E -- IPEA}:} \\
        \texttt{
The EPO acts as International Preliminary Examination Authority under the condition only if the international search is or has been performed by the EPO acting as International Searching Authority or another International Searching Authority located in and operating for any State party to the European Patent Convention has prepared the international search report (Annex A EPO-WIPO Agreement). These International Searching Authorities are the Austrian Patent Office, the Finnish Patent and Registration Office (PRH), the Nordic Patent Institute, the Spanish Patent and Trademark Office, the Swedish Intellectual Property Office (PRV), the Turkish Patent and Trademark Office (Turkpatent) or the Visegrad Patent Institute.}
    \end{enumerate}

    \item % Question 7
    What documents must an applicant file when filing Article 19 amendments? \\
    (single choice)
    \begin{enumerate}[label=(\alph*)]
        \item Only the amended claims \& a letter indicating the differences plus the basis for the amendments --- \textbf{False}
        \item A complete set of claims in replacement of the claims originally filed \& a letter indicating the differences plus the basis for the amendments  --- \textbf{True}. [\textbf{Art. 19(1)}] specifies that the statement regarding impact, etc., is optional, while [\textbf{R. 46(5)}] makes it clear that the replacement sheets have to contain a complete set of claims and an accompanying letter shall identify the amended claims, the differences, any cancelled claims and indicate basis for amendments.
        \item Only the amended claims \& a letter indicating the differences plus the basis for the amendments \& statement by the applicant explaining the amendment and indicating any impact it might have on the description and the drawings --- \textbf{False}
        \item A complete set of claims in replacement of the claims originally filed \& a letter indicating the differences plus the basis for the amendments \& statement by the applicant explaining the amendment and indicating any impact it might have on the description and the drawings --- \textbf{False}
    \end{enumerate}


    \section{October 20, 2025}

\begin{enumerate}[label=\textbf{Answer \arabic*}]

    \item % Question 1
    L4-06 (Basic selection) \\
    A US applicant, resident in the US, files his international application at the USPTO. Subsequently he wishes to pursue his application before the EPO as designated office.
    \begin{enumerate}[label=(\alph*)]
        \item The US applicant can initiate the national entry procedure himself; but cannot do any subsequent steps himself [\textbf{Art. 27(7) PCT}; \textbf{R. 51\textit{bis} PCT}; \textbf{Art. 133 EPC}]. ``\texttt{However, up to expiry of the 31-month time limit under Rule 159
EPC, non-resident applicants may either comply with any
requirement themselves or act through a professional
representative entitled to practise before the EPO. This means
that, within the 31-month time limit, non-resident applicants may
themselves sign and file EPO Form 1200, submit amendments, file
a translation of the application, file a request for early processing,
etc.}' -- \textsc{Euro-PCT Guide 5.3.008}'
        \item ''\texttt{Where applicants have failed to appoint a professional
representative as required, they will be invited by the EPO to do so
within a time limit of two months. Until the EPO is informed of a
(valid) appointment, any procedural step taken by such applicants
will be deemed not to have been taken. If the deficiency is not
corrected in time, the application will be refused.}`` --- \textsc{Euro-PCT Guide 5.3.015}; \textbf{R. 163(5), (6) EPC}.
        \item  The minimum acts [\textbf{R. 159 EPC}] to enter the regional phase before the EPO \textit{for surcharges to be avoided}: 
        \begin{enumerate}[label=(\roman*)]
         \item no translation needed in this case as it's in English from the USPTO;
\item pay filing fee;
\item pay the designation fee (and any extension/validation fees, if applicable);
\item pay the search fee (EPO needs to do a search);
\item file request for examination if period under \textbf{R. 70(1) EPC} has already expired;
\item pay the renewal fee for the third year;
\item (not always applicable) [\textbf{DOO--DOO}] file sequence listing if not available to EPO;
\item (not always applicable) file the certificate of exhibition. 


        \end{enumerate}

    \end{enumerate}

    \item % Question 2
    L4-12 - T/F question (Basic selection) \\
    An international application was published in Korean together with the international search report established by the Korean patent office as ISA. The international application does not claim priority. The applicant wants to enter the European regional phase today, the last day of the 31-month period without any penalty fees. \\
    Which fees need to be paid within what time limit to avoid any penalty fees?

    \vspace{1em}
    For each of the statements below, indicate whether the statement is true or false.
    
    \vspace{0.5em}
    The \textbf{examination fee}….
    \begin{enumerate}[label=(\alph*)]
        \item ... needs to be paid upon entry today. --- \textbf{True}. No priority claimed. ISA was published about 12 months ago, so period of [\textbf{R. 70(1)}], i.e., 6 months, has long expired.
        \item ... may be paid, without any additional fee, until 6 months from the publication of the translation of the international application under Art.153(4) EPC, first sentence, as the publication of the translation comprises the translation of the international search report. --- \textbf{False}.
        \item ... may be paid, without any additional fee, until 6 months from the mention of the supplementary European search report in the European patent Bulletin. --- \textbf{False}.
        \item ... will be refunded if the applicant already paid the examination fee on entry and withdraws the application shortly after receipt of the supplementary European search report. --- \textbf{True}. \newline
        
       \adforn{61} [\textbf{R. 159(1) EPC}] -- necessary acts for entry into Euro phase. \newline
       \adforn{61} [\textbf{Rfees 11}] -- refund of examination fee (full or partial).
    \end{enumerate}

    \item % Question 3
    L4-22 \\
    How many claims fees must the applicant pay or does he get refunded, and when, in the following cases:
    
    \vspace{1em}
    \textbf{Case 1:} He must pay a further 7 claims fees. In this case, he ignored it, so the claims 21--27 are deemed abandoned. 
    
    \vspace{1em}
    \textbf{Case 2:} Calculated on the basis of the amended set. So now he owes 17--5=12 claims fees, otherwise those claims will be deemed to have been abandoned. 

    \item % Question 4
    L4-23 (Basic selection) \\
    An international application is filed on 7 September 2015 without claiming priority. The applicant wants to enter the regional phase before the EPO as designated Office. Today is 26 February 2018. \\
    What is the last day on which the renewal fee in respect of the third year can be paid to the EPO…
    \begin{enumerate}[label=(\alph*)]
        \item ... without additional fee? --- { \textsc{9 APR 2018}}
        \item ... with additional fee? --- { \textsc{9 OCT 2018}}
    \end{enumerate}

    \textsc{9 APR 2018} is the 31-month deadline. 
    \textsc{30 SEP 2017} is the renewal date (\textit{\textbf{de ultimo ad ultimo}}), where it just goes to the end of the month regardless.
    
    The first is the \underline{later} one. 
    
    Then for 6-month extension we \textbf{DO NOT} have \textit{de ultimo ad ultimo}. So it goes to \textsc{9 OCT 2018}.
    
    \nt{Summary}{This is the case where \textbf{there is no priority}. \\This means the 31-month date                will always be the \textbf{later date}. \\
    Standard [\textbf{R. 134 EPC}] applies, i.e., move to the next working day if Office closed for both this 31-month calculation \underline{and} the 6-month ``aggregate period''. \\ The 31-month due date is the \textsc{end of a period} so then we use 6-month extension \textbf{without} \textit{de ultimo ad ultimo}.}                
    
    \vspace{1em}
    L4-24 \\
    An international application is filed on 9 June 2016 claiming priority of a national application filed on 7 September 2015. The applicant wants to enter the regional phase before the EPO as designated Office. Today is 26 February 2018. \\
    What is the last day on which the renewal fee in respect of the third year can be paid to the EPO…
    \begin{enumerate}[label=(\alph*), resume]
        \item … without additional fee? --- \textsc{2 JUL 2018}
        \item … with additional fee? --- \textsc{2 JAN 2019}
    \end{enumerate}

     \textsc{9 APR 2018} is the 31-month deadline. 
    \textsc{30 JUN 2018} is the renewal date (\textit{\textbf{de ultimo ad ultimo}}), where it just goes to the end of the month regardless. \textsc{30 JUN 2018} is a Saturday, so the final deadline is \textsc{2 JUL 2018}.
    
    For the additional fees calculation, we add 6 months to the \textbf{due date}. However, the \textbf{due date} is not the same as ``the last possible day for payment,'' but rather just what came out of the calculation, i.e., \textsc{30 JUN 2018}. We add 6 months and we get 30 DEC 2018. However, since the 31-month period is long over, we are not calculating the \textsc{end of a period}, so \textit{de ultimo ad ultimo} applies also to the 6-month ``aggregate period,'' and we end up with 31 DEC 2018. The EPO is closed and the next day it is open is \textsc{2 JAN 2019}.
    
     \nt{Summary}{This is the case where \textbf{there is a priority}. \\This means the renewal date (\textsc{2 years + end of month}) will likely be the \textbf{later date}. \\
    Standard [\textbf{R. 134 EPC}] \underline{does NOT apply}, i.e., DO NOT move to the next working day if Office closed for both this calculation \underline{and} the 6-month ``aggregate period''. \\ The renewal due date is NOT the \textsc{end of a period}, so then we use 6-month extension from the \textit{due date} (regardless whether it's on a weekend) \textbf{with} \textit{de ultimo ad ultimo} \underline{and} [\textbf{R. 134}], if applicable, for the 6-month addition.}                
    
    \vspace{1em}
    
    \item % Question 5
    L4-30 (Basic selection) \\
    A Swedish applicant has filed an international application indicating the Swedish Patent and Registration Office as International Searching Authority. The application is filed in English. During the international search, the ISA is of the opinion that the international application contains 3 inventions, A, B and C, and that with respect to the second and third invention an additional fee has to be paid for each of them if the International Search Report is to cover these inventions. An additional fee is only paid for the second invention B. No demand was filed. The applicant wants to enter the regional phase before the EPO in May 2023, and let the EPO conduct searches on all three inventions A, B and C in the regional phase before deciding with which invention to continue into examination.
    
    
    Which amount of search fee(s) and/or further search fees must the applicant pay at or shortly after entry into the regional phase before the EPO (assuming the EPO shares the non-unity opinion)?
    \vspace{1em}
    
    \s \textbf{EPO does not trust any (S)ISA that wasn't them}. So, EPO will need to do a new search anyway, so the fee for a supplementary search under [\textbf{R. 164(1)}] is 1520 \textsc{EUR} --- see [\textbf{Rfees2(1)2}].
    
    \s Further, since ISA was \textsc{SE}, the search fee is actually reduced by \textsc{1300 EUR} because of an agreement --- this reduction is in \textbf{\textit{Summary of requirements for entry into the national phase}} (\textsc{BIO} tab after \textsc{Annex L} in PCT Applicant's Guide).
    
    \s \textbf{It's irrelevant that he's already paid for B in the international phase}. EPO will draw up a partial supplementary search report based on A only, so he will have to pay a further search fee for B and C, before deciding what invention to limit to for the examination --- \textsc{1520 EUR} each. 
    
    
    
    
    \item % Question 6
    L4-41 (Basic selection) \\
    Today, 25 February 2019, your Japanese client asks you to enter the regional phase before the EPO for a PCT-application filed in Japanese with the Japanese Patent Office on 26 July 2017, validly claiming priority from a Japanese patent application P-JP dated 1 August 2016. The International Preliminary Examination was carried out by the Japanese Patent Office. Due to an overload of work in the translation department of your client, you are informed that you will receive the English translation of the PCT-application on 23 April 2019.
    
    \vspace{1em}
    How would you proceed in order to enter the regional phase, keeping costs for your client to a minimum?

        \vspace{1em}
        
     \s Pay all other necessary fees before 31-month entry, i.e., search fee (because EPO wasn't IPEA); designation fee; filing fee; request for examination + examination fee; renewal fee if applicable (here not applicable).
     \s Do not file translation by EP entry. Will get it in time before 2-month extension deadline of withdrawal. It only costs \textsc{300 EUR} for further processing (FP), which should be cheaper than rushing an emergency translation before 31-month entry [\textbf{Rfees2(1)12}] --- \textsc{not super sure why it's 300 EUR --- ask Zsofia maybe???}. 
    \item % Question 7
    L4-42 (Advanced selection) \\
    An applicant wanted to enter the EP phase with his international application and duly filed a translation into English, and paid the filing fee and search fee. Shortly after the 31m time limit, the applicant received a loss-of-rights communication, indicating that the application was deemed to be withdrawn due to non-payment of the designation and examination fee and not filing the request for examination. One week after having received the communication, the applicant got into the hospital and despite all due care, he missed the time limit to request further processing with respect to the missed periods. 
    
    \vspace{1em}
    How many re-establishment fees does the applicant need to pay? 

    \vspace{1em}
    
    He will have to pay \textbf{2} re-establishment of rights (\texttt{RE}) fees, i.e:\\
    Designation fee -- 1 \\
    Examination fee -- 1\\
    \texttt{RE} fee is always a \underline{flat fee of \textsc{750 eur}}. 
    
\end{enumerate}
\section{October 27, 2025}


\begin{enumerate}[label=\textbf{Answer \arabic*}]

    \item % Question 1
    B4-02 \quad T/F question (Basic selection) \\
    Today is late February 2025. A Dutchman living in Sweden wishes to file a European patent application at the EPO in Munich via a Polish friend, who is a European patent attorney.
    
    \begin{enumerate}[label=(\alph*)]
        \item For each of the statements below, indicate whether the statement is true or false.
        \begin{enumerate}[label={(\alph{enumi}.\arabic*)}]
            \item The Dutchman is entitled to file the European patent application in Dutch. --- \textbf{True}.
            \item The Dutchman is entitled to file the European patent application in Swedish.  --- \textbf{True}.
            \item The Dutchman is entitled to file the European patent application in Polish.  --- \textbf{True}.
            \item The Dutchman is entitled to file the European patent application in German.  --- \textbf{True}.
        \end{enumerate}
        
        \item For each of the statements below, indicate whether the statement is true or false
        \begin{enumerate}[label={(\alph{enumi}.\arabic*)}]
            \item The Dutchman is entitled to a reduction of the filing fee under R.7a(1) when the European patent application is filed in Dutch, and a translation into English is filed at the same time. --- \textbf{True}.
            \item The Dutchman is entitled to a reduction of the filing fee under R.7a(1) when the European patent application is filed in Swedish, and a translation into English is filed at the same time. --- \textbf{True}.
            \item The Dutchman is entitled to a reduction of the filing fee under R.7a(1) when the European patent application is filed in Polish, and a translation into English is filed at the same time. --- \textbf{False}.
            \item The Dutchman is entitled to a reduction of the filing fee under R.7a(1) when the European patent application is filed in German, and a translation into English is filed at the same time. --- \textbf{False}.
            \item The reduction of the filing fee that the Dutchman is entitled to under R.7a(1) if the relevant requirements as to languages, translations and time limits are met is 20\% of the filing fee, including a 20\% reduction of any "page fees".  --- \textbf{False}.
            \item The reduction of the filing fee that the Dutchman is entitled to under R.7a(1) if the relevant requirements as to languages, translations and time limits are met is 30\% of the filing fee, including a 30\% reduction of any "page fees". --- \textbf{True}.
            \item The reduction of the filing fee that the Dutchman is entitled to under R.7a(1) if the relevant requirements as to languages, translations and time limits are met is 50\% of the filing fee, including a 50\% reduction of any “page fees". --- \textbf{False}.
            \item The Dutchman is (also) entitled to a reduction of the filing fee under R.7a(3) when the European patent application is filed in any language, and, where applicable, a translation into an official EPO language is filed at the same time. --- \textbf{True}.
        \end{enumerate}
        
        \item For each of the statements below, indicate whether the statement is true or false.
        \begin{enumerate}[label={(\alph{enumi}.\arabic*)}]
            \item The Dutchman is entitled to a reduction of the filing fee under R.7a(1) when the European patent application is filed in Dutch, and a translation into French is filed at the same time. --- \textbf{True}.
            \item The Dutchman is entitled to a reduction of the filing fee under R.7a(1) when the European patent application is filed in Dutch, and a translation into German is filed at the same time. --- \textbf{True}.
            \item The Dutchman is entitled to a reduction of the filing fee under R.7a(1) when the European patent application is filed in Dutch, and a translation into English is filed a week later. --- \textbf{True}.
            \item The Dutchman is entitled to a reduction of the filing fee under R.7a(1) when the European patent application is filed in English, and a translation into Dutch is filed a week later. --- \textbf{False}.
        \end{enumerate}
        
        \item \textit{Upon filing the EP application in Dutch by the Polish friend, the friend also pays all necessary fees.} \\
        For each of the statements below, indicate whether the statement is true or false.
        \begin{enumerate}[label={(\alph{enumi}.\arabic*)}]
            \item The EP application is validly filed when the EP application is filed in Dutch, and a translation into English is filed at own motion, without an invitation from the EPO, one week later.--- \textbf{True}.
            \item The EP application is validly filed when the EP application is filed in Dutch, and a translation into English is filed six weeks of filing the application. --- \textbf{True}.
            \item The EP application is validly filed when the EP application is filed in Dutch, and a translation into English is filed three months after of filing the application. --- \textbf{True} because EPO sends an invitation during formal examination and that will have a further 2-month time limit under \textbf{Rule 6(1) EPC} and \textbf{Rule 58 EPC}.
            \item If no translation is filed within 1 month of filing the EP application in Dutch, the EPO will issue an invitation to file a translation into an official EPO language within a time limit of two months. --- \textbf{False}; \textbf{Rule 6(1) EPC} says EPO will do so after 2 months, not 1 month.
        \end{enumerate}
    \end{enumerate}

    \item % Question 2
    B4-07
    \begin{enumerate}[label=(\alph*)]
        \item \underline{After filing} the European patent application, a Dutch person may file the following documents in Dutch: \textbf{Amendments which have to be filed within a time period}, e.g., response to \textbf{Art. 94(3)} communication, request for examination, notice of opposition, request for revocation/limitation. An example of a non-applicable ``further document'' would be third-party observations.
        \item A translation into \textsc{EN/DE/FR} must be filed under \textbf{R. 6(2)} within 1 month of filing the document [Or, where the document is a notice of opposition or appeal, or a statement of grounds of appeal, or a petition for review, the translation may be filed \textit{within the period for filing such a notice or statement or petition, if that period expires later.}] If amendments are being filed, they have to be filed in the language of proceedings (\textbf{R. 3(2) EPC)}. 
        \item The consequence of not filing a required translation is that the document is deemed to not have been filed at all (\textbf{Art. 14(4) EPC}). Further processing is available.    \end{enumerate}

    \item % Question 3
    B4-08 \quad T/F question (Basic selection) \\
    A European patent EP1 has been granted in English.
    
    \begin{enumerate}[label=(\alph*)]
        \item Mr. Jansen, a Dutch national living in the Netherlands, wants to file an opposition against European patent EP1. \\
        For each of the statements below, indicate whether the statement is true or false.
        \begin{enumerate}[label={(\alph{enumi}.\arabic*)}]
            \item Mr. Jansen may file the notice of opposition in Dutch. --- \textbf{True}
            \item Mr. Jansen may file the notice of opposition in English. --- \textbf{True}
            \item Mr. Jansen may file the notice of opposition in German. --- \textbf{True}
            \item Mr. Jansen may file the notice of opposition in Chinese. --- \textbf{False}
            \item Mr. Jansen may file the notice of opposition in Korean. --- \textbf{False}
        \end{enumerate}
        
        \item Mr. Xiao, a Chinese national living in the Netherlands, wants to file an opposition against European patent EP1. \\
        For each of the statements below, indicate whether the statement is true or false.
        \begin{enumerate}[label={(\alph{enumi}.\arabic*)}]
            \item Mr. Xiao may file the notice of opposition in Dutch. --- \textbf{True}
            \item Mr. Xiao may file the notice of opposition in English. --- \textbf{True}
            \item Mr. Xiao may file the notice of opposition in German. --- \textbf{True}
            \item Mr. Xiao may file the notice of opposition in Chinese. --- \textbf{False} -- even though he is Chinese, he is resident in \textsc{NL}, meaning the \textit{``language of \textbf{that state}''} referred to in \textbf{Art. 14(4)} refers to Dutch, not Chinese.
            \item Mr. Xiao may file the notice of opposition in Korean. --- \textbf{False}
        \end{enumerate}
        
        \item Mrs. Lee, a Korean national living in Korea, wants to file an opposition against European patent EP1 via Mr. Koch, her European patent attorney based in Munich. \\
        For each of the statements below, indicate whether the statement is true or false.
        \begin{enumerate}[label={(\alph{enumi}.\arabic*)}]
            \item Mrs. Lee may file the notice of opposition in Dutch through Mr. Koch. --- \textbf{False}
            \item Mrs. Lee may file the notice of opposition in English through Mr. Koch. --- \textbf{True}
            \item Mrs. Lee may file the notice of opposition in German through Mr. Koch. --- \textbf{True}
            \item Mrs. Lee may file the notice of opposition in Chinese through Mr. Koch. --- \textbf{False}
            \item Mrs. Lee may file the notice of opposition in Korean through Mr. Koch. --- \textbf{False}
        \end{enumerate}
    \end{enumerate}

    \item % Question 4
    B4-13 \quad T/F question (Advanced selection) \\
    A European patent application has been filed in Swedish by a Danish applicant living in Sweden. A translation of the patent application into English was duly filed. In reply to a communication from the Examining Division setting a period of 4 months to respond, the applicant files, on the last day of the period:
    \begin{itemize}
        \item a letter with his observations in Danish,
        \item in an Annex amendments to the claims in Swedish, and 
        \item in a further Annex as evidence a Japanese patent application in Japanese.
    \end{itemize}
    
    \begin{enumerate}[label=(\alph*)]
        \item For each of the statements below, indicate whether the statement with regard to the letter is true or false.
        \begin{enumerate}[label={(\alph{enumi}.\arabic*)}]
            \item The letter is filed in a non-admissible language, so the letter is deemed not filed. --- \textbf{False}
            \item A translation of the letter has to be filed within 1 month of filing the letter. The translation has to be into English. --- \textbf{False} --- \textbf{R. 3(1) EPC} allows the translation to be into any official EPO language, i.e., \textsc{EN/DE/FR}. 
            \item A translation of the letter has to be filed within 1 month of filing the letter. The translation has to be into any one of English, French or German. --- \textbf{True}.
            \item If the applicant fails to file a translation within the required time limit, the letter is deemed not filed. --- \textbf{True} --- it is \underline{deemed not filed} as \textit{per} \textbf{Art. 14(4) EPC}.
            \item If the applicant fails to file a translation of the letter nor of the claims within the required time limit, the application is deemed withdrawn. --- \textbf{True} --- failure to respond to \textit{any} communication from the ED on time leads to withdrawal -- \textbf{Art. 94(4) EPC}. 
        \end{enumerate}
        
        \item For each of the statements below, indicate whether the statement with regard to the Annex with amendments to the claims is true or false.
        \begin{enumerate}[label={(\alph{enumi}.\arabic*)}]
            \item The claims are filed in a non-admissible language, so the amended claims are deemed not filed. --- \textbf{False} --- the guy is an \textbf{Art. 14(4)} person, so he can use Swedish in this case and needs to file a translation into English.
            \item \colorbox{Thistle}{A translation of the} amended claims has to be filed within 1 month of filing the amended claims. The translation has to be into English. --- \textbf{True} --- \underline{Complicated issue}. \textbf{Art. 14(4) EPC} prevails here over \textbf{R. 3(2)}; i.e, the \textbf{amendments themselves} may be filed in Swedish, but the translation has to be into language of proceedings, i.e., \textsc{EN}.
            \item \colorbox{Thistle}{A translation of the amended claims} has to be filed within 1 month of filing the amended claims. The translation has to be into any one of English, French or German. --- \textbf{False} --- \textbf{R. 3(2)} specifies it's the language of proceedings, and the claims are part of the ``application'' proper, not just further documents relating to it.
            \item If the applicant fails to file a translation within the required time limit, the amended claims are deemed not filed. --- \textbf{True} --- the amended claims in response to a communication with a time period become an \textbf{Art. 14(4) EPC} document and they will be \textit{deemed to have never been filed}.
            \item If the applicant fails to file a translation of the claims nor of the letter within the required time limit, the application is deemed withdrawn. --- \textbf{True} --- simply \textbf{Art. 94(4) EPC}.
        \end{enumerate}
        
        \item For each of the statements below, indicate whether the statement with regard to the Annex with the Japanese patent application is true or false.
        \begin{enumerate}[label={(\alph{enumi}.\arabic*)}]
            \item The Japanese patent application is filed in a non-admissible language, so the Japanese patent application is deemed not filed. --- \textbf{False} --- this document is considered \textbf{evidence}, and provisions of \textbf{R. 3(3) EPC} apply to it; meaning it can be filed in any language.
            \item A translation of the Japanese patent application has to be filed within 1 month of filing the letter. The translation has to be into English.  --- \textbf{False} --- into any official EPO language \textit{if EPO requests it under \textbf{R. 3(3) EPC}}.
            \item A translation of the Japanese patent application has to be filed within 1 month of filing the letter. The translation has to be into any one of English, French or German. --- \textbf{False} --- into any official EPO language \textit{if EPO requests it under \textbf{R. 3(3) EPC}}.
            \item If the applicant fails to file a translation within the required time limit, the Japanese patent application is deemed not filed.  --- \textbf{False} --- the EPO \textit{may disregard it} under \textbf{R. 3(3) EPC}.
            \item If the applicant fails to file a translation within the required time limit, the application is deemed withdrawn. --- \textbf{False}.
        \end{enumerate}
    \end{enumerate}

    \item % Question 5
    B4-17 (Basic selection) \\
    An Italian applicant files two European patent applications, EP1, which claims the priority of Italian application IT1, and EP2, which claims the priority of Italian application IT2. 
    EP1 is filed in Italian and is identical to IT1. An English translation of EP1 is supplied to the EPO two weeks later.
    EP2 is filed in English. It is the translation of IT2. 
    During the examination procedure, the English texts of EP1 and EP2 are found to contain translation errors.
    
    \begin{enumerate}[label=(\alph*)]
        \item Would it be possible to correct these errors for EP1? --- \textbf{Yes}, according to \textbf{Art. 14(2) EPC}, the translation of the text in an official EPO language may be brought into conformity with the EP application as filed in the original language. In this case, EP1 was filed in Italian, so the translation can be corrected at any time. 
        \item Would it be possible to correct these errors for EP2? --- \textbf{No}, as \textbf{Art. 14(2) EPC} -- and \textbf{R. 139 EPC} more generally -- only pertain to documents filed with the EPO. As such, the priority documents cannot be used for the purposes of correcting a translation error (see also case law of \textsc{G 3/89} and \textsc{G 11/91} and \textsc{GL H--VI, 2.2.1}). A possible remedy might be to re-file the application if the priority year still hasn't passed. 
    \end{enumerate}

    \item % Question 6
    B5-08 \quad T/F question (Basic selection) \\
    A Dutch national lives in the USA and wishes to file a European patent application.
   
    \begin{enumerate}[label=(\alph*)]
        \item For each of the statements below, indicate whether the statement is true or false.
        \begin{enumerate}[label={(\alph{enumi}.\arabic*)}]
            \item He does not need to appoint a professional representative for filing a European patent application in Dutch at the EPO, but he can do so. \T  --- \textbf{Art. 133(2) EPC}.
            \item He does not need to appoint a professional representative for filing a European patent application in English at the EPO, but he can do so. \T  --- \textbf{Art. 133(2) EPC}.
        \end{enumerate}
        
        \item For each of the statements below, indicate whether the statement is true or false.
        \begin{enumerate}[label={(\alph{enumi}.\arabic*)}]
            \item He does not need to appoint a professional representative for paying the filing fee and search fee to the EPO, but he can do so. \T
            \item He does not need to appoint a professional representative for paying the filing fee and search fee to the EPO if he pays the fees simultaneously with filing the application, but he shall appoint a professional representative when the fees are paid later. \F
            \item He is obliged to let a professional representative pay the filing fee and search fee to the EPO. \F
        \end{enumerate}
        
        \item \textit{The Dutch national living in the USA files a European patent application at the EPO in Dutch.} \\
        For each of the statements below, indicate whether the statement is true or false.
        \begin{enumerate}[label={(\alph{enumi}.\arabic*)}]
            \item He does not need to appoint a professional representative when he files a translation in English 3 weeks after the filing date, but he can do so. \F --- for \textbf{anything} \underline{after} the filing, he needs professional representation.
            \item He does not need to appoint a professional representative when he files a translation in English simultaneously with filing the application in Dutch. \T -- it was done \underline{at the same time} as filing so he can do it.
        \end{enumerate}
        
        \item \textit{The Dutch national files only a fully completed request form, but does not file a description nor a reference to a previously filed application, nor any claims.} \\
        For the statement below, indicate whether the statement is true or false.
        \begin{enumerate}[label={(\alph{enumi}.\arabic*)}]
            \item He does not need to appoint a professional representative when he files a description 5 days after having filed the request form. \T
        \end{enumerate}
        
        \item \textit{The Dutch national files only a fully completed request form as well as a description but no claims.} \\
        For the statement below, indicate whether the statement is true or false.
        \begin{enumerate}[label={(\alph{enumi}.\arabic*)}]
            \item He does not need to appoint a professional representative when he files a set of claims 5 days after having filed the request form. \F
        \end{enumerate}
        
        \nt{Prof. rep. at filing}{If something is done at the step of filing (before having a filing date), no representation is needed. However, for a US resident, a representative must be appointed for all the steps (other than payment of the fees) after a filing date has been accorded.}
    \end{enumerate}

    \item % Question 7
    B5-09 (Basic selection) \\
    Who is deemed to be the common representative for a European patent application with three applicants, where no common representative has been appointed, in the following situations?
    
    \begin{enumerate}[label=(\alph*)]
        \item All applicants are from a Contracting State and none has appointed a professional representative. --- the first named Applicant will be the deemed common representative under \textbf{R. 151(1) EPC}. 
        \item All applicants are from a Contracting State and applicant 3 has appointed a professional representative. --- the first named Applicant will be the deemed common representative under \textbf{R. 151(1) EPC}. 
        \item Applicant 1 and 3 are from a Contracting State; applicant 2 is a US company; no professional representative has been appointed. --- Applicants 1 and 3 are not obliged to have a professional representative. Applicant 2 is obliged to have a professional representative under \textbf{Art. 133(2) EPC}, but he hasn't named one yet. Once he names one, his prof. rep. will become the common rep. 
        \item Applicant 1 and 3 are from a Contracting State; applicant 2 is a US company; applicant 1 has appointed a professional representative. -- in this case, Applicant 1 (from CS) has an appointed professional representative, so they override anything from Applicant 2 from non-CS state [\textbf{R. 151(1) EPC}].
        \item Applicant 1 and 3 are from a Contracting State; applicant 2 is a US company; applicant 3 has appointed a professional representative. --- Applicants 1 and 3 are not obliged to have a professional representative. Applicant 2 is obliged to have a professional representative under \textbf{Art. 133(2) EPC}. Even though Applicant 3 has named one, they weren't \underline{\textit{obliged}} to do so. So professional rep. of Applicant 2 \underline{must} be appointed and will be the common rep. 
    \end{enumerate}

\end{enumerate}
\section{November 03, 2025}

\begin{enumerate}[label=\textbf{Answer \arabic*}]

    \item % Question 1
    C7-11 \qquad T/F question (Basic selection) \\
    A European application EP1 claims priority from a first, earlier patent application FR1 disclosing feature C and a second, earlier patent application DE1 disclosing feature D. None of the earlier applications discloses the combination of features C and D. FR1 does not disclose feature D in isolation either. \\
    The filing dates of FR1, DE1 and EP1 are 12 January 2022, 12 February 2022 and 18 December 2022, respectively.
    
    \vspace{1em}
    Claim 1 of the European patent application EP1 defines the combination of features C and D. \\
    Claim 2 of the European patent application EP1 defines individual feature D.
    
    \vspace{1em}
    For each of the statements below, indicate whether the statement is true or false.
    
    \begin{enumerate}[label=(\alph*)]
        \item The effective date of claim 1 of EP1 is the filing date of FR1. \F
        \item The effective date of claim 1 of EP1 is the filing date of DE1. \F
        \item The effective date of claim 1 of EP1 is the filing date of EP1. \T
        \item The effective date of claim 2 of EP1 is the filing date of EP1. \F
    \end{enumerate}
    
    \vspace{1em}
    A further European application EP2 claims priority from FR1, DE1 and EP1.
    
    \vspace{0.5em}
    Claim 1 of the European patent application EP2 defines the combination of features C and D. \\
    Claim 2 of the European patent application EP2 defines individual feature D.
    
    \vspace{0.5em}
    The filing date of EP2 is 11 January 2023.
    
    \vspace{1em}
    For each of the statements below, indicate whether the statement is true or false.
    \begin{enumerate}[label=(\alph*), resume]
        \item The effective date of claim 1 of EP2 is the filing date of DE1. \F
        \item The effective date of claim 1 of EP2 is the filing date of EP1. \T
        \item The effective date of claim 2 of EP2 is the filing date of DE1. \T
        \item The effective date of claim 2 of EP2 is the filing date of EP1. \F
    \end{enumerate}

    \vspace{1em}
    A further European application EP3 claims priority from FR1, DE1, EP1 and EP2.
    
    \vspace{0.5em}
    Claim 1 of the European patent application EP3 defines the combination of features C and D. \\
    Claim 2 of the European patent application EP3 defines individual feature D.
    
    \vspace{0.5em}
    The filing date of EP3 is 17 December 2023.
    
    \vspace{1em}
    For each of the statements below, indicate whether the statement is true or false.
    \begin{enumerate}[label=(\alph*), resume]
        \item The effective date of claim 1 of EP3 is the filing date of EP1. \T
        \item The effective date of claim 1 of EP3 is the filing date of EP3. \F
        \item The effective date of claim 2 of EP3 is the filing date of DE1. \F
        \item The effective date of claim 2 of EP3 is the filing date of EP3. \T --- the \textit{first application} disclosing subject-matter \textsc{D} was \textsc{DE1}, and it is too late to claim it now (cannot do it off \textsc{EP1} or \textsc{EP2} as they were \textbf{not the first application} for the same invention/subject-matter. 
    \end{enumerate}

    \item % Question 2
    C7-13 (Basic Selection) \\
    An applicant filed a first patent application SE1 describing and claiming a mosquito repellent comprising a combination of compounds C and K, which causes the mosquito repellent to very efficiently keep mosquitos at a distance. After further experimentation, the applicant found that the mosquito repellent is also very efficient in keep mosquitos at a distance if instead of K, a compound L is used. The applicant compiled a new European application EP2 by taking the full description of SE1, adding the newly found mosquito repellent comprising a combination of compounds and L to the description, and claiming a mosquito repellent comprising a combination of compounds C and at least one of K and L. The applicant duly filed the new European application EP2 at the end of the priority period, while claiming priority from SE1. \\
    A few months after filing SE1, the applicant handed out samples of his mosquito repellent comprising a combination of compounds C and K at an outdoor event in Sweden. \\
    Is the hand-out of the samples at the outdoor event in Sweden novelty-destroying for the claim of EP2?


\p No, the outdoor event in Sweden is not novelty destroying for the claim 1 of \textsc{EP2}.  \\
\p \textsc{EP2} discloses combinations CK, CL and CKL. It can only claim prio. from \textsc{SE1} on CK. \\
\p Outdoor event only showed CK, so CK is sitll novel. \\
\p Outdoor event is prior art for CKL and CL, but since CK is novel, the whole claim 1 of \textsc{EP2} is novel.

    \item % Question 3
    C7-14 (Basic Selection) \\
    An applicant filed a first patent application NL1 claiming a new and inventive process P comprising a heating step at a temperature of 100-150°C, and describing process P with experimental data at 110°C and 140°C. After further experimentation, the applicant found that the process also gives very satisfactory results at temperatures down to 70°C and up to 250°C. The applicant compiled a new European application EP2 by taking the description of NL1, adding experimental data at 70°C and 250°C to the description, and replacing the range 100-150°C in the claim by 70-250°C. The applicant duly filed the new European application EP2 at the end of the priority period, while claiming priority from NL1. \\
    A few months after filing NL1, the applicant published details about his new and inventive process P at a temperature of 120°C at a scientific conference. \\
    Is the publication at the conference novelty-destroying for the claim of EP2?

\p No, the conference is not novelty destroying for the claim 1 of \textsc{EP2}.  \\
\p \textsc{EP2} can claim prio. for range 100--150. It then adds an "\textsc{or}" to its claim by including also ranges 70--100 and 150--250. \\
\p The conference spoils range 100--150 by disclosing 120. But this was secured with prio. claim. The conference doesn't spoil either the range 70--100 or 150--250. \\
\p So since neither of the 3 ranges is spoiled, the whole claim of \textsc{EP2} is novel. 

    \item % Question 4
    C7-18 (Basic selection) \\
    A French Company X filed a French patent application A-FR in July 2023. Inventor I1 employed by Company X and inventor I2 employed by the Spanish Company Y were named as inventors. \\
    Today, 21 March 2024, a European patent application EP was filed by Company Y in its name only, claiming the priority of A-FR in accordance with Art.88(1), R.52 and 53 EPC, and designating inventor I2 as the sole inventor.
    \begin{enumerate}[label=(\alph*)]
        \item The priority situation today is that the EPO will work off the \textit{rebuttable presumption} that there was an agreement between the parties and that, in this case, the French company X transferred rights to the Spanish company Y, and Y can validly claim priority (we are in the prio. year) because it is the presumed "same" Applicant (see also \textsc{G 1/22} and \textsc{G 2/22}). 
        \item If EP had been filed by Companies X and Y as joint applicants, then it is fine anyway, becase \textbf{you can always add Applicants; but you can't remove them} without "transfer" of rights being mentioned in the question. If there is no "transfer," just assume it was legit! \\
\textsc{Inventors are irrelevant for priority!}
    \end{enumerate}

    \item % Question 5
    C7-21 - T/F question (Basic selection) \\
    A German patent application DE1 was filed in a hurry by a German applicant. Two months later the applicant filed a better-compiled German patent application DE2 for the same subject-matter. DE1 and the priority right that originated from its filing were duly withdrawn "without leaving any rights outstanding" before DE2 is filed. Six months later a European patent application EP1 is filed claiming priority of DE2. Two months later the same applicant filed a second European patent application EP2 claiming priority of DE1.
    
    \vspace{1em}
    For each of the statements below, indicate whether the statement is true or false.
    \begin{enumerate}[label=(\alph*)]
        \item EP1 validly claims priority from DE2. \T --- \textbf{Art. 87(4) EPC} exception applies.
        \item EP2 validly claims priority from DE1. \F --- \textbf{Art. 87(4) EPC} exception does not apply (last sentence rules out using DE1 ever again). \\
\p Nonetheless, the declaration of priority may be corrected within 16 months of the earliest priority date in question under \textbf{R. 52(3) EPC}. In this case, only 8 months have passed since DE1, so he can still use it to claim prio. from the withdrawn DE1.
    \end{enumerate}

    \item % Question 6
    C7-24 \qquad T/F question \\
    A European patent application EP1 is filed on 29 July 2017 indicating a priority date of 29 January 2017. Shortly after the filing of EP1, the applicant recognized that the priority date that was given is incorrect. The applicant wants to correct the priority date to 27 September 2016.
    
    \vspace{1em}
    For each of the statements below, indicate whether the statement is true or false.
    
    \begin{enumerate}[label=(\alph*)]
        \item The declaration of priority shall be made on filing and cannot be corrected. \F --- \textbf{R. 52(3) EPC} explicitly provides for correcting a priority date within \textbf{\textit{the later of}}: \textbf{(i)} 16 months of the \underline{earliest} of claimed priority dates; or \textbf{(ii)} 4 months from filing date of EP application.
        \item The declaration of priority may be corrected until 29 November 2017. \F --- this is the 4 months from EP filing, but it's the earlier of (i) or (ii).
        \item The declaration of priority may be corrected until 29 January 2018. \T --- this is 16 months from the earliest mentioned priority date claim.
        \item The declaration of priority may be corrected until 29 May 2018. \F
    \end{enumerate}

    \item % Question 7
    C7-29 \\
    On 14 December 2022 you filed a European patent application EP2 claiming priority from a European patent application EP1 filed on 17 August 2022. Today, 7 March 2023, you received a telephone call from your client who instructs you also to claim priority from a German utility model filed on 19 December 2021.
    \begin{enumerate}[label=(\alph*)]
        \item Is this possible? If so, what is the last date on which the applicant can add the priority claim? \T --- this is possible, because 16 months from the \textsc{DE} utility model runs out on 19.04.2023, so we still have over a month to add a priority claim to it. The German utility model falls within the priority year of \textsc{EP2}. 
    \end{enumerate}
    
    \vspace{1em}
    Your client further informs you that the certified copy of the priority document of the German utility model will not be issued by the German Patent and Trademark Office before 10 May 2023.
    \begin{enumerate}[label=(\alph*), resume]
        \item What can you undertake? --- 10 May 2023 is after the 16-month deadline. However, under \textbf{R. 59 EPC -- \textit{Deficiencies in claiming priority}}, the EPO will send an invitation to provide the certified copy of the prio. document within \textit{a period to be specificed}. This period cannot be shorter than \textbf{2m} or longer than \textbf{4m} under \textbf{R. 132(2) EPC}. This will cover the period in question here. Even if it didn't, under \textbf{R. 132(2); second sentence}, this period to be specified may be extended upon request. 
    \end{enumerate}

\end{enumerate}
\section{November 10, 2025}

\begin{enumerate}[label=\textbf{Question \arabic*}]

    \item % Question 1
    H2-01 - T/F question \\
    A European patent application is filed today, 25 February 2019, in English.
    
    \begin{enumerate}[label=(\alph*)]
        \item For each of the statements below, indicate whether the statement is true or false.
        \begin{enumerate}[label={(\alph{enumi}.\arabic*)}]
            \item The filing and search fee fall due today, 25 February 2019. Failure can be remedied. \T
            \item The filing and search fee must be paid today, 25 February 2019. Failure can be remedied. \F --- they must be paid within \textbf{1 month} of filing [\textbf{A. 78(2) EPC}].
            \item The filing and search fee must be paid at the latest on 25 March 2019. Failure can be remedied. \T --- \textbf{FP} is available (see list of excluded \textbf{FP} stuff under \textbf{R. 135(2) EPC}).
            \item No time limit for payment of the filing and search fee is set yet; the EPO will send a communication with an invitation to pay within a time limit set in the communication. \F --- under \textbf{R. 38(1) EPC} it clearly says \textbf{1 month}.
        \end{enumerate}
        
        \item \textit{The filing and search fee are not paid in time.} \\
        For each of the statements below, indicate whether the statement is true or false.
        \begin{enumerate}[label={(\alph{enumi}.\arabic*)}]
            \item In case the filing and search fee are not paid in time, the EPO sends a communication concerning the failure to observe the time limit and the fees may still be paid without additional fee within one month of the communication. \F
            \item In case the filing and search fee are not paid in time, the EPO sends a communication concerning the failure to observe the time limit. FP can be performed by completing the omitted act, i.e. pay the filing and search fee, within two months from the communication, and payment of the flat FP fee of RFees 2(1).12, last dash. \F --- ``last dash'' refers to \textit{other cases}, but here we have 50\% of the missed fee; not a flat FP fee.
            \item In case the filing and search fee are not paid in time, the EPO sends a communication concerning the failure to observe the time limit. RE in respect of the one month time limit for payment of the filing and search fee can be applied if all due care can be proven. \T --- see \textit{\textbf{Lobato}; page 203}. \colorbox{Thistle}{--- \textbf{Ask Zsofia}}
            \item In case the filing and search fee are not paid in time, the applicant may of his own motion remedy this failure by payment of the filing and search fee, as well as an additional 50\% of both fees on the next day. \T --- see \textit{\textbf{Visser}} commentary under \textbf{R. 135(1)}.
        \end{enumerate}
        
        \item \textit{The application as filed today comprised 40 pages, but no claims. The applicant will file a set of claims after having received an invitation to do so. Four pages comprising 16 claims in total will be filed with online filing on 14 April 2019. The applicant does not want to pay any penalty fees.} \\
        For each of the statements below, indicate whether the statement is true or false.
        \begin{enumerate}[label={(\alph{enumi}.\arabic*)}]
            \item As the application as filed comprises 40 pages, an additional fee shall be paid for 5 pages (RFees 2(1).1a) and at the latest on 25/3/2019. \F --- it needs to be paid within \textbf{1 month} of filing but it is missing claims and will fail \textbf{R. 57(c) EPC}, so likely EPO will invite under \textbf{R. 58 EPC} and give \textbf{2 months} to pay from the date of invitation.
            \item An additional fee shall be paid for 9 pages (Fees 2(1).1a) and at the latest on 25/3/2019. \F --- the claims are filed later than this date, so they couldn't have known that. 
            \item An additional fee shall be paid for 9 pages (Fees 2(1).1a) and at the latest on 14/5/2019.        \T --- it needs to be paid within \textbf{1 month} of filing or first claim set or reference to another app under \textbf{R. 38(3) EPC}. In this case, it's from the claim set of 14 \textsc{APR} 2019.
            \item Claims fees shall be paid in respect of one claim (RFees 2(1), 15) and at the latest on 14/5/2019 or, without a surcharge, within one month from a communication concerning the failure to meet this time limit.  \T --- \textbf{R. 45(2) EPC}.
        \end{enumerate}
    \end{enumerate}

    \item % Question 2
    H2-02 - T/F question (Basic selection) \\
    On filing, a European patent application comprises thirteen claims. In response to the search opinion, the applicant submits a set of amended claims, whereby the number of claims has become seventeen, and requests examination on the new set of seventeen claims.
    
    \vspace{1em}
    For each of the statements below, indicate whether the statement is true or false.
    
    \begin{enumerate}[label=(\alph*)]
        \item Claims fees are to be paid in respect of 2 claims within one month of filing the 17 claims. \F
        \item Claims fees are to be paid in respect of 2 claims within one month of filing the 17 claims and, if the patent is granted with at most 15 claims, any claims fees paid will be refunded. \F
        \item No claims fees are to be paid yet. However, claims fees will become payable when the EPO intends to grant the application with the 17 amended claims. \T --- \textbf{R. 71(4) EPC}.
        \item No claims fees are to be paid yet. If the number of claims is reduced to 15 or less before a R.71(3) communication is issued and not increased anymore afterward, no claims fee need to be paid.           \T \\
        Relevant section in \textbf{\textit{Lobato}} here is \textsc{page 250}.
    \end{enumerate}

    \item % Question 3
    H2-05 - T/F question (Basic selection) \\
    A European patent application was filed on-line on 18 April 2017 claiming the priority of an earlier British application filed on 14 April 2016. The European patent application comprises only a description, drawings and an abstract, together 33 pages. On 6 May 2017 the EPO received a document with an annex of 3 pages containing eighteen claims.
    
    \vspace{1em}
    For each of the statements below, indicate whether the statement is true or false, if today is May 2017.
    
    \begin{enumerate}[label=(\alph*)]
        \item The filing date accorded to the European patent application is 18/4/2017. \T
        \item The filing fee of Fees 2(1).1 may be validly paid until 18/5/2017. An additional fee under RFees 2(1).1a must be paid within the same period. \F --- additional fee can be paid \textbf{1m} after that. 
        \item The filing fee of Fees 2(1).1 may be validly paid until 6/6/2017. Any additional fee under Fees 2(1). 1a must be paid within the same period. \F --- the filing fee must be paid by 18/5/2017.
        \item Failure to pay the filing and search fee will result in the application being deemed not filed. \F
        \item Failure to pay the filing and search fee will result in the application being deemed withdrawn. \T
        \item Failure to pay the additional fee of Fees 2(1).1a will result in the application being deemed withdrawn. \T
        \item Failure to pay the additional fee of Fees 2(1).1a will result in a communication with an invitation to pay the additional fee within one month of the communication. \F --- no invitation for the additional fee.
        \item Failure to pay the claims fees will result in the application being deemed withdrawn. \F
        \item Failure to pay the claims fees will result in a communication with an invitation to pay the claims fees within one month of the communication. \T
        \item Failure to pay the claims fees will result in a communication with an invitation to pay the claims fees within one month of the communication with an additional fee of 50\% of the claims fees. \F
        \item Failure to pay the claims fees in response to the communication with the invitation to pay the claims fees may be remedied by payment of the claims fees and an additional 50\% of the claims fees in response of a loss-of-rights communication of the EPO. \T
    \end{enumerate}
    
    \vspace{1em}
    The A2 publication was on Wednesday 25 October 2017 and the A3 publication on Wednesday 22 November 2017.
    
    \vspace{0.5em}
    For each of the statements below, indicate whether the statement is true or false.
    \begin{enumerate}[label=(\alph*), resume]
        \item The examination fee had to be paid at the latest on 22/5/2018. \T
        \item The designation fee had to be paid at the latest on 25/4/2018. \F
    \end{enumerate}
    
    \vspace{1em}
    Today is early April 2018. The applicant asks you what the effect is if the examination and/or designation fee are not timely paid, and whether further processing would be available.
    
    \vspace{0.5em}
    For each of the statements below, indicate whether the statement is true or false.
    \begin{enumerate}[label=(\alph*), resume]
        \item Non-payment of the examination fee will result in the application being deemed withdrawn with further processing available as remedy. \T
        \item Non-payment of the designation fee will result in the application being deemed withdrawn with further processing available as remedy. \T
    \end{enumerate}

    \item % Question 4
    H3-07 (Basic selection) \\
    Which date will be accorded as the filing date if an applicant:
    \begin{enumerate}[label=(\alph*)]
        \item sends his European patent application to the EPO by registered mail, date stamped at his local post office at Thursday 8 September 2022 and delivered to the EPO in The Hague on Friday 9 September 2022; --- \textbf{9 \textbf{SEP} 2022} --- date of \underline{actual receipt}.
        \item sends European patent application to the EPO by registered mail, date stamped at his local post office at Saturday 10 September 2022 and delivered to the EPO in The Hague on Monday 12 September 2022; --- \textbf{12 \textbf{SEP} 2022}
        \item submits his European patent application to the EPO by EPO Online Filing 2.0 on Saturday 10 September 2022; --- \textbf{10 \textbf{SEP} 2022}
        \item hands his European patent application to the reception at the EPO in The Hague on Saturday 10 September 2022; --- \textbf{10 \textbf{SEP} 2022}
        \item puts his European patent application in the automated mailbox at the EPO in Berlin on Saturday 10 September 2022; --- \textbf{10 \textbf{SEP} 2022}
    \end{enumerate}
    where the applicant files a description, claims, drawings and a fully completed Request for Grant form.

    \item % Question 5
    H3-13 - T/F question (Basic selection) \\
    A Dutch applicant wishes to file a new European patent application EP2 directly with the EPO in The Hague on 14 March 2023 by reference to the claims, description and drawings of an earlier European patent application EP1 filed on 1 May 2017 with the EPO in The Hague. No filing and search fee were paid for the earlier European patent application EP1.
    
    \begin{enumerate}[label=(\alph*)]
        \item For each of the statements below, indicate whether the statement is true or false.
        \begin{enumerate}[label={(\alph{enumi}.\arabic*)}]
            \item This is possible by filing a Request-for-Grant form with all necessary information in the relevant boxes filled in. \T
            \item This is not allowed, because the priority period has already expired. \F
        \end{enumerate}
        
        \item \textit{The box for indicating the file number of the earlier European application in the Request-for-Grant form is left empty.} \\
        For each of the statements below, indicate whether the statement is true or false.
        \begin{enumerate}[label={(\alph{enumi}.\arabic*)}]
            \item The filing date accorded to the new European patent application will be 14 March 2023 if the applicant provides the file number within a period of 2 months from an invitation of the EPO to do so. \F ---  the filing date will change to the date the file number is provided (it is a minimum requirement for the filing date) --- \textbf{R. 40(2) EPC} and \textit{page 179} of \textbf{Lobato}.
            \item The filing date accorded to the new European patent application will be 14 March 2023, but the application will be deemed withdrawn if the applicant fails to provide the file number within a period of 2 months from an invitation of the EPO to do so, but remedies this failure with further processing. \F -- it won't be withdrawn; it won't even be treated as an \textsc{ep} application.
        \end{enumerate}
    \end{enumerate}

    \item % Question 6
    H3-16 - Multiple choice \\
    On the last day of the priority period, you filed an EP application, claiming priority of an earlier British application. The EP application was identical to the British application. \\
    The next day, when preparing the filing receipt and a copy of the filed documents to your client, you find out that you accidentally forgot to file the claims. The claims are not literally cited in the description, and are somewhat broadened relative to the specifically disclosed embodiments.
    
    \vspace{1em}
    What can you do to recover the claims in their original wording?
    
    \begin{enumerate}[label=\Alph*.]
        \item File the claims as missing parts, which allows to maintain the filing date because the claims were completely contained in the priority document. \F --- you can \textbf{never} file claims as missing parts.
        \item File the claims in their original form within 2 months from an invitation under Rule 55 EPC. \F --- date of filing under \textbf{R. 40 EPC} is fine because you don't need claims for that.
        \item File the claims in their original form within 2 months from an invitation under Rule 58 EPC. \T -- EPO will issue invitation under \textbf{R. 58 EPC} to ``correct deficiencies,'' i.e., add at least one claim. But you will likely get an \textbf{Art. 123(2) EPC} strike. \\\colorbox{Thistle}{maybe ask Zsofia about how to handle this type of leading question.}
        \item None of A-C. \F
    \end{enumerate}

    \item % Question 7
    H3-22 (Basic selection) \\
    European patent application EP 1 describes inventions A, B, C and D, but only claims inventions A, B and C. EP1 is still pending. A divisional application DIV1 was timely divided out of EP1, with the same description as EP1, but claiming only invention D. After quite some debate with the Examining Division, DIV1 was refused in oral proceedings last week.
    
    \vspace{1em}
    DIV2 was timely divided out of EP1, describing and claiming inventions B and C. In response to a non-unity objection in the EESR issued in February 2020, DIV2 was limited to only invention B before examination. The applicant just received a communication under Rule 71(3) EPC in respect of DIV2.
    
    \vspace{1em}
    Today is 17 March 2023.
    
    \begin{enumerate}[label=(\alph*)]
        \item Until when may another divisional DIV3 be filed out of EP1, directed to C? --- you can still do a DIV3 off of EP1 as long as EP1 is pending.
        \item Until when may another divisional DIV4 be filed out of DIV1, directed to C? --- you have 2m of appeal window to file a divisional as even a refused app is considered pending until then. You may even get 4m if appeal filed and appeal fee paid.
        \item Until when may another divisional DIV5 be filed out of DIV2, directed to C? --- until day before mention of grant of DIV2.
    \end{enumerate}

    \item % Question 8
    H3-30 - T/F question \\
    A European patent application EP1 is filed. The filing and search fee are paid in time. The applicant has received an EESR, dated 2 February 2018, indicating that the claims appear to be novel and inventive. The mention of the publication of the search report is published in the EP Bulletin on 23 May 2018. The applicant however is no longer interested in the invention as claimed and therefore decides to let the application lapse by not paying the examination and designation fee. The EPO sends a communication under Rule 112(1) EPC noting a loss of rights, dated 13 Dec 2018. The applicant files a European divisional application DIV1 on 28 December 2018 with the same description as EP1 but with claims defining another invention. The claims of the divisional are supported by the description. The applicant also pays the filing and search fee upon filing the divisional.
    
    \vspace{1em}
    For each of the statements below, indicate whether the statement is true or false.
    
    \begin{enumerate}[label=(\alph*)]
        \item The loss of rights communication indicates that the application is refused. \F --- the app. will be \textit{deemed to be withdrawn} after not responding to communication.
        \item The loss of rights ensues at the end of 23 November 2018 \T
        \item The loss of rights ensues at the end of 3 December 2018. \F
        \item The loss of rights ensues at the end of 13 December 2018. \F
        \item The loss of rights ensues at the end of 23 December 2018. \F
        \item The loss of rights ensues at the end of 24 February 2019. \F
        \item As the loss of rights may still be remedied with further processing and the term for requesting further processing has not yet expired, the parent application was still pending when the divisional was filed. No further processing of any kind is needed. \F --- parent was not pending upon being deemed to be withdrawn. The FP remedy should have been applied \textit{before} filing the divisional.
        \item As the parent was no longer pending when DIV1 was filed, DIV1 was filed too late. However, this failure may be remedied by requesting further processing in respect of the missed term for filing the divisional. \F --- parent was not pending upon being deemed to be withdrawn. The FP remedy should have been applied \textit{before} filing the divisional. You cannot do anything in terms of the divisional, because you cannot validly file it. 
        \item As the parent was no longer pending when DIV1 was filed, DIV1 was filed too late. However, this failure may be remedied by requesting further processing in respect of the missed terms for paying the examination and designation fee for the parent application. \T --- The FP remedy may be applied to restore the parent and then you can file a new divisional based off that restored parent. 
        \begin{center}
         
\p \p \p

        \end{center} 
        Key consideration here is that the \textbf{loss of rights ensues \underline{directly on expiry of applicable period}}. So even though the communication was sent later, the expiry was actually on 23 \textsc{Nov} 2018 (6 months from publication of EESR) and that's when loss of rights ensued. \\
        Further, the only way of being able to file a divisional is to restore the pendency of the parent first by performing FP requirements. 
    \end{enumerate}
        \end{enumerate}
    
    \end{enumerate}
  
  
\section{November 17, 2025}

 \begin{enumerate}[label=\textbf{Question \arabic*}]

    \item % Question 1
    H4-09 (Basic selection) \\
    An applicant has filed a second European patent application EP2, identical to and claiming priority of an earlier European patent application EP1. The Extended European Search report of the earlier European patent application EP1 indicates that the claims are novel and inventive and a communication under Rule 71(3) EPC could be expected. EP1 has been withdrawn.
    
    \vspace{1em}
    The first steps for the applicant to come to a rapid grant of \textsc{EP2} would be to indicate upon filing of \textsc{EP2} that the claims of \textsc{EP1} are taken in full for \textsc{EP2}. EPO already did the search report on this invention, so it is available to them and the search fee for \textsc{EP2} will be refunded in full. No PACE request is needed. The Examination Fee and Designation Fee should be paid ASAP. Communication under \textbf{R. 70(2) EPC} (invitation to indicate whether you wanna proceed with the application) should be \underline{\textsc{waived}} by ticking the appropriate box on \textbf{RfG}. Then after \textbf{R. 71(3) EPC} comm. for \textsc{EP2}, file the translation of claims and pay the fee for publishing and grant. 

    \item % Question 2
    H4-12 \qquad T/F question (Basic selection) \\
    The Search Division informs the applicant in accordance with Rule 64(1) EPC that the application contains three inventions, and that with respect to the second and third invention a further search fee has to be paid for each of them if the European search report is to cover these inventions. A further search fee is only paid for the second invention. In the first communication C1 from the Examining Division, the objection of non-unity is maintained in respect of the three inventions.
    
    \vspace{1em}
    For each of the statements below, indicate whether the statement is true or false.
    
    \begin{enumerate}[label=(\alph*)]
        \item The applicant may obtain protection for invention one by limiting the application to invention one and cancelling the claims to inventions two and three accordingly. \T
        \item The applicant may obtain protection for invention two by limiting the application to invention two and cancelling the claims to inventions one and three accordingly. \T
        \item The applicant may obtain protection for invention three by limiting the claims of the application to invention three and cancelling the claims to inventions one and two accordingly. \F
        \item The applicant may obtain protection for invention one and two with this single application by limiting the application to invention one and two and cancelling the claims to invention three accordingly. \F
        \item The applicant may obtain protection for invention one by timely filing a divisional application with only the claims directed to invention one. \T
        \item The applicant may obtain protection for invention two by timely filing a divisional application with only the claims directed to invention two. \T
        \item The applicant may obtain protection for invention three by timely filing a divisional application with only the claims directed to invention three. \T
    \end{enumerate}
    
    \item % Question 3
    H5-06 \qquad T/F question (Basic selection) \\
    European patent application EP1 was filed on 17 April 2021 and published on Wednesday 19 October 2022. European patent application EP2 was filed on 6 June 2022. European patent application EP3 was filed on 2 December 2022. EP1, EP2 and EP3 do not claim priority. \\
    EP1 was withdrawn by the applicant before publication but too late to avoid publication.
    
    \vspace{1em}
    For each of the statements below, indicate whether the statement is true or false.
    
    \begin{enumerate}[label=(\alph*)]
        \item EP1 is prior art under Art. 54(2) EPC and may as such be used in assessing novelty of EP2. \F -- it's a 54(3) doc.
        \item EP1 is prior art under Art. 54(2) EPC and may as such be used in assessing inventive step of EP2. \F
        \item EP1 is prior art under Art. 54(3) EPC and may as such be used in assessing novelty of EP2. \F --- this is a tricky one. \textsc{EP1} is not \underline{\textbf{pending}} on its publication date; therefore it cannot be used under \textbf{Art. 54(3) EPC}! --- see also \textsc{\textbf{J5/81}} and page 85 of \textbf{\textit{Lobato}}.
        \item EP1 is prior art under Art. 54(3) EPC and may as such be used in assessing inventive step of EP2. \F
        \item EP1 is prior art under Art. 54(2) EPC and may as such be used in assessing novelty of EP3. \T
        \item EP1 is prior art under Art. 54(2) EPC and may as such be used in assessing inventive step of EP3. \T
        \item EP1 is prior art under Art. 54(3) EPC and may as such be used in assessing novelty of EP3. \F
        \item EP1 is prior art under Art. 54(3) EPC and may as such be used in assessing inventive step of EP3. \F
    \end{enumerate}

    \item % Question 4
    H5-09 \\
    A European patent application has been filed on 11 January 2023 as a first filing. On 13 July 2023 the applicant files a request for early publication of his application. The EPO informs him that the application will be published on Wednesday 13 September 2023. Today, 9 August 2023, the applicant realises that he has forgotten to claim priority from an earlier national application.
    
    \vspace{1em}
    Early publication request removes ability to add a new declaration of priority --- see \textbf{R. 52(4) EPC}. This is the case here. \\
    Nonetheless, there might still exist two potential avenues of introducing priority for this invention: \\
    \textbf{(i)} -- withdraw the EP app \textit{\textbf{before 5 weeks before scheduled publication}} and claim priority from the withdrawn EP app and the earlier national app; if it falls within the priority year. To withdraw and avoid publication, in practice, EPO will likely not publish it if you tell them up to 2 weeks before scheduled publication, so in this case, it may be possible that they agree since there is like 4 weeks left;\\
    \textbf{(ii)} -- the other thing is to try to withdraw the request for early pub (perhaps under \textbf{R. 139 EPC}). Again, it's not quite before 5 weeks, but the EPO may still accept it and then you still have 16 months under \textbf{R. EPC} to make a declaration of priority. 

    \item % Question 5
    H6-05 \qquad T/F question (Basic selection) \\
    Upon filing a European patent application, the applicant also paid the examination and designation fee. Early 2022, the EPO established an Extended European Search report, which mentions one prior art document that seems to take away the novelty of claim 1 as filed. The publication of the European search report is mentioned in the EP Bulletin on 21 September 2022. Shortly thereafter, the applicant receives an invitation from the EPO inviting him, among others, to indicate whether he wishes to proceed with the application.
    
    \vspace{1em}
    For each of the statements below, indicate whether the statement is true or false.
    
    \begin{enumerate}[label=(\alph*)]
        \item If the applicant does not respond to the invitation, the application will be deemed withdrawn and the examination fee will be refunded in full. \T
        \item If the applicant confirms that he wishes to proceed with the application but does not comment on the EESR nor files any amendments, the application will be deemed withdrawn and the examination fee will be refunded in full. \T -- through not commenting on a negative EESR, the application will be \underline{deemed to be withdrawn} under \textbf{R. 70(3) EPC}. Once it's deemed to be withdrawn, it effectively terminates and \textbf{Rfees 11(a)} provides for a \textit{full refund} of the examination fee before the substantive examination has begun. 
        \item If the applicant files comments and amendments to address the deficiencies mentioned in the EESR, but does not explicitly confirm that he wishes to proceed with the application, the application will be deemed withdrawn and the examination fee will be refunded in full. \F -- filing a response is the same as declaring proceeding with app.
        \item The applicant may, together with filing any amendments to address the deficiencies mentioned in the EESR, also file further amendments of his own volition. \T
    \end{enumerate}

    \item % Question 6
    H6-07 \qquad T/F question (Advanced selection) \\
    On 2 July 2021, an applicant filed a European patent application, claiming priority of a Czech patent application filed on 3 July 2020. The claims of the European patent application correspond to claims 3 - 16 of the Czech priority application, to account for prior art cited in the search report established by the Industrial Property Office of the Czech Republic. \\
    Last week, the Examining Division issued an invitation to supply a copy of the search report of the Czech priority application. \\
    Today is 24 Aug 2022.
    
    \vspace{1em}
    For each of the statements below, indicate whether the statement is true or false.
    
    \begin{enumerate}[label=(\alph*)]
        \item The applicant may request an extension of the period. The request will be granted. \F -- you can only request extensions for periods \textit{to be specified by the EPO}, i.e., \textbf{if there is an explicit time limit in the EPC, you \textit{cannot} extend it!!!}.
        \item The EPO will include a copy of the search report automatically, at no cost. \F -- generally true for \textsc{CZ}, but only from \textsc{OCT 2022}, so in this case false.
        \item The application is deemed to be withdrawn if the applicant fails to respond to the invitation. \T --- \textbf{Art. 124(2) EPC} \& \textbf{R. 70b(2) EPC}.
        \item The application is deemed to be withdrawn if the applicant fails to supply copies of the documents cited in the search report of the Czech priority application. \F -- the cited documents don't have to be supplied, the article just says it will be withdrawn if you \textit{fail to reply in due time}. 
        
    \end{enumerate}

 
    \item % Question 7
    H6-08 - T/F question (Basic selection) \\
    Five days after expiry of the period for payment of the examination fee, the applicant of a European patent application finds out that he failed to pay the examination fee. The failure was due to an isolated mistake in an otherwise well-functioning office.
    
    \vspace{1em}
    For each of the statements below, indicate whether the statement is true or false.
    
   
    The applicant can most likely successfully remedy the failure to pay the examination fee
    \begin{enumerate}[label=(\alph*)]
        \item ... by immediately requesting further processing in respect of the term for filing the request for examination. \T
        \item ... by requesting further processing in respect of the term for filing the request for examination, but the applicant needs to await a loss-of-rights communication before doing so. \F
        \item ... by immediately requesting re-establishment in respect of the term for filing the request for examination. \F
        \item ... by requesting re-establishment in respect of the term for filing the request for examination, but the applicant needs to await a loss-of-rights communication before doing so. \F
    \end{enumerate}
 
 
\end{enumerate}    
\section{November 24, 2025}

\begin{enumerate}[label=\textbf{Question \arabic*}]

    \item % Question 1
    H7-02 \quad T/F question (Basic selection) \\
    An applicant filed a European patent application with the EPO in Munich.
    
    \vspace{1em}
    For each of the statements below, indicate whether the statement is true or false. \\
    For the purposes of \textbf{Article 123(2) EPC},
    \begin{enumerate}[label=(\alph*)]
        \item ... the claims are part of the application as filed, if filed on the accorded filing date of the application. \T
        \item ... the drawings are part of the application as filed, if filed on the accorded filing date of the application. \T
        \item ... the abstract is part of the application as filed, if filed on the accorded filing date of the application. \F
        \item ... any document referred to in the description of the application as filed as "incorporated by reference" is in its entirety part of the application as filed, if the passage citing the referenced document is part of the description on the accorded filing date of the application. \F -- see \textbf{\textit{Lobato}} \s 617 and \textsc{GL H--IV 2.2.1}. Cross-referenced documents are \textit{prima facie} \textbf{not within the content of the app as filed}. There are 4 conditions from case law that need to be met for these to be included.
        \item ... the priority document is part of the application as filed, if filed on the accorded filing date of the application \F.
    \end{enumerate}
    
    \vspace{1em}
    For each of the statements below, indicate whether the statement is true or false.
    \begin{enumerate}[label=(\alph*), resume]
        \item If an application is filed in a non-EPO language and a translation into an official EPO-language is filed on the accorded filing date of the application, the translation into an official EPO-language is part of the application as filed for the purposes of Article 123(2) EPC. \F -- the original language is always the original disclosure.
        \item If an application is filed by reference to a previously filed patent application with a new set of claims filed on the accorded filing date of the application, the new set of claims is part of the application as filed for the purposes of Article 123(2) EPC. \T
        \item If a divisional application is filed, the parent application is part of the application as filed for the purposes of Article 123(2) EPC. \F
    \end{enumerate}

    \item % Question 2
    H7-05 \quad T/F question (Basic selection) \\
    A European patent application relates to a multi-layer laminated panel. The description includes several examples of different layered arrangements, one of these having an outer layer of polyethylene. During substantive examination, the applicant files an amendment of own motion in which the outer layer is altered to polypropylene.
    
    \vspace{1em}
    For each of the statements below, indicate whether the statement is true or false.
    
    \begin{enumerate}[label=(\alph*)]
        \item The amendment in which the outer layer is altered to polypropylene is not allowable. \T
        \item The amendment in which the outer layer is altered to polypropylene is not allowable, unless the applicant provides evidence that polypropylene is a commonly known alternative to polyethylene. \F
        \item The amendment in which the outer layer is altered to polypropylene would be allowable if the amendments would only have been made during opposition, provided the amendment is occasioned by a ground for opposition. \F
    \end{enumerate}

    \vspace{1em}
    \textit{Assume that the patent is granted with the amendment in which the outer layer is altered to polypropylene. An opposition is filed based on a lack of novelty of the arrangement with the outer layer being polypropylene.}
    
    \vspace{0.5em}
    For the statement below, indicate whether the statement is true or false.
    
    \begin{enumerate}[label=(\alph*), resume]
        \item The patent proprietor may amend the patent by altering the outer layer back to polyethylene. \F --- this is the \textbf{\textit{inescapable trap}} between \textbf{Art. 123(2) EPC} and \textbf{Art. 123(3) EPC}. Amending it after grant would extend the scope of protection. 
    \end{enumerate}

    \item % Question 3
    H7-19 - T/F question (Basic selection) \\
    In opposition proceedings before the EPO, the patentee replaced the granted claims, which were directed to originally disclosed: "Substance A" and "Mixture S containing substance A", by amended claims directed to the "Use of substance A in mixture S to achieve technical effect E". Technical effect E is a certain originally disclosed technical effect. The use of substance A in mixture S is already known for other purposes, but achieving technical effect E appears to be novel and inventive.
    
    \vspace{1em}
    For each of the statements below, indicate whether the statement is true or false.
    
    \vspace{0.5em}
    Such amendment is not allowable during opposition, as.
    \begin{enumerate}[label=(\alph*)]
        \item ... it violates Article 123(2) EPC. \F
        \item ... it violates Article 123(3) EPC. \F -- ``\textit{If a patent is so amended that a claim to a product (a physical entity) is replaced by a claim to the use of this product, the degree of protection is not extended, provided that the use claim in reality defines the use of a particular physical entity to achieve an effect and does not define such a use to produce a product}'' -- \textsc{G 2/88}.
        \item ... the EPC does not allow a change of claim category after grant. \F
        \item ... the amended claim is not new. \F
    \end{enumerate}

    \item % Question 4
    H7-24 \quad T/F question (Basic selection) \\
    The Examining Division intends to grant a European patent and has dispatched the communication under Rule 71(3) EPC. Shortly after having received the communication, the applicant finds an obvious error in the description of the patent application.
    
    \vspace{1em}
    For each of the statements below, indicate whether the statement is true or false.
    
    \begin{enumerate}[label=(\alph*)]
        \item It is too late to request correction after a Rule 71(3) EPC communication has been received. \F --  error can be corrected under \textbf{R.71(6) EPC }in response to \textbf{R. 71(3) }communication, or \textbf{R. 139 EPC} (until decision is handed over to internal postal service - \textsc{G 12/91})
        \item It is too late to request correction after the applicant responded to the Rule 71(3) EPC communication with his approval. \F -- you can file amendments up to the day before handing over decision to grant to EPO postal service.
    \end{enumerate}
    
    \vspace{1em}
    \textit{The obvious error resulted from a translation error when the European patent application was prepared starting from a Korean priority document. The European patent application was filed directly in English. The Korean priority document was filed together with the application.}
    
    \vspace{0.5em}
    For each of the statements below, indicate whether the statement is true or false.
    
    \begin{enumerate}[label=(\alph*), resume]
        \item The EPO will allow the correction, because the error as well as its correction are obvious in view of the priority document. \F -- it should be obvious from application as filed --- \textbf{priority document is not part of the application as filed}. Careful though, as prio. doc and abstract may be used as source of \textit{common general knowledge} for \textbf{R. 139 EPC} corrections; see \textsc{G 3/89} and \textit{page 247} of \textbf{\textit{Lobato}}.
        \item The EPO will allow the correction, because a translation may be brought into conformity with the original text at any time. \F -- it was filed directly in English.
    \end{enumerate}

    \item % Question 5
    H8-03 - T/F question (Basic selection) \\
    The applicant received a communication of the Examining Division, wherein the Examining Division informs the applicant of the French text in which it intends to grant. The communication provides a period of 4 months for the associated acts.
    
    \vspace{1em}
    For each of the statements below, indicate whether the statement is true or false.
    
    \begin{enumerate}[label=(\alph*)]
        \item The applicant can request extension of the period with two months, which he will get granted provided it is requested before the expiry of the original period. \F -- you cannot extend periods that are set in the EPC. \textbf{R. 71(3) EPC} has a specified period of 4 months.
        \item If the applicant fails to pay the fee for grant and publishing, he may request further processing. Such further processing shall comprise the payment of the fee for grant and publishing as well as a further processing fee at the level of 50\% of the fee for grant and publishing. \F -- in the event of \textit{late performance of acts under \textbf{R. 71(3) EPC}}, there is a flat fee of 300 \textsc{eur} provided for in \textbf{Rfees2(1).12}.
        \item If the applicant fails to pay the fee for grant and publishing and fails to file a translation of the claims into English and German, he may request further processing. Such further processing shall comprise the payment of the fee for grant and publishing as well as a further processing fee at the level of 50\% of the fee for grant and publishing, and a further processing fee at the flat amount specified in RFees 2(1).12 for the late filing of the translations of the claims. \F -- in the event of \textit{late performance of acts under \textbf{R. 71(3) EPC}}, there is a flat fee of 300 \textsc{eur} provided for in \textbf{Rfees2(1).12}.
        \item If the applicant fails to file a translation of the complete application into English and fails to file a translation of the claims into English and German, he may request further processing. Such further processing shall comprise the filing of the translations, as well as the payment of the associated further processing fee. \F -- you only need to translate the claims.
    \end{enumerate}

    \item % Question 6
    H8-08 - T/F question \\
    Upon filing a European patent application contained 25 claims. Upon an invitation by the EPO, the applicant has paid 5 claims fees for claims 19-23. \\
    During substantive examination, the applicant files an amended set comprising 20 claims. A national UK prior right is cited in the search report. The applicant files a separate set of claims with 23 claims for the UK. \\
    The applicant receives a communication from the EPO under Rule 71(3) EPC wherein the EPO expresses its intention to grant all 20 claims together with the separate set of 23 claims for the UK. The applicant wishes to get the application granted accordingly.
    
    \vspace{1em}
    For each of the statements below, indicate whether the statement is true or false.
    
    \begin{enumerate}[label=(\alph*)]
        \item In response to the Rule 71(3) EPC communication, ...
        \begin{enumerate}[label={(\alph{enumi}.\arabic*)}]
            \item ... ten claims fees still have to be paid. \F
            \item ... eight claims fees still have to be paid. \F
            \item ... five claims fees still have to be paid. \F
            \item ... three claims fees still have to be paid. \T -- fees are incurred for the claim set with the highest number of claims and they \textit{do not} include fees for claims that have already been paid. 
        \end{enumerate}
        
        \item If claims fees are only paid in respect of the claim set of 20 claims, ...
        \begin{enumerate}[label={(\alph{enumi}.\arabic*)}]
            \item ...the UK designation will be deemed withdrawn. \F
            \item ... the applicant will be invited to pay further claims fees within 1 month from the invitation. \F
            \item ... the applicant may request further processing in respect of the non-paid fees. \T
            \item ... the applicant may request further processing in respect of the acts of Rule 71(3) EPC by completing the payment of the missed claims fees and payment of the further processing fee referred to in RFees 2(1).12 as "fee for FP in the event of late performance of the acts required under Rule 71(3) EPC". \F -- The flat FP fee of RFees 2(1).12, for the missed R.71(3) acts does not cover the claims fees under R.71(4)
        \end{enumerate}
    \end{enumerate}

    \item % Question 7
    H8-10 - T/F question (Basic selection) \\
    An EP application was filed on 5 June 2019. In response to objections raised during examination, the applicant amended the claims from the initially filed 16 claims, for which claims fees were duly paid upon filing the EP application, to 20 claims. The Examining Division issued a Rule 71(3) EPC communication dated 7 November 2023 (communication IGRA1) wherein the Examining Division made amendments to two dependent claims of the latest text submitted by the applicant. The applicant agrees with the text, but not with the amendments made by the Examining Division. The applicant wants to proceed to grant as quick as possible, and therefore wishes to go to grant by cancelling the two dependent claims that the Examining Division had amended. The applicant considers to file a set of amended claims consisting of the 18 claims that the Examining Division did not amend. Today is 22 March 2024.
    
    \vspace{1em}
    What needs to be done to get the 18 remaining claims?
    
    \vspace{0.5em}
    For each of the statements below, indicate whether the statement is true or false.
    
    \begin{enumerate}[label=(\alph*)]
        \item The applicant can file the amended claims on or before the expiry of the time limit to respond to communication IGRA1 and wait for a second Rule 71(3) communication before filing the translations of the amended claims into the other two EPO languages and paying the fee for grant and publishing. \T
        \item The applicant must file translations of the amended claims into the other two EPO languages on or before the expiry of the time limit to respond to communication IGRA1. \F -- must not do anything if he doesn't agree with the final text.
        \item The applicant can, within the time limit to respond to communication IGRA1, file the amended claims, file translations of the amended claims into the other two EPO languages, pay the fee for grant and publishing, pay two claims fees, and expressly waive a further Rule 71(3) communication. \F --- waiving \textbf{R. 71(3) EPC} existed from 2015 but was abolished in 2020.
        \item The amendments may be presented as handwritten amendments on a copy of the set of the original claims, after which the Examining Division will send a new Rule 71(3) without any further intermediate communications. \F -- no handwritten amendments allowed.
        \item The applicant must pay two claims fees on or before the expiry of the time limit to respond to communication IGRA1. \F -- you don't have to pay if you don't agree with the final text.
        \item The time limit to respond to communication IGRA1 is 7 February 2024. \F -- it's 4 months.
    \end{enumerate}

    \vspace{1em}
    Optional question from EPAC 2023 \\
    On 16 November 2020, you filed a European patent application EP-A, which did not claim priority. Today, 12 October 2023, you receive a communication from the EPO entitled "Decision to grant a European patent pursuant to Article 97(1) EPC". This communication states that the mention of the grant will be published in the European Patent Bulletin on 8 November 2023. You wish to validate the patent in only two contracting states to the EPC.
    
    \vspace{1em}
    Which of the following statements is correct? (single choice)
    
    \begin{enumerate}[label=\Alph*.]
        \item The renewal fee for the fourth year must be paid centrally to the EPO. The due date is 16 November 2023. \F -- it's \textit{de ultimo ad ultimo} = 30 Nov 23.
        \item The renewal fee for the fourth year must be paid centrally to the EPO. The due date is 30 November 2023. \F -- renewal fee falls due after mention of grant. It needs to be paid to the two validated country national offices.
        \item The renewal fee for the fourth year must be paid separately in each state where the patent is validated. The renewal fee for the fourth year can be validly paid without any additional fee until 8 January 2024 at the latest. \T -- \textbf{Art. 141(2) EPC} provides for a 2-month ``safety'' period regardless of national law after mention of grant.
        \item The renewal fee for the fourth year must be paid separately in each state where the patent is validated. The renewal fee for the fourth year can be validly paid without any additional fee until 16 January 2024 at the latest. \F -- it's 8 \textsc{jan} 2024 at the latest.
    \end{enumerate}
\end{enumerate}
\section{December 01, 2025}

\begin{enumerate}[label=\textbf{Question \arabic*}]

    \item % Question 1
    C1-08 \\
    The applicant has identified that for some users it is easier to understand data when it is displayed as numerical values, whereas others might prefer a colour-coded display. He filed a European patent application directed to a device comprising a display and means for allowing the user to choice of the one or other manner of displaying the data, with the effect that choice of the manner allow a user of the device to easier understand the data that is presented.
    
    \vspace{1em}
    Is the device excluded from patentability under Art.52(2)(d) EPC as presentation of information? \\
    If not, can the device be considered inventive based on the mentioned effect?

\p Claim must be considered as a whole - GL G-II, 2; GL G-II, 3.7 \\
\p Invention may not be subject of EP application if it is non-technical as such - Art. 52(2)(b), (3) EPC \\
\p Once it is established that the claimed subject-matter as a whole is not excluded from patentability, it is examined in respect of the other requirements of patentability, in particular novelty and inventive step (G-I, 1) - GL G-II, 3.7. \\
\p As the claim is directed to a device (and also has a technical feature: the display), the claim as a whole is considered technical and not excluded under Art.52(2)(d)/(3) EPC as presentation of information as such. \\
\p The alleged effect depends on subjective interests or preferences of the user and is thus not considered to be a technical effect [GL G-II, 3.7]. \\
\p Therefore, the alleged effect cannot be used to support inventive step. - Art. 56 EPC \\


    \item % Question 2
    C2-03 (Basic selection)
    \begin{enumerate}[label=(\alph*)]
        \item What happens to the prior right effect of a conflicting earlier European patent application if that application was withdrawn before the date of publication? -- An \textbf{Art. 54(3) EPC} app. needs to be \textbf{pending} at publication to be used as prior right (see \textbf{J 5/81}).
        \item After publication of the earlier European patent application, the applicant withdraws the priority claim. What happens to the prior right effect? -- it remains unaffected.
    \end{enumerate} 

    \item % Question 3
    C2-05 (Basic selection) \\
    In October 2016 applicant P filed a European patent application EP1 as a first filing claiming subject-matter A. EP1 is subsequently published. In March 2017, applicant Q has filed a European application EP2 as a first filing claiming subject-matter A and independently subject-matter B. 
    
    \vspace{1em}
    Who has the right to a patent and for what subject-matter? \\
\p Applicant P has the right to subject-matter A. Applicant Q has right to sjm. B. Novelty is spoiled by EP1 under \textbf{Art. 54(3) EPC}, so Q cannot get protection for A. \\

    \item % Question 4
    C2-10 (Basic selection) \\
    An invention is displayed at the Exhibition "Future energy: Solutions for Tackling Mankind's Greatest Challenge" in Astana (KZ). The product was displayed for the first time on 9 September 2017, allowing members of the public to view and fully understand the invention. The inventor subsequently wants to obtain patent protection in Europe for his invention. \\
    If today is 26 February 2018, would this still be possible? And if so, what actions must be performed to secure his rights?
    
    \vspace{1em}
    "\textit{Future energy: Solutions for Tackling Mankind's Greatest Challenge}" was a recognised International Exhibition, see OJ 2016 A38, with the first day of the exhibition being 10.06.2017. \\
    So under \textbf{Art. 55(1)(b) EPC}, you have (non-extendable in any way) 6 months from \textit{\textbf{the first day of the display of the invention}} to file an EP app. So the deadline for filing is 09.03.2018 -- so in this case it is still possible. The Applicant must present an official supporting certificate from the expo within 4 months of the filing of the application under \textbf{R. 25 EPC}. The cert. has to be issued during the expo to make it legally viable. The full list of conditions is in  \textbf{R. 25 EPC}. For Euro-PCT entry, the certificate needs to be filed within the 31-month period of \textbf{R. 159(1) EPC}. \\  

    \item % Question 5
    C3-04 (Basic selection) \\
    A European patent application EP-A describes and claims a composition X1 comprising components A1 and B2. During the European search, a European patent application EP-B is found describing "a class of compositions X wherein X1 comprises a first component selected from a first list of twelve components A1, A2, ..., A12 and a second component selected from a second list of nine components B1, B2, ...., B9". No specific combinations are disclosed in EP-B. EP-B is a prior right with respect to EP-A.
    
    \vspace{1em}
    Is the claim novel in EP-A in view of EP-B? \T --- \textsc{ep-a} is novel in view of \textsc{ep-b} under EPO's ``\underline{\textbf{two-list principle}},'' i.e., the selection of the sepcific combination from two lists with more than 2 or 3 members each was not preferably disclosed in \textsc{ep-b}. See also \textsc{GL G--VI, 7. Selection of inventions}. \\

    \item % Question 6
    C3-05 (Basic selection) \\
    A European patent application EP-A describes and claims a compound comprising chlorine. During the European search, a European patent application EP-B is found describing compound wherein the compound comprises a halogen. EP-B is a prior right (= prior art under Art.54(3) EPC) with respect to EP-A. (\textit{Comment: Chlorine is a halogen.})
    
    \begin{enumerate}[label=(\alph*)]
        \item Is the claim novel in EP-A in view of EP-B? \T -- the disclosure of the \textit{genus} cannot spoil the novelty of the \textit{species}.
        \item Would your answer have been different if EP-B had enumerated all halogens? \T -- once chlorine had been explicitly disclosed in \textsc{ep-b}, it would be novelty-destroying for \textsc{ep-a}. Nice way to think about this is as a \underline{\textbf{selection of a single element from a single list}}. 
        \item Would your answer to the questions (a) and (b) have been different if EP-B was a normal prior art document? (= prior art under Art.54(2) EPC) \F
    \end{enumerate}

    \item % Question 7
    C4-15 (Basic selection) \\
    Are the following claims formulations allowable under the EPC as independent claims directed to a further therapeutic use?
    
    \begin{enumerate}[label=(\alph*)]
        \item Substance X for use in a method for the treatment of Y \T
        \item Substance X for use in the therapy of Y \T
        \item Substance X for use in a method of treating Y \T
        \item Substance X for use in a method of therapy of Y \T
        \item Substance X for use as an anti-inflammatory medicament (where the medicament is defined by its function) \T
        \item Composition comprising X for use in a method for the treatment of Y \T
    \end{enumerate}

    The ``\textit{for use}'' formulation is explained as allowable in \textsc{GL G--VI, 6.1.2 therapeutic uses pursuant to Art. 54(5)}. 
    
    \vspace{1em}
    Are the following claims formulations allowable under the EPC if the further therapeutic use is based on the use of the same product in a different treatment of the same disease?
    
    \begin{enumerate}[label=(\alph*), resume]
        \item Substance X for use in a method for the treatment of Y wherein the substance is administered topically \T -- OK because it specifies what the treatment is for.
        \item Substance X for use in the therapy of Y wherein the substance is administered three times daily \T -- OK because it specifies what the treatment is for.
        \item Substance X for use in a method of treating Y wherein the substance is administered three times daily at a dose in a range of 10-20 mg \T -- OK because it specifies what the treatment is for.
        \item Substance X for use in a method of therapy of Y wherein the substance is administered topically \T -- OK because it specifies what the treatment is for.
        \item Substance X for use as an anti-inflammatory medicament wherein the substance is administered topically (where the medicament is defined by its function) \T -- OK because it specifies what the substance is doing.
    \end{enumerate}

    \vspace{1em}
    Are the following claims formulations allowable under the EPC?
    
    \begin{enumerate}[label=(\alph*), resume]
        \item Substance X for a method for the treatment of Y \F
        \item Substance X in a method for the treatment of Y \F
        \item Medicament for topical treatment \F
        \item Anti-inflammatory medicament for topical treatment \F
        \item Pharmaceutical comprising substance X for topical treatment \F
        \item Composition comprising X as an anti-inflammatory agent \F
        \item Composition comprising X for use as an antifungal agent \T --- Not a further medical use within the meaning of \textbf{Art. 53(c) EPC}, because the claim does not define a specific medical use of the claimed product. It encompasses non-medical uses, because antifungal/antibacterial agents are also used in e.g., agriculture for treating plants.
        \item Composition comprising X for use as an antibacterial agent \T --- Not a further medical use within the meaning of \textbf{Art. 53(c) EPC}, because the claim does not define a specific medical use of the claimed product. It encompasses non-medical uses, because antifungal/antibacterial agents are also used in e.g., agriculture for treating plants.
    \end{enumerate}

\end{enumerate}    
\section{December 08, 2025}

\begin{enumerate}[label=\textbf{Question \arabic*}]

    \item % Question 1
    C5-18 \\
    A European patent application EP1 has a main claim directed to a specially-shaped paddle wheel used in a pump in which the motive power is provided by a hydraulic motor. EP1 validly claims priority of an earlier national application DE1. The search report of EP1 mentions a European patent application EP2 published before the filing date of DE1 disclosing the same paddle wheel used in a pump powered by an electric motor. 
    (\textit{Zsofia's comment: hydraulic motor and electric motor are equivalents.})
    \begin{itemize}[]
        \item Assess the patentability of EP1 in view of EP2. --- The claim is directed to a technical feature, so it is---in principle---patentable under \textbf{Art. 52 EPC}. The same paddle wheel is found in the prior art under \textbf{Art. 54(2) EPC} with a valid effective date before the priority date of \textsc{EP1}. The claim is novel over \textsc{EP2} because it discloses at least one feature that is different. However, since the hydraulic and electric motors are equivalent alternatives, the claim is certainly not inventive.
        \item Would your answer have been different if EP2 was published on the filing date of DE1? --- if it's published \textbf{on the filing date} it becomes an \textbf{Art. 54(3) EPC} document, i.e., it can only be taken into account for a novelty attack and not inventive step. Hence, \textsc{EP1} would have been patentable.
    \end{itemize}

    \item % Question 2
    C5-20 (Basic selection) \\
    For many, many years, books are printed using well-known ink X.
    Company C wished to have a more environmental-friendly ink than ink X. After an extensive R\&D program, Company C found that ink Y has similar printing properties when printing on paper as ink X but that the use of ink Y results surprisingly, in a very environmental-friendly printing on paper. Company C filed a European patent application describing their findings and experiments in detail, and claiming a book printed with ink Y.
    The search report from the EPO cites two prior documents:
    \begin{itemize}
        \item D1 describes a plastic packaging where information is printed on the packaging using ink Y. D1 describes that ink Y is particularly suited for printing on plastic packaging as ink Y attached very well to plastic;
        \item D2 describes printing of books using ink X and discusses in detail the color stability and ease of operation when using ink X for printing books.
    \end{itemize}
    
    \vspace{1em}
    Which is the closest prior art? Why? Argue why the claim is, or is not, inventive using the problem-solution approach. \\
    
    \p CPA is \textbf{D2} as it deals with printing books [our claim is directed to a product (book) printed using ink Y.] \textbf{D1} cannot really be CPA as the field of plastic packaging is too remote from printing books. \\
    
    \p Distinguishing feature over \textbf{D2} is the use of ink Y rather than ink X. So we are novel over \textbf{D2}. \\
    
    \p If the skilled person were to come across \textbf{D1} and its disclosed ink Y, he would see that \textbf{D1} discloses Y being particularly useful for printing on plastic. There is no direct pointer away from saying you cannot or should not try it on paper. The skilled person using common general knowledge that ink is used to print books on paper could have a reasonable expectation of success by trying to use ink Y in the process of \textbf{D2}. Nonetheless, the problem of \textbf{D1} is not to make more environmentally friendly printing of books but rather plastic packaging (a remote technical field). So we can assume the skilled person would not consult \textbf{D1} at all. \\
    
    \nt{Technical effect}{The use of ink Y results in very environmentally-friendly printing on paper}
    
    \nt{Objective technical problem}{How to achieve environmentally-friendly printing on paper when printing books?}
    
    So in light of the objective technical problem to be solved, one may argue that the invention is novel over the combination of \textbf{D1} applied to CPA \textbf{D2}. Skilled person has no incentive to consider \textbf{D1}, as the skilled person would not expect to find a solution to the OTP in a document relating to plastic packaging. \\
    But even if the skilled person would consult \textbf{D1}, the skilled person would not find a solution to his OTP, i.e., to achieve a more environmental friendly printing on paper when printing books, as \textbf{D1} is \underline{silent about environmental effects}; it only discloses color stability, and because \textbf{D1} also \underline{silent about printing on paper}; it only discloses printing on plastic. \\
    
    So, the skilled person would not recognize from \textbf{D1} that ink Y would solve his problem. There is no teaching in the prior art that \textit{would} (not simply \textit{could}, but \textit{would}) have prompted the skilled person, faced with the OTP, to modify or adapt the CPA while taking account of that teaching (of \textbf{D1}), thereby arriving at something falling within the terms of the claims, and is thus not achieving what the invention achieves.


    \item % Question 3
    C6-13 \\
    Consider whether there is unity of invention between the following claims in a European patent application and indicate whether each of the claims is independent or dependent:
    \begin{itemize}
        \item 1. Display with features A, B and C.
        \item 2. Display with features A and B.
        \item 3. Display according to claim 2 with added feature D.
    \end{itemize}

    All the claims have unity of invention as they are linked by the special technical features \textsc{a and b}. Claim 3 refers to claim 2 directly so it should be dependent on it. Claim 1, in turn, comprises \textbf{ALL the features} of claim 2 and also has another feature. So by \textbf{R. 43(4) EPC} it needs to be a \underline{dependent} claim off of claim 2. \\
    
    \s Hence, claim 1 and 3 should refer to claim 2, which should be made the first claim.
So, the order of claim 1 and 2 has to be reversed. \\

    
    
    
    \item % Question 4
    C6-09 (Basic selection) \\
    Consider whether there is unity of invention between the following claims in a European patent application. Assume that substance X is new and inventive.
    \begin{itemize}
        \item 1. Substance X.
        \item 2. Method for the preparation of substance X.
        \item 3. Use of substance X as a dye.
    \end{itemize}

    \p Yes, these are independent claims in three different categories (product; method; use), which are all linked by a single inventive concept and a special technical feature, which distinguishes the invention over the prior art (X) by being novel and inventive. \\
    
    \item % Question 5
    C6-25 (Basic selection) \\
    Consider whether the following claims will be accepted by the EPO:
    \begin{itemize}[]
        \item A cabinet substantially as hereinbefore described with reference to any one of the accompanying Figures 1 to 5. \F -- this is an \textbf{\textit{omnibus}} claim that only consists of references and will not be allowable under \textbf{R. 43(6) EPC}. 
        \item A cylinder head (12) for a motor engine (1) characterised in that ... \T -- looks OK. Has correct reference signs, ``configured for'' language and two-part form, with the preamble being clearly rather generic in describing the prior art.
        \item A (concrete) moulded brick characterised by ... \F -- only references allowed in parentheses. The alternative of being made of concrete needs to be specified in a dependent claim maybe. 
        \item Chemical substance comprising (meth)acrylate ... \T -- Bracketed expressions with a generally accepted meaning are allowable, e.g., "\textit{(meth)acrylate}," which is known as an abbreviation for "\textit{acrylate and methacrylate}" or other polymer-type language.
    \end{itemize}

\end{enumerate}
  
%%%%%%%%%%%%%J ANUARY 12 2026
    
    
    
    
    
% --- PREAMBLE REQUIREMENT ---
% Ensure you have  in your preamble

\section{January 12, 2026}

\begin{enumerate}[label=\textbf{Question \arabic*}, leftmargin=*]

    % --- QUESTION 1 ---
    \item H9-04 - T/F question (Basic selection) \\
    Company X has filed an opposition against EP1. Company X is the sole opponent. Company Y, a competitor, is interested in continuing the opposition.

    \vspace{0.5em}
    \textbf{(a) For each statement, indicate True or False:}
    \begin{itemize}[leftmargin=3.5em]
        \item[\textbf{(a.1)}] Company X can transfer the opposition to Company Y by contract, wherein (only) the opposition is sold to Company Y. The transfer has effect vis-a-vis the EPO only at the date when and to the extent that the documents providing evidence of such transfer have been produced. Company Y will thus take Company X's position in the opposition proceedings as from that date. \F -- Opposition status can only be transferred as part of assets in a wholesale takeover of a company and assets related to the opposition.
        \item[\textbf{(a.2)}] Company Y can join Company X to become joint opponents. Company X can subsequently withdraw from the opposition, leaving it to Company Y. \F --- it's too late to joint after 9-month opposition period.
        \item[\textbf{(a.3)}] If Company P has instituted infringement proceedings against Company Y, Company Y can join the opposition as intervener. If Company X subsequently withdraws, the opposition will continue with Company Y as sole opponent. \T
        \item[\textbf{(a.4)}] Company Y may submit third party observations while the opposition proceedings are pending. Company Y does not become an opponent by doing so, but if the submitted documents are prima facie relevant, the EPO shall continue the opposition at own motion, also after Company X has withdrawn. \T
    \end{itemize}

    \vspace{0.5em}
    \textit{Company X and Company Y agree to merge into new company Z while the opposition is pending.}

    \textbf{(b) For each statement, indicate True or False:}
    \begin{itemize}[leftmargin=3.5em]
        \item[\textbf{(b.1)}] As Company X no longer exists as a legal entity, no opponents remain. Opposition may however continue at the EPO own motion. \F -- Company \textsc{Z} simply takes over assets of company X; including opposition. 
        \item[\textbf{(b.2)}] As Company Z is the universal successor of Company X, Company Z will replace Company X as opponent. \T
    \end{itemize}

    \vspace{0.5em}
    \textit{Company Z appealed as sole appellant. Z and P reach an agreement where Z obtains a license and agrees to withdraw the opposition.}

    \textbf{(c) For each statement, indicate True or False:}
    \begin{itemize}[leftmargin=3.5em]
        \item[\textbf{(c.1)}] Upon receipt of the notice of withdrawal, the EPO will terminate the opposition appeal proceedings, as no Opponents remain. \T --- G7/91, G8/93 and G8/91. Appeal is terminated if sole appellant/opponent withdraws opposition. opposition \textit{appeal} is withdrawn; then the proceedings are automatically ended.
        \item[\textbf{(c.2)}] After receipt of the notice of withdrawal, the EPO may continue the opposition appeal proceedings at its own motion under R.84(2), last sentence. \F -- that rule is about oppositions; not opposition appeals.
    \end{itemize}

\vspace{1cm}
  

    % --- QUESTION 2 ---
    \item H9-06 - T/F question (Basic selection) \\
    A notice of opposition is filed on the last day and the fee is paid. However, the notice was not signed.

    \vspace{0.5em}
    \textbf{(a) Signature:}
    \begin{itemize}[leftmargin=3.5em]
        \item[\textbf{(a.1)}] A signature is required. If the signature is not provided on the same day, the opposition will be deemed not to have been filed. \F -- you will get invite from EPO within a period to be specified; usually \textbf{2m} -- see \ru{50(3)}.
        \item[\textbf{(a.2)}] A signature is required. The opponent will be invited to provide it within a period set by the EPO. If provided, the opposition is validly filed. \T
        \item[\textbf{(a.3)}] A signature is required. If not provided within the set period, the opposition is deemed inadmissible. \F -- it will be deemed to not have been filed at all.
    \end{itemize}

    \textbf{(b) Fee:}
    \begin{itemize}[leftmargin=3.5em]
        \item[\textbf{(b.1)}] If the deficiency is not timely rectified, the opposition fee will be refunded. \T
        \item[\textbf{(b.2)}] Irrespective of whether the deficiency is rectified or not, the opposition fee will not be refunded. \F
    \end{itemize}

  \vspace{1cm}

    % --- QUESTION 3 ---
    \item H9-12 - T/F question (Basic selection) \\
    A third party wishes to file an opposition against a European patent.

    \vspace{0.5em}
    \textbf{(a) An opposition may be filed...}
    \begin{itemize}[leftmargin=3.5em]
        \item[\textbf{(a.1)}] ... On the ground of lack of novelty over public prior use in Australia. \T
        \item[\textbf{(a.2)}] ... On the ground of non-patentability of methods for treatment of the animal body by surgery. \T
        \item[\textbf{(a.3)}] ... On the ground of lack of unity. \F
        \item[\textbf{(a.4)}] ... On the ground of lack of clarity of the claims. \F
        \item[\textbf{(a.5)}] ... against a patent granted from a divisional application on the ground of extension beyond the content of the parent. \T
    \end{itemize}

    \textbf{(b) Based on arguments...}
    \begin{itemize}[leftmargin=3.5em]
        \item[\textbf{(b.1)}] ... as to lack of novelty over an intermediate prior art document in combination with an invalid priority claim. \T
        \item[\textbf{(b.2)}] ... that the description is so unclear that the skilled person cannot carry it out. \T
        \item[\textbf{(b.3)}] ... that the applicant is not entitled to the patent. \F
    \end{itemize}

    \textbf{(c) An opposition is limited to...}
    \begin{itemize}[leftmargin=3.5em]
        \item[\textbf{(c.1)}] ... the grounds of opposition given in the notice of opposition. \F
        \item[\textbf{(c.2)}] ... the extent of opposition given in the notice of opposition. \T
        \item[\textbf{(c.3)}] ... the facts and evidence cited in the notice of opposition. \F
    \end{itemize}

   \vspace{1cm}

    % --- QUESTION 4 ---
    \item H9-14 - T/F question (Basic selection) \\
    Opposition filed on the last day and fee paid. The notice indicated grounds but contained no facts, arguments, or evidence to substantiate them.
    \begin{itemize}[label=\alph*)]
        \item The opposition will be considered filed. \T
        \item The opposition will be considered admissible. \F
    \end{itemize}

 \vspace{1cm}

    % --- QUESTION 5 ---
    \item H9-18 - T/F question (Basic selection) \\
    Grant published 29 June 2022. Patent includes Claim 1 (product) and Claim 2 (process), not linked by unity. Today is 17 March 2023.
    \begin{itemize}[label=\alph*)]
        \item A notice of opposition must be filed at the latest by 29 March 2023. \T
        \item If filed against claim 1 only, it can be extended to claim 2 after the expiry of the opposition period. \F
        \item The statement setting out the grounds for opposition must be filed at the latest by 29 May 2023. \F -- can only be remedied within \textbf{9m} opposition period.
        \item Lack of unity of the invention is a valid ground for opposition. \F
    \end{itemize}

  \vspace{1cm}

    % --- QUESTION 6 ---
    \item (based on) H9-23 \\
    Proprietor amends claims to a single independent claim by incorporating features of granted claim 2 into granted claim 1.
    \begin{itemize}[label=--]
        \item Is the opponent correct that the OD must always examine clarity of claims amended during proceedings (referring to G 9/91)? \F -- A claim amended during opposition proceedings is not subject to examination for compliance with \art{84} if it results from \underline{inserting a complete dependent claim as granted} into an independent claim.
        \item Can the Opposition Division object to the amended claim as lacking clarity? \F
    \end{itemize}

 \vspace{1cm}

    % --- QUESTION 7 ---
    \item (based on) H9-39 \\
    Proprietor requests limitation. Later, an opposition is filed.
    \begin{itemize}[label=\alph*)]
        \item What is the legal situation/priority of proceedings? --- EPO will terminate limitation proceedings and give the opposition precedence and reimburse the limitation fee. The basis for the opposition is the patent as granted. 
        \item Does the prior art document found by the Japanese patent office have to be filed when requesting limitation? \F -- no reasoning or prior art has to be filed with limitation request.
        \item Would the answer to (a) be different if the proprietor requested revocation? \T -- Revocation proceedings are not terminated when an opposition is filed. (\ru{93(2)} only refers to \textit{\textbf{limitation}}. Therefore, in the case of revocation proceedings, there is \underline{no precedence of opposition.} Revocation proceedings continue after an opposition is filed, and the case proceeds to opposition only if the request for revocation is deemed not to have been filed, is rejected as inadmissible or is withdrawn. Otherwise, if the patent is revoked, the opponent(s) will be informed of this situation and the opposition proceedings will be terminated.

    \end{itemize}

   \vspace{1cm}

    % --- QUESTION 8 ---
    \item H9-41 - T/F question (Basic selection) \\
    Grant: Nov 2020. Limitation requested: Jan 2022. Decision published: 25 May 2022.
    \begin{itemize}[label=\alph*)]
        \item The third party can file a 1967 publication as third party observations during limitation proceedings. \T
        \item The EPO will examine the request for limitation for novelty. \F
        \item The third party can file the publication as third party observations after the decision to limit is published. \T
        \item The third party may file a notice of opposition until 25 February 2023. \F -- what matters is the mention of grant; not anything to do with limitation decisions.
    \end{itemize}

\section{January 19, 2026}

\begin{enumerate}[label=\textbf{Question \arabic*}, leftmargin=*]

    % --- Question 1 ---
    \item H10-04 (Basic selection) \\
    \begin{enumerate}[label=(\alph*)]
        \item By when must an appeal be filed? --- within \textbf{2m} of notification of the decision (\art{108}).
        \item Is this period extendable? -- no; it's a stated period in the EPC so it cannot be extended. It can only be extended through \ru{134(1)}, i.e., weekends, etc.
        \item When must the appeal fee be paid? What is the amount? Any reduction possible when the opponent is a person referred to in Article 14(4) EPC? --- it must be paid when filing the notice of appeal (otherwise notice not considered to have been filed). 2015 EUR when filed by a natural person or entity referred to in \ru{7a(2)}. \underline{No fee reduction for \art{14(4)} persons}. Otherwise 2925 EUR.
        \item Can a late filing of the notice of appeal or a late payment of the appeal fee be remedied with further processing or re-establishment of rights? \underline{\textbf{FP} is specifically ruled out} for filing a notice appeal by \textbf{Art. 121(4) EPC}. Since \textbf{FP} is ruled out; \textbf{RE} is available. To be precise; \textbf{RE} is available for the Applicant/proprietor, and \textbf{RE} is available for the appellant when they miss the \textbf{4m} period for filing the statement of grounds of appeal.
    \end{enumerate}
\vspace{1cm}
    % --- Question 2 ---
    \item H10-12 (Basic selection) \\
    For each of the following situations, indicate whether the statement is true or false.

    \begin{enumerate}[label=\textbf{Case \arabic*}, leftmargin=3.5em]
        \item After a refusal, the applicant filed a main request and an auxiliary request with the letter of appeal. The main request is the same as the one refused (i.e. not amended). However, the auxiliary request corresponds to a suggestion made by the examining division and would thus be allowable. \\
        \textbf{1.} There must be an interlocutory revision. \F -- the MR will be rejected, so the case has to go to BoA.

        \item After a refusal for lack of novelty only, the applicant filed new claims with the letter of appeal. The new claims are clearly novel but not inventive. The question of inventive step had not been raised in the decision or in the previous procedure. \\
        \textbf{2.} There must be an interlocutory revision. \T -- the grounds for refusal have been overcome.

        \item After a refusal for lack of novelty over D1, the applicant filed a new claim 1 with the letter of appeal. New claim 1 filed which includes a feature from the description. This feature had not been previously discussed per se; however, it is clearly disclosed in D1. \\
        \textbf{3.} There must be an interlocutory revision. \F -- grounds for refusal have not been overcome. 

        \item After a refusal for lack of inventive step vis-a-vis D1 and D2, the applicant filed new claims with the letter of appeal. The new claims include a feature from the description. This feature had not been previously discussed, but is clearly disclosed in D1. \\
        \textbf{4.} There must be an interlocutory revision. \F -- the feature is not novel over D1, hence it cannot be a distinguishing feature contributing to an IS argument that could overcome the grounds for refusal. 

        \item After a refusal for lack of inventive step vis-a-vis D1 and D2, the applicant filed a new claim with the letter of appeal. The new claim includes five new features from the description. These features have not been previously discussed. The examiner notes that although these features are disclosed in D2, the lack-of-inventive-step argumentation would have to be revised. \\
        \textbf{5.} There must be an interlocutory revision. \T -- grounds for refusal have been overcome. The key critera there are that: \textit{1.} the text is no longer the same; and \textit{2.} substantial amendments have been made. 

        \item After a refusal for lack of clarity of the claims, the applicant included a wording that has already been suggested by the examiner. The new claims are ready for grant but the description needs to be adapted. \\
        \textbf{6.} There must be an interlocutory revision. \T -- grounds for refusal have been overcome.

        \item After a refusal, the applicant points out in the letter of appeal that the examining division has overlooked a request for oral proceedings. Neither claims nor description are amended. The examining division looks at the file and notes that the request for oral proceedings was indeed overlooked.
        \begin{enumerate}[label=\textbf{7.\arabic*}]
            \item There must be an interlocutory revision. \T
            \item Interlocutory revision must even be made if it is expected to result in a further refusal after oral proceedings have been held. \T
            \item If interlocutory revision is made, the appeal fee must be refunded. \T -- refusing a requested oral proceeding is a substantial procedural violation.
        \end{enumerate}

        \item On the date of filing, the application documents did not comply with the requirements of Rule 46 EPC. The application was subsequently refused pursuant Art. 90(5) EPC since the applicant filed the same poor-quality drawings in reply to the communication under Rule 58 EPC. When filing an appeal complying with the requirements of Art. 108, the applicant also files drawings of sufficient quality, thereby correcting the deficiency on which the refusal was based. \\
        \textbf{8.} There must be an interlocutory revision. \T -- grounds for refusal have been overcome.
    \end{enumerate}
\vspace{1cm}
    % --- Question 3 ---
    \item H10-17 \\
    The notice of appeal and the statement setting out the grounds of appeal are filed and the fee for appeal is paid in time. \\
    In which of the following cases is the notice of appeal inadmissible? Can the deficiency be remedied? What period applies? Will the appeal fee be refunded?
    \begin{enumerate}[label=(\alph*)]
        \item The notice of appeal is not signed. --- Must be signed. Deficiency to be corrected after EPO invitation (usually \textbf{2m} deadline -- and can be after expiry of period for appeal). If not rectified in time, the appeal will be \textbf{\textit{deemed to not have been filed}}. As there is then no legal basis for having paid the appeal fee, the fee will be refunded in full. The filed documents will be saved in the file as observations.
        \item The address of the appellant is incomplete. --- Identifying Appellant is sufficient to have successfully filed the appeal. The address issues is a deficiency to be corrected within \textbf{2m} after invitation from EPO. If not corrected on time, the appeal will be \textbf{\textit{deemed inadmissable}}. There is \underline{no fee refund for inadmissable appeals.} Can also use \ru{139} to correct incorrect Appellant information.          
        \item The notice of appeal is filed by the patent proprietor following a decision of the Opposition Division to maintain the patent based on the main request and in a form as amended by the proprietor. --- \textbf{cannot appeal a decision based on a granted main request}. The appeal will be deemed inadmissable and there will be no fee refund. 
        \item The grounds give a mere reference in general terms to passages in the literature showing the state of the art and to the Guidelines without making their inference sufficiently clear. --- this is not sufficient reason for appeal and will be deemed inadmissable and there will be no fee refund. Can be remedied before the end of the 4-month period only -- \textbf{R.101(1) EPC}.
    \end{enumerate}

    \vspace{1cm}
    % --- Question 4 ---
    \item H10-23 - T/F question (Basic selection) \\
    European patent application EP-Y was refused at oral proceedings before the examining division. The oral proceedings took place on 19 February 2021. Today, 1 March 2021, you received the written reasoned decision. \\
    For each of the statements below, indicate whether the statement is true or false:
    \begin{enumerate}[label=\Alph*.]
        \item A notice of appeal against the decision to refuse the patent application must be filed at the latest by 19 April 2021. \F -- period runs from the receipt of the written decision (but you can file the appeal once you hear the decision at oral proceedings before receiving the written decision). 
        \item The statement setting out the grounds of appeal and possibly amendments must be filed together with the notice of appeal.    \F   -- deadline for grounds is \textbf{4m} from the decision; notice of appeal is \textbf{2m} from the decision. 
        \item The examining division shall rectify its decision if it finds the appeal to be admissible and well-founded. \T -- this is the case of the \textbf{interlocutory revision}.
        \item The appeal fee is always reimbursed, if the Board of Appeal sets aside the decision of the examining division.      \F -- there needs to be a procedural violation. 
    \end{enumerate}
\vspace{1cm}
    % --- Question 5 ---
    \item H10-24 (Basic selection)
    \begin{enumerate}[label=(\alph*)]
        \item Can a decision of a Board of Appeal be challenged? \T -- one can file a \textbf{petition for review} (never another appeal) if they are a party adversely affected by the appeal decision. EBoA will only find it admissable if an objection in respect of a procedural defect was raised during the appeal proceedings but dismissed by the BoA.
        \item On which grounds may a petition for review be based? - the objection that was dismissed must be raised in such a form that it is recognizable for the BoA that an objection pursuant to \ru{106} is intended \textbf{and} the objection must be specific in that it is obvious which from the list of objections under \art{112a}, i.e,: member of BoA took part despite being exluded; a person not appointed to the BoA took part; violation of \art{113} happened; fundamental procedural defect happened; or a criminal act may have had an impact on the decision, e.g., falsified evidence. 
        \item Does a petition for review have suspensive effect? Why (not)? \F --  matter of extraordinary legal remedy, the filing of which does not affect the \textit{res judicata} quality of the decision.
        \item Are third parties protected at all if a successful petition for review results in the revival of lost patent protection? \art{112a(6)} protects third parties, giving them right to continue using the invention if they, in good faith, made effective and serious preparations to use the invention in the period between the BoA decision and the EBoA decision.
    \end{enumerate}

\end{enumerate}    
    
    
\end{enumerate} % End of LEVEL 1
\section{January 26, 2026}

\begin{enumerate}[label=\textbf{Question \arabic*}, leftmargin=*]

    % --- QUESTION 1 ---
    \item H11-04 - T/F question (Basic selection) \\
    For each of the statements below, indicate whether the statement is true or false.
    \begin{itemize}[leftmargin=3.5em]
        \item[\textbf{(a)}] Any third party may file observations concerning the patentability of an invention but, by that act, do not become a party to the proceedings. \T
        \item[\textbf{(b)}] Any party to the proceedings may file third party observations concerning the patentability of an invention. \F -- the provision of \art{115} is only for \textit{third} parties, i.e., those not already party to the proceedings.
        \item[\textbf{(c)}] If \textit{prima facie} relevant third party observations are filed after the applicant has given his approval under \ru{71(3)}, the Examining Division may not resume examination proceedings. \F -- the ED can do it if they consider the observations relevant up handover to postal.
        \item[\textbf{(d)}] Third party observations can be filed during opposition proceedings. \T
        \item[\textbf{(e)}] Third party observations can be filed during limitation proceedings. \T
        \item[\textbf{(f)}] Third party observations can be filed during central revocation proceedings. \T
    \end{itemize}

    \vspace{1cm}

    % --- QUESTION 2 ---
    \item H11-05 - T/F question (Basic selection) \\
    For each of the statements below, indicate whether the statement is true or false.
    \begin{itemize}[leftmargin=3.5em]
        \item[\textbf{(a)}] Oral proceedings may only be held at the request of any party to the proceedings. \F -- can also be held at motion of EPO.
        \item[\textbf{(b)}] A first request for oral proceedings by a party to the proceedings may not be refused before a decision to refuse the application is taken. \T
        \item[\textbf{(c)}] Oral proceedings before the Examining Division are, in principle, public. \F
        \item[\textbf{(d)}] Oral proceedings before the Opposition Division are, in principle, public. \T
    \end{itemize}

      \vspace{1cm}

    % --- QUESTION 3 ---
    \item H12-03 - T/F question (Basic selection) \\
    Today, 17 March 2023, an applicant wishes to file a European patent application claiming priority of his earlier national German application. The earlier application was filed on 27 April 2022 with the German patent office. Thursday 27 April 2023 is a national holiday in the Netherlands, but not in Germany.

    \vspace{0.5em}
    \textbf{The last day to validly file the European patent application, while validly claiming priority:}
    \begin{itemize}[leftmargin=3.5em]
        \item[\textbf{(a)}] ... with the EPO in Munich is 27 April 2023. \F -- the last valid day should be the next day after 27 Apr 2023 on which all the EPO offices are open; since claiming priority is the end of a \textbf{12m} \textit{period} specified in the EPC.
        \item[\textbf{(b)}] ... with the EPO in The Hague is 27 April 2023. \F
        \item[\textbf{(c)}] ... with the German patent office in Munich is 27 April 2023. \F -- this would be an option to keep a valid prio claim but it is not the \textit{last} valid day to do so because of option (\textbf{a}).
    \end{itemize}

    
    \vspace{0.5em}
    \textit{The applicant wishes to file another EP application claiming priority of a German application filed on 14 July 2022. Friday 14 July 2023 is a holiday in France, but a normal working day in Germany and the Netherlands.}

    \vspace{0.5em}
    \textbf{The last day to validly file the application, while validly claiming priority:}
    \begin{itemize}[leftmargin=3.5em]
        \item[\textbf{(d)}] .. with the EPO in Munich is 14 July 2023. \T
        \item[\textbf{(e)}] ... with the EPO in Munich is 15 July 2023. \F
        \item[\textbf{(f)}] ... with the EPO in Munich is 17 July 2023. \F
        \item[\textbf{(g)}] ... with the German patent office in Munich is 14 July 2023. \T
        \item[\textbf{(h)}] ... with the German patent office in Munich is ./17 July 2023. \F
        \item[\textbf{(i)}] ... with the national French Patent Office is 14 July 2023. \F
        \item[\textbf{(j)}] ... with the national French Patent Office is 15 July 2023. \F
        \item[\textbf{(k)}] ... with the national French Patent Office is 17 July 2023. \T
    \end{itemize}

    \vspace{0.5em}
    \textbf{General Priority Statements:}
    \begin{itemize}[leftmargin=3.5em]
        \item[\textbf{(l)}] The applicant may request an extension of the priority period with another 2 months. \F
        \item[\textbf{(m)}] Failure to meet the priority period may be remedied using further processing provided all requirements are fulfilled. \F
        \item[\textbf{(n)}] Failure to meet the priority period may be remedied using re-establishment provided all requirements are fulfilled. \T
        \item[\textbf{(o)}] Failure to meet the priority period may be remedied by adding a declaration of priority to the earlier application within 16 months from the earliest priority date claimed. \F -- \ru{52(2)} only applies when priority \textit{\textbf{was observed}} but the \textbf{\textit{declaration}} has not been filed.
    \end{itemize}

      \vspace{1cm}

    % --- QUESTION 4 ---
    \item H12-08 \\
    Representing Mr. A: Communication C1 (\art{94(3)}), dated 21 June 2019 (4m period). 21 Oct 2019: 2m extension granted. 18 Dec 2019: further 1m extension requested (rejected in C2 on 10 Jan 2020). Loss of rights noted in C3 on 22 Jan 2020.
    \begin{itemize}[leftmargin=3.5em]
        \item[\textbf{(a)}] What is the legal status today, 15 March 2020, and what would you do to save the application? -- Applicant has \textbf{2m} from date of communication about loss of rights (not the date of the decision) to request \textsc{fp}. Cheaper than requesting a decision and filing an appeal. 
        \item[\textbf{(b)}] Would an appeal against the rejection dated 10 January 2020 be admissible? -- you cannot appeal a decision to not grant an extension under \art{106(2)}. In any case; appeal would only work if there was a substantial procedural violation, which in this case there wasn't. The EPO doesn't have to grant two extensions in a row and Christmas holidays is not a \textit{force majeure} reason for a second extension. 
        \item[\textbf{(c)}] Can you have the finding as to the loss of rights reviewed by a Board of Appeal? \T -- but loss of rights comm. is \textbf{not a decision}. You need to request a decision within \textbf{2m} under \art{112(2)} in order to have it reviewed by BoA.
    \end{itemize}

    \vspace{1cm}

    % --- QUESTION 5 ---
    \item H13-11 - T/F question (Basic selection) \\
    NL1 filed 2 Feb 2018. Today, 4 April 2019, the applicant discovers he forgot to file an EP application claiming priority.
    \begin{itemize}[leftmargin=3.5em]
        \item[\textbf{(a)}] Filed today claiming priority without specific remedy if NL1 is not yet published. \F
        \item[\textbf{(b)}] Filed today and priority added until 2 June 2019 without FP or RE. \F
        \item[\textbf{(c)}] Filed today and priority added until 3 June 2019 without FP or RE. \F
        \item[\textbf{(d)}] Too late to file and request RE, as the last day for RE was 2 April 2019. \F
        \item[\textbf{(e)}] Only possible if filed today and requesting RE today. \T
    \end{itemize}

    \vspace{1cm}

    % --- QUESTION 6 ---
    \item H13-12 - T/F question (Basic selection) \\
    Applicant mistyped application number in \art{94(3)} response. EPO faxed/phoned before expiry. EPO employee said "OK" on phone, but loss of rights noted a month later.
    \begin{itemize}[leftmargin=3.5em]
        \item[\textbf{(a)}] Remedy by asking for a decision and arguing good faith/telephone call date. \T
        \item[\textbf{(b)}] Remedy by requesting free 2m extension of original 4m limit. \F
        \item[\textbf{(c)}] Remedy by payment of FP fee and filing correct response within 2m of C3. \T
        \item[\textbf{(d)}] Remedy by payment of RE fee and arguing all due care within 2m of C3. \F
    \end{itemize}

  \vspace{1cm}

    % --- QUESTION 7 ---
    \item H14-03 - T/F question (Basic selection) \\
    EP filed 12 March 2015; claims priority from P-NL filed 16 June 2014.
    \begin{itemize}[leftmargin=3.5em]
        \item[\textbf{(a)}] The "first" renewal fee is due on... \textbf{(a.1)} 31/03/16, \textbf{(a.2)} \underline{\textbf{31/03/17}}, \textbf{(a.3)} 31/03/18.
        \item[\textbf{(b)}] Last day for payment (no additional fee) for 3rd year is also \underline{\textbf{31/03/17}} since EPO is not closed on that day. 
        \item[\textbf{(c)}] Last day for payment (with additional fee) for 3rd year is... \textbf{(c.2)} 30/09/17, \textbf{(c.3)} \underline{\textbf{02/10/17}}.
        \item[\textbf{(d)}] Last day for payment (no additional fee) for 4th year is... \textbf{(d.2)} 31/03/18, \textbf{(d.4)} \underline{\textbf{03/04/18}}.
        \item[\textbf{(e)}] The last day for valid payment, with additional fee, of the renewal fee in respect of the fourth year is \underline{\textbf{01/10/18}} (\textit{de ultimo ad ultimo} from due date + \ru{134(1)}). 
    \end{itemize}

  \vspace{1cm}

    % --- QUESTION 8 ---
    \item H14-05 - T/F question (Basic selection)
    (a) A renewal fee has not been paid before or on the due date.   For each of the statements below, indicate whether the statement is true or false.   
    \begin{itemize}[leftmargin=3.5em]
    \item[ \textbf{(a.1)}]  A time limit of 2 months applies to remedy the deficiency; the remedy consists of payment of the missed renewal fee together with an additional 50\% \F -- 6 months
        \item[\textbf{(a.2)}] A time limit of 2 months from a communication informing the applicant of a loss-of-rights applies to remedy the deficiency; the remedy consists of payment of the missed renewal fee together with an additional 50\%. \F -- 6 months
        \item[\textbf{(a.3)}] A time limit of 6 months applies to remedy the deficiency; the remedy consists of payment of the missed renewal fee together with an additional 50\% \T -- \ru{51(2)}. 
        \item[\textbf{(a.4)}] A time limit of 6 months from a communication informing the applicant of a loss-of-rights applies to remedy the deficiency; the remedy consists of payment of the missed renewal fee together with an additional 50\% \F -- not from loss-of-rights comm.
        \item[\textbf{(a.5--8)}] If the missed renewal fee and the additional 50\% are not timely paid --- application is deemed withdrawn and you need to request \textbf{RE}.
        \item[\textbf{(b.1--2)}] The deficiency due to non-payment of the renewal fee takes effect on the expiry of the additional \textbf{6m} period for payment of the additional fee; NOT on the due date!
        \item[\textbf{(c.1--3)}] The renewal fee may be refunded if the fee has been \textit{not validly} paid, e.g., application was not pending on the due date, or if refund requested \textbf{in respect the 4th or later year substantially more than 3 months before the due date}. 
    \end{itemize}  
\end{enumerate}
\end{document}
