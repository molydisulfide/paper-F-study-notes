\documentclass{report}
\input{preamble}
\input{macros}
\input{letterfonts}
\usepackage{adforn}
\usepackage[T1]{fontenc}
\usepackage{lmodern}
\setlength\parindent{0pt}
\newcommand{\n}{\newline}
\newcommand{\p}{\adforn{61} \ }
\newcommand{\s}{\adforn{74} \ }



\begin{document}
	
	\thispagestyle{empty}
	\mytitleb{Paper F Notes}{Jakub Jadwiszczak}{jadwiszj@tcd.ie}{2026}
	\newpage% or \cleardoublepage
	\tableofcontents

\chapter{Introduction to the Patent Cooperation Treaty --- PCT}
\section{Historical Notes and Some Key Facts}

The \textbf{Patent Cooperation Treaty} (PCT) is an international patent law treaty, concluded in 1970. It provides a unified procedure for filing patent applications to protect inventions in each of its contracting states. \newline

\p   Signed on June 19, 1970 in Washington D.C.. Entered into force on January 24, 1978. The first application was filed on June 1, 1978.\newline

\p A patent application filed under the PCT is called an \underline{\textbf{international application}}, or \textbf{PCT application}. \n

\p No case law. No appeal body -- with exception of questions relating to unity of invention. \n

\p Rules have to cover all situations that might occur -- therefore are very detailed. \n

\p Contracting States have agreed to accept international filing date and the form and content of an international application having the effect of a national application, but they \textbf{have not limited the freedom to grant patents} to an International Authority (IA). This runs in contrast to the EPC --- for example --- where that freedom has been ceded.



\nt{Article 1}{\textbf{this says OK}}



\end{document}
