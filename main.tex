\documentclass{report}
\input{preamble}
\input{macros}
\input{letterfonts}
\usepackage{adforn}
\usepackage[T1]{fontenc}
\usepackage{lmodern}
\usepackage{mathtools}
\setlength\parindent{0pt}
\newcommand{\n}{\newline}
\newcommand{\p}{\adforn{61} \ }
\newcommand{\s}{\adforn{74} \ }



\begin{document}
	
	\thispagestyle{empty}
	\mytitleb{Paper F Notes}{Jakub Jadwiszczak}{jadwiszj@tcd.ie}{2026}
	\newpage% or \cleardoublepage
	\tableofcontents

\chapter{Introduction to the Patent Cooperation Treaty --- PCT}
\section{Historical Notes and Some Quick Facts}

The \textbf{Patent Cooperation Treaty} (PCT) is an international patent law treaty, concluded in 1970. It provides a unified procedure for filing patent applications to protect inventions in each of its contracting states. \newline

\p   Signed on June 19, 1970 in Washington D.C.. Entered into force on January 24, 1978. The first application was filed on June 1, 1978.\newline

\p A patent application filed under the PCT is called an \underline{\textbf{international application}}, or \textbf{PCT application}. \n

\p No case law. No appeal body -- with exception of questions relating to unity of invention. \n

\p Rules have to cover all situations that might occur -- therefore are very detailed. \n

\p Contracting States have agreed to accept international filing date and the form and content of an international application having the effect of a national application, but they \textbf{have not limited the freedom to grant patents} to an \textbf{International Authority (IA)}. This runs in contrast to the EPC --- for example --- where that freedom has been ceded. \n

\p The international phase includes filing + search + publication + (optional) non-binding examination. No decision to grant. \n

\nt{Definition}{\textbf{``National phase'' for the purposes of PCT should be understood to mean ``national'' \underline{or} ``regional phase.''}} 
\vspace{5mm}
\p Any signatory of the Paris Convention may accede to the PCT. \n

\p The national office (NO) must approve form and content of the application as approved in the international phase. \n

\p The cost of entering national phase is similar to a direct national application. Further search + examination may be carried out in the national phase, but the use of the \textbf{international search report (ISR)} and examination may result in a fee reduction at national phase. \n

\p NO will grant patent on initial application with the same effect as on a direct national application. \n

\p PCT is administered by the Word Intellectual Property Organization (WIPO), which is a UN agency. The International Bureau (IB) of WIPO carries out admin for the International Patent Cooperation Union that the Contracting States make up. \n

\p The international phase is carried out before International Authorities. There are > 120 of them and > 20 can act as \textbf{International Search Authorities (ISAs)} or \textbf{International Preliminary Examination Authorities (IPEAs)}.  \n

\p After the international phase, all NOs act as so-called \textbf{designated Offices} (dO) or \textbf{elected Offices} (eO) [the latter in \textit{Chapter II}.]

\section{The Paris Convention for the Protection of Industrial Property}

\begin{center}
 \href{https://en.wikisource.org/wiki/Paris_Convention_for_the_Protection_of_Industrial_Property_(1883)}{\textbf{Full Text}}

\end{center}


\p Signed on March 20, 1883. \n

\p Agreement between countries for mutual recognition of IP rights. Nationals of Signatory Countries enjoy the same rights in other States as nationals of those other States. \n

\p It secured \textbf{the right of priority} of a first filing in one State for subsequent applications in other States. \n

\p A \textbf{priority right} is a time-limited right triggered by the \textbf{first filing} of an application for a patent, industrial design, or trademark. It allows the applicant to file a \textbf{subsequent application} in another country that is effectively treated as if filed on the date of the first application, known as the \textbf{priority date}. To use this right, the applicant (\textbf{or their successor in title}) must \textbf{claim priority} in the subsequent application. \n

\p The priority period is \textbf{12 months} for patents and utility models (the \textbf{priority year}) and \textbf{6 months} for industrial designs and trademarks. In the original Paris Convention it was 6 months and 3 months, respectively. \n

\p For patents, this right is crucial because \textbf{novelty} and \textbf{inventive step} are assessed against prior art that was made public \textbf{before the priority date}, not the actual (later) filing date of the subsequent application. \n

\nt{Rationale}{\textbf{according to the EPO: \textit{``(...) basic purpose [of the right of priority] is to safeguard, for a limited period, the interests of a patent applicant in his endeavour to obtain international protection for his invention, thereby alleviating the negative consequences of the principle of territoriality in patent law.''}}} 
\vspace{5mm}

\p \textbf{\s Art. 19} of the Paris Convention allows for special agreements between Signatory Countries. The Paris Convention \underline{takes precedence} over laws of the Countries and over such special agreements. \n

\p EPC is a ``regional patent treaty'' in the sense of \textbf{Art. 19} of the Paris Convention, e.g., under \textbf{Art. 45 PCT}, a PCT applicant can obtain an \underline{\underline{EP patent}} by filing an initial international application. \n

\nt{Definition}{\textbf{\[
    \begin{dcases}
        \mathrm{Patent \ \underline{in} \ a \ state} & = \ \ \mathrm{\underline{national}}\\
        \mathrm{Patent \ \underline{for} \ a \ state} & = \ \ \mathrm{\underline{regional}} \\
    \end{dcases}
\]}} 
\vspace{5mm}



\end{document}
